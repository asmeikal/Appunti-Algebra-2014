
\documentclass[11pt,a4paper,draft,twoside]{report}
% caratteri grandi che non ci vedo!

\usepackage[italian]{babel}
% per date e ToC in italiano
\usepackage{nameref}
\usepackage{hyperref}

\usepackage{amsmath}        % matematica
\usepackage{amsthm}         % teoremi
\usepackage{amsfonts}       % font matematicosi
\usepackage{centernot}      % per semplificare
\usepackage{pgfplots}       % per i grafici
\pgfplotsset{compat=1.5}    % consigliata compatibilita'
                            % con la versione 1.5

\usepackage{fullpage}       % troppo margine normalmente
\usepackage{parskip}        % preferisco spazio fra i paragrafi
                            % all'indentazione sulla prima riga
\usepackage{fancyhdr}

\theoremstyle{plain}% default
\newtheorem{theorem}{Teorema}[section]
\newtheorem{lem}[theorem]{Lemma}
\newtheorem{prop}[theorem]{Proposizione}
\newtheorem{cor}[theorem]{Corollario}

\theoremstyle{definition}
\newtheorem{defn}{Definizione}[section]
\newtheorem{conj}{Congettura}[section]
\newtheorem{exmp}{Esempio}[section]

\theoremstyle{remark}
\newtheorem*{comm}{Commento}
\newtheorem*{oss}{Osservazione}
\newtheorem*{note}{Nota}
\newtheorem{caso}{Caso}

\pagestyle{fancyplain}
\fancyhead{} % clear all header fields
\fancyfoot{} % clear all footer fields
\fancyfoot[C]{\today}
\fancyfoot[LE,RO]{\thepage}
\renewcommand{\headrulewidth}{0pt}
\renewcommand{\footrulewidth}{0pt}

\usetikzlibrary{shapes,arrows,calc,fit,backgrounds}

\DeclareMathOperator{\covers}{<\!\cdot}
\newcommand{\dotcup}{\ensuremath{\mathaccent\cdot\cup}}

% per le liste non numeriche preferisco un bel dash
\renewcommand{\labelitemi}{$-$}

\makeatletter
\let\orgdescriptionlabel\descriptionlabel\renewcommand*{\descriptionlabel}[1]{%
\let\orglabel\label
\let\label\@gobble
\phantomsection
\edef\@currentlabel{#1}%
%\edef\@currentlabelname{#1}%
\let\label\orglabel
\orgdescriptionlabel{#1}%
}
\makeatother

\begin{document}

\title{Appunti di Algebra}
\author{Michele Laurenti}
\date{\today}

\maketitle

\newpage

\tableofcontents
\newpage

\part{Algebra}

\chapter{Nozioni e concetti fondamentali}

\begin{center}
\indent
\textit{Insiemi, relazioni, funzioni, numeri naturali e principio di induzione, calcolo combinatorio.}
\end{center}

\section{Insiemi}

L'insieme \`e un concetto primitivo senza definizione. Ma definiamo come si fa ad assegnare un insieme. Il modo pi\`u semplice \`e quello di elencarne gli elementi. 
\[
A = \left\{ \text{giallo}, \text{rosso}, \text{blu} \right\}
\]

Un altro modo \`e definendo una propriet\`a.
\[
A = \left\{ x \in X : P(x) \right\}
\]

Un insieme fondamentale \`e quello delle coppie ordinate. $A$, $B$ insiemi. $A \times B =$ insieme delle coppie ordinate.
\[
A \times B = \left\{ (a,b) : a \in A, b \in B \right\}
\]

C'\`e una differenza fondamentale fra insiemi e coppie ordinate.
\begin{gather*}
(a,b) = (c,d) \Leftrightarrow a = c \land b = d \\
\{a,b\} = \{c,d\} \Leftrightarrow (a = c \land b = d) \lor (a = d \land b = c)
\end{gather*}
Il concetto pu\`o essere generalizzato da coppie a terne, quaterne, e in generale $n$-uple.

Una relazione binaria tra due insiemi \`e un sottonisieme di $A \times B$. Il prodotto cartesiano non \`e associativo, ma i due insiemi $A \times B$ e $B \times A$ hanno la stessa cardinalit\`a.

Una funzione da un'insieme $A$ ad un insieme $B$ \`e una terna $(A, B, f)$ o $f : A \to B$, dove $A$ si dice Dominio, $B$ si dice Codominio, ed $f$ \`e una relazione ($f \subseteq A \times B$) t.c. $\forall \ a \in A \ \exists \text{ un solo } b \in B \text{ t.c. } (a,b) \in f$, ossia che $b = f(a)$.

Esiste una funzione da $A = \emptyset$ a $B \neq \emptyset$? S\`i, la funzione $(\emptyset, B, \emptyset)$. Viceversa, una funzione da $A \neq \emptyset$ a $B = \emptyset$ non esiste.

\[
R_1 = \left\{ (m, n) \in \mathbb{N}^2 : m = n^2 \right\}
\]
Non \`e una funzione. Perch\'e? \vspace{3cm}

\[
R_2 = \left\{ (r, s) \in \mathbb{R}^+ \times \mathbb{R} : r = s^2 \right\}
\]
Non \`e una funzione. Perch\'e?
\vspace{3cm}

\[
R_3 = \left\{ (r, s) \in \mathbb{R}^+ \times \mathbb{R}^+ : r = s^2 \right\}
\]
\`E una funzione. Perch\'e?
\vspace{3cm}

\begin{defn}
Due funzioni sono uguali solo se coincidono come terne.
\end{defn}

Per definire una funzione bisogna definire l'insieme del Dominio, definire l'insieme del Codominio, e una relazione che gode della propriet\`a che ogni elemento del dominio ha una sola immagine.

Sia $f : A \to B$ una funzione, $Im_f =$ immagine di $f$.
\[
Im_f = \left\{ b \in B : \exists \ a \in A \text{ t.c. } f(a) = b \right\}
\]

\[
f : \mathbb{N} \to \mathbb{N} \text{ t.c. } \forall \ n \in \mathbb{N} \ f(n) = 
\begin{cases} 
n - 1 & \text{se $n$ \`e pari} \\
n + 1 & \text{se $n$ \`e dispari}
\end{cases}
\]
Non \`e una funzione perch\'e lo zero non ha immagine.

\[
f : \mathbb{R} \to \mathbb{R} \text{ t.c. } \forall \ r \in \mathbb{R} \ f(r) = 
\begin{cases}
r^2 + 1 & \text{ se r $\leq$ 0 } \\
r & \text{ se r $\geq$ 0} 
\end{cases}
\]
Non \`e una funzione perch\'e lo zero ha due immagini.

\subsection{Propriet\`a delle funzioni}

\begin{description}
\item[Suriettiva\label{itm:suriettiva}] Ogni elemento di $B$ \`e immagine di un elemento di $A$, ossia $Im_f = B$.
\item[Iniettiva\label{itm:inettiva}] Due elementi diversi di $A$ non hanno la stessa immagine.
\end{description}

\`E pi\`u facile dimostrare l'iniettivit\`a di una funzione in un altro modo.

$f$ \`e iniettiva $\Leftrightarrow a, a' \in A $, se $ f(a) = f(a')$ allora $a = a'$.

Una funzione iniettiva e suriettiva \`e detta ``biunivoca''.

\[
f: \mathbb{R} \times \mathbb{R} \to \mathbb{R} \text{ t.c. } f(x,y) = \sqrt{2}x + y
\] 
Non \`e iniettiva, perch\'e $(1,0)$ e $(0,\sqrt{2})$ hanno la stessa immagine. \`E suriettiva.

\[
\forall \ r \in \mathbb{R} \ \exists \ (x,y) \in \mathbb{R} \times \mathbb{R} \text{ t.c. } f(x,y) = \sqrt{2} x + y = r
\]
Basta scegliere $(0,r)$ e $f(0,r) = r$.

Determiniamo le immagini della funzione per i Domini dati.

\begin{gather*}
A = \left\{ \left( a, - \sqrt{2}a\right) : a \in \mathbb{R}  \right\} \\
f(A) = \left\{ r \in \mathbb{R} : \exists \ (a, - \sqrt{2}a) \text{ t.c. } f(a, - \sqrt{2} a) = r \right\} = \left\{ 0 \right\}
\end{gather*}
\begin{gather*}
C = \left\{ (a, \sqrt{2} b) : a, b \in \mathbb{Z} \right\} \\
f(C) = \left\{ r \in \mathbb{R} : \exists \ (a, \sqrt{2} b) \text{ t.c. } f(a, \sqrt{2}b) = r \right\} \\
r = f(a, \sqrt{2} b) = \sqrt{2} a + \sqrt{2} b = \sqrt{2} (a + b) \\
(a+b) \in \mathbb{Z} \\
r \in \sqrt{2} \mathbb{Z} = \left\{z : z = \sqrt{2} t \text{ con } t \in \mathbb{Z} \right\}
\end{gather*}

L'insieme delle funzioni da $A$ in $B$ si indica con $B^A$. Questo insieme verifica tutte le propriet\`a delle potenze. Alla somma corrisponde l'unione, al prodotto corrisponde il prodotto cartesiano. Inoltre $|B| = m$, $|A| = n$, $|B^A| = m^n$.

\subsubsection{Composizione}

Date le funzioni $f : A \to B$ e $g : B \to C$, la composizione $g \circ f : A \to C$ \`e definita come:
\[
\forall \ a \in A , \ {g \circ f}(a) = g(f(a))
\]
L'operazione di composizione \`e una funzione $f : B^A \times C^B \to C^A$.

\begin{description}
  \item[Inversa destra] Data $f : X \to Y$, $g : Y \to X$ \`e un'inversa destra se $f \circ g = id_Y$ (funzione identit\`a)
  \item[Inversa sinistra] Data $f : X \to Y$, $g : Y \to X$ \`e un'inversa destra se $g \circ f = id_X$
\end{description}

\section{Relazioni su un insieme}

Una relazione su un insieme $A$ \`e un sottoinsieme $R \subseteq A \times A$. I tipi di relazioni su un insieme sono due:

\begin{description}
    \item[Relazioni d'ordine] Si indica con $\le$, $a \leq b$ indica che $a$ \`e in relazione con $b$. Propriet\`a:
    \begin{description}
        \item [Riflessiva] $\forall \ x \in A : x \leq x$
        \item [Antisimmetrica] $\forall \ x, y \in A \text{ se } x \leq y \text{ e }  y \leq x \Rightarrow x = y$
        \item [Transitiva] $\forall \ x, y, z \in A \text{ se } x \leq y \text{ e } y \leq z \Rightarrow x \leq z$
    \end{description}
    \item[Relazioni di equivalenza] Si indica con $\varepsilon$, $a \varepsilon b$ indica che $a$ \`e in relazione con $b$. Propriet\`a:
    \begin{description}
      \item [Riflessiva] (vedi sopra), posso scriverla anche come $\forall \ a \in A \ \exists \ x \in A : a \varepsilon x$
      \item [Simmetrica] $x \varepsilon y \Rightarrow y \varepsilon x$.
      \item [Transitiva] (vedi sopra)
    \end{description}
\end{description}

Le tre propriet\`a di una relazione d'ordine (riflessiva, antisimmetrica e transitiva) sono \textit{indipendenti}, ossia nessuna propriet\`a deriva dalle altre. Per dimostrarlo, forniamo degli esempi.
\begin{itemize}
  \item Antisimmetrica e transitiva, ma non riflessiva: $x \le y \Leftrightarrow x \neq y$. \textit{A me sembra sia invece simmetrica e transitiva. $x \le y \Leftrightarrow x < y$ \`e antisimmetrica e transitiva, essendo una relazione di ordine stretto.}
  \item Riflessiva e transitiva, ma non antisimmetrica: definita su $A = $ \{insieme di persone\}, $ \ a R b \Leftrightarrow a$ ha la stessa et\`a di $b$.
  \item Riflessiva e antisimmetrica, ma non transitiva: definita su $\mathbb{N}, x R y \Leftrightarrow x - y \le 2$.
\end{itemize}

\textbf{Esercizio:} trovare altri esempi.

\vspace{5cm}

Anche le propriet\`a delle relazioni d'equivalenza sono indipendenti. Potrebbe sembrare che la propriet\`a riflessiva sia conseguenza della propriet\`a simmetrica e della propriet\`a transitiva.
\begin{gather*}
x \varepsilon y \Rightarrow y \varepsilon x \text{ per la propriet\`a simmetrica} \\
x \varepsilon y , y \varepsilon x \Rightarrow x \varepsilon x \text{ per la propriet\`a transitiva}
\end{gather*}
L'errore \`e che $x \varepsilon y$ non \`e dato \textit{per ogni $x$}: la relazione \`e riflessiva se $\forall \ x \in A \ \exists \ y : x \varepsilon y \Leftrightarrow $ la relazione \`e riflessiva.

Altri esempi di relazioni con solo alcune delle propriet\`a delle relazioni d'ordine:
\begin{itemize}
  \item Relazione transitiva e simmetrica, ma non riflessiva:
  \[
  (m, n) \in \mathbb{N} \times \mathbb{N}, \ (m,n) R (p, q) \Leftrightarrow \frac{m}{p} = \frac{n}{q}
  \]
  La coppia $(m,0)$ non \`e in relazione con nessuno se $m \neq 0$.
  \item Relazione riflessiva e transitiva, ma non simmetrica: una qualsiasi relazione d'ordine (essendo tutte antisimmetriche).
  \item Relazione riflessiva, simmetrica ma non transitiva: dato l'insieme delle rette nello spazio euclideo, $r R s \Leftrightarrow r $ \`e complanare a $s$ ($r // s$ - $r$ parallela ad $s$ - oppure $r \cap s = \{ P \}$ - si incontrano in un punto).
\end{itemize}

\textbf{Esercizio}: verificare le propriet\`a delle seguenti relazioni.
\begin{itemize}
  \item Parallelismo tra rette dello spazio \vspace{3cm}
  \item $\mathbb{N} \times \mathbb{N}, (m, n) \rho (p, q) \Leftrightarrow (m + q) = (p + n)$, ossia $m - n = p - q$. \vspace{3cm}
  \item $a, b \in \mathbb{Z}, a \equiv_n b \Leftrightarrow n | (a - b) \Leftrightarrow (a-b) = k n$ (congruenza modulo $n \in \mathbb{N}, n \ge 2$) \vspace{3cm}
  \item Data una funzione $f: A \to B $, $ \varepsilon_f = $ \`e relazione di equivalenza individuata da $f$, detta ``nucleo di $f$''. \`E una relazione su $A$ t.c. $a, a' \in A, a \varepsilon_f a' \Leftrightarrow f(a) = f(a')$. \vspace{3cm}
\end{itemize}

Data una relazione $R \subseteq A \times B$, possiamo definire $R^{\ast} = $ relazione duale di $R$. Due elementi $(a,b) \in A \times B$ sono $a \ R^{\ast} \ b \Leftrightarrow b \ R \ a$. Ad esempio, se $A = $ insieme di persone e $R = a $ figlio di $b$, la relazione duale \`e $R^{\ast} = b $ \`e genitore di $a$.

La duale di una relazione d'ordine \`e ancora una relazione d'ordine, e viene tipicamente indicata con $\geq$.

La coppia $(P, \leq)$ data dall'insieme $P \neq \emptyset$ con una relazione d'ordine $\le$ si dice insieme parzialmente ordinato.

Se $\forall \ x, y \in P$ e si ha che $x \leq y \lor y \leq x$, $(P, \leq)$ si dice linearmente ordinato (o totalmente ordinato).

L'insieme dei numeri naturali $\mathbb{N}$ con la relazione d'ordine naturale $(\mathbb{N}, \leq)$ \`e totalmente ordinato. Ma possiamo prendere anche l'insieme dei numeri naturali con la relazione di divisibilit\`a $(\mathbb{N}, |)$, ossia $m | n \Leftrightarrow m \text{ divide } n \Leftrightarrow \exists \ k \in \mathbb{N}$ t.c. $n = k \ m$. Rispetto a questa relazione d'ordine, $\mathbb{N}$ non \`e totalmente ordinato.

L'insieme delle parti di un insieme $\Gamma$ qualunque con la relazione ``\`e sottoinsieme di'' $\left(\mathbb{P}(\Gamma), \subseteq \right)$ \`e un insieme parzialmente ordinato. 

Come si costruisce in modo naturale una relazione su un prodotto cartesiano? Ad esempio, prendo la relazione d'ordine naturale sui reali $(\mathbb{R}, \leq)$ e $(\mathbb{R}^2, \leq)$, ho $(a,b) \leq (c,d) \Leftrightarrow a \leq c \land b \leq d$.

\[
B = \{ n \in \mathbb{N} : n = 2^r 3^s, r, s \in \mathbb{N}\}
\]
Definisco $\rho$ su $B$ come $2^r 3^s \leq 2^t 3^u \Leftrightarrow r \leq t \land s \leq u$. \`E una relazione d'ordine parziale. Perch\'e?
\vspace{3cm}

\begin{defn}[Ordine naturale sul prodotto cartesiano]
Dati due insiemi parzialmente ordinati $(P_1, \le_1)$ e $(P_2, \le_2)$, posso definire naturalmente una relazione d'ordine sul prodotto dei due insiemi $(P_1 \times P_2, \le)$ come:
\[
(x_1, x_2) \le (y_1, y_2) \Leftrightarrow x_1 \le_1 y_1 \land x_2 \le_2 y_2
\]
Si pu\`o generalizzare anche a $(P^n, \le)$, prendendo le $n$-uple al posto delle coppie.
\end{defn}

\begin{defn}[Ordine fra funzioni]\label{ordine_funzioni}
Se prendo un insieme parzialmente ordinato $(P, \le)$ e un insieme qualsiasi $A$, posso definire una relazione su $(P^A, \le)$ (ricordando che $P^A$ \`e l'insieme di tutte le funzioni da $A$ a $P$). Come faccio a dire che date $f, g \in P^A \ f \le g$? Se $\forall \ a \in A \ f(a) \le g(a)$.
\end{defn}

\subsection{Diagramma di Hasse}

Dato un insieme e una relazione di ordine parziale su di esso $(A, \leq)$, rappresento gli elementi di $A$ con dei punti sul piano. Posiziono il punto $x \in A$ pi\`u in basso di $y \in A$ se $x \leq y$. Congiungo con un segmento gli elementi $x, y$ se $y$ ``copre'' $x$, ossia se l'intervallo individuato dagli estremi $x, y = [x, y] = \left \{ z \in A : x \leq z \leq y \right \}$ contiene solo $x$ ed $y$. Si dice $y$ ``copre'' $x$ o $x$ ``\`e coperto'' da $y$ e si indica con $x \covers \ y$.

\begin{figure}[ht]
\centering
\begin{tikzpicture}
  \node (A) {$A$};
  \node (13) [below of=A, node distance=2cm] {$\{1,3\}$};
  \node (12) [left of=13, node distance=2cm] {$\{1,2\}$};
  \node (23) [right of=13, node distance=2cm] {$\{2,3\}$};
  \node (1) [below of=12, node distance=2cm] {$\{1\}$};
  \node (2) [below of=13, node distance=2cm] {$\{2\}$};
  \node (3) [below of=23, node distance=2cm] {$\{3\}$};
  \node (O) [below of=2, node distance=2cm] {$\emptyset$};
  \path[-]  (O) edge node {} (1)
            (O) edge node {} (2)
            (O) edge node {} (3)
            (1) edge node {} (12)
            (1) edge node {} (13)
            (2) edge node {} (12)
            (2) edge node {} (23)
            (3) edge node {} (13)
            (3) edge node {} (23)
            (12) edge node {} (A)
            (13) edge node {} (A)
            (23) edge node {} (A)
            ;
\end{tikzpicture}
\caption{Diagramma di Hasse della relazione $(\mathbb{P}(A), \subseteq)$ sull'insieme $A = \{1, 2, 3\}$}
\end{figure}

\begin{figure}[ht]
\centering
\begin{tikzpicture}
  \node (24) {24};
  \node (8) [below left of=24, node distance=2cm] {8};
  \node (12) [below right of=24, node distance=2cm] {12};
  \node (4) [below left of=12, node distance=2cm] {4};
  \node (6) [below right of=12, node distance=2cm] {6};
  \node (2) [below left of=6, node distance=2cm] {2};
  \node (3) [below right of=6, node distance=2cm] {3};
  \node (1) [below left of=3, node distance=2cm] {1};
  \path[-]  (24) edge node {} (8)
            (24) edge node {} (12)
            (8) edge node {} (4)
            (12) edge node {} (4)
            (12) edge node {} (6)
            (4) edge node {} (2)
            (6) edge node {} (2)
            (6) edge node {} (3)
            (2) edge node {} (1)
            (3) edge node {} (1)
            ;
\end{tikzpicture}
\caption{Diagramma di Hasse della relazione $(A,|)$ sull'insieme $A = \{n \in \mathbb{N} : n | 24\} = \{ 1, 2, 3, 4, 6, 8, 12, 24 \} $}
\end{figure}

\begin{theorem}
Ogni insieme \`e suscettibile di un ordine totale, ossia si pu\`o totalmente ordinare.
\end{theorem}

\textbf{Esercizio:} determinare due ordini diversi su $\mathbb{Q}$.

\vspace{5cm}

\subsection{INF e SUP}

\begin{defn}[INF]
Dato un insieme $(P, \leq)$ parzialmente ordinato, e due elementi $x,y \in P$, si dice INF di $x$ e $y$ l'elemento $\inf(x,y)$ tale che:
\begin{description}
    \item[INF1\label{itm:inf1}] $\inf(x,y) \leq x $ e $ \inf(x,y) \leq y$
    \item[INF2\label{itm:inf2}] $z \in P$, se $ z \leq x, y \Rightarrow z \leq \inf(x,y)$
\end{description}
\end{defn}
\begin{defn}[SUP]
Analogamente possiamo definire il SUP di $x,y$, un elemento di P che si indica con $\sup(x,y)$ tale che:
\begin{description}
    \item[SUP1\label{itm:sup1}] $\sup(x,y) \geq x $ e $ \sup(x,y) \geq y$
    \item[SUP2\label{itm:sup2}] $z \in P$, se $ z \geq x, y \Rightarrow z \geq \sup(x,y)$
\end{description}
\end{defn}

Ossia, $\inf(x,y)$ \`e il pi\`u grande dei minoranti di $x,y$ e $\sup(x,y)$ \`e il pi\`u piccolo di tutti i maggioranti di $x, y$.

Prendiamo l'insieme $\Gamma$, il suo insieme delle parti e la relazione ``\`e sottoinsieme di'' $(P(\Gamma),\subseteq)$, e due sottoinsiemi $A, B \subseteq \Gamma$, l'$\inf(A,B) = A \cap B$. Il $\sup(A,B) = A \cup B$.

Prendiamo $(\mathbb{N},|)$ e due elementi $m, n$. L'$\inf(m,n) = \operatorname{MCD}(m,n)$, mentre il $\sup(m,n) = \operatorname{mcm}(m,n)$.

\section{Reticoli}
\begin{defn}
L'insieme parzialmente ordinato $(L, \leq)$ \`e un reticolo se $\forall \ x, y \in L $ esiste $\inf(x,y)$ ed esiste $\sup(x,y)$.
\end{defn}

Posso interpretare i reticoli anche come strutture algebriche. Si possono definire due operazioni su $L$. 

\begin{defn}[Operazione di $\inf$]
Indicata con $\wedge: L \times L \to L$, definita come $ \forall \ (x,y) \in L \times L , \ x \wedge y = \inf(x,y) $.
\end{defn}
\begin{defn}[Operazione di $\sup$]
L'operazione di $\sup $, indicata con $ \vee: L \times L \to L$, definita come $\forall \ (x,y) \in L \times L , \ x \vee y = \sup(x,y)$.
\end{defn}

Ho definito quindi la struttura algebrica $(L, \wedge, \vee)$. Le due operazioni hanno le seguenti propriet\`a, dette \label{proprieta_dei_reticoli} \textbf{propriet\`a dei reticoli}:
\begin{description}
    \item[R1\label{itm:r1}] Idempotenza: $x \wedge x = x ; \ x \vee x = x$.
    \item [R2\label{itm:r2}] Commutativa: $x \wedge y = y \wedge x ; \ x \vee y = y \vee x$.
    \item [R3\label{itm:r3}] Associativa: $x \wedge ( y \wedge z) = (x \wedge y) \wedge z; \ x \vee ( y \vee z) = (x \vee y) \vee z$.
    \item [R4\label{itm:r4}] Assorbimento: $x \wedge (x \vee y) = x ; \ x \vee ( x \wedge y) = x$.
\end{description}
Sono le propriet\`a classiche dell'unione e dell'intersezione.

\begin{proof}[Dimostrazione della propriet\`a \ref{itm:r3}]
Equivale a dimostrare, per doppia inclusione, che $x \wedge ( y \wedge z) \leq (x \wedge y) \wedge z$ e che $(x \wedge y) \wedge z \leq x \wedge ( y \wedge z)$. La prima parte equivale a dire che
\[
x \wedge (y \wedge z) \leq 
\begin{cases}
(x \wedge y) \\
 z
 \end{cases}
\]
Il primo caso equivale a dimostrare che
\[
x \wedge (y \wedge z) \leq x, y \Rightarrow x \wedge (y \wedge z) \leq x \wedge y
\]
Il secondo caso si dimostra subito perch\'e $y \wedge z \leq z, y \Rightarrow y \wedge z \leq z$ \`e vero per definizione di $\inf$. 
\end{proof}

\begin{theorem}
Data una struttura algebrica $(L, \wedge, \vee)$ verificante le quattro propriet\`a dei reticoli, \`e possibile definire su $L$ una relazione d'ordine $\le$ t.c. $ x \wedge y = \inf(x,y) $ e $ x \vee y = \sup(x,y) $ in $ (L, \le)$, ossia $x \le y \Leftrightarrow x \wedge y = x $ (o anche $ x \vee y = y$).
\end{theorem}
\begin{proof}
Dimostriamo che \`e una relazione d'ordine.
\begin{itemize}
    \item Propriet\`a riflessiva: $x \le x$ per \ref{itm:r1}.
    \item Antisimmetrica: $\forall \ x, y \in A \text{ se } x \leq y \text{ e }  y \leq x \Rightarrow x = y$. Dimostrazione: $x \le y \Rightarrow x \wedge y = x$, e $y \le x \Rightarrow y \wedge x = y$. Per \ref{itm:r2} $x = x \wedge y = y \wedge x = y$, per cui $x = y$.
    \item Transitiva: $\forall \ x, y, z \in A \text{ se } x \leq y \land y \leq z \Rightarrow x \leq y$. Dimostrazione: $x \le y \Rightarrow x \wedge y = x$, $y \le z \Rightarrow y \wedge z = y$, da cui $x \le z$. Devo quindi dimostrare che $x \wedge z = x$, quindi che $(x \wedge y) \wedge z = x$. Per \ref{itm:r3} $(x \wedge y) \wedge z = x \wedge (y \wedge z) = x \wedge y = x$. 
\end{itemize}
Rimane da dimostrare che $x \wedge y = \inf(x,y)$. Devo dimostrare due cose:
\begin{enumerate}
    \item $x \wedge y \le x, y$. Infatti $(x \wedge y) \wedge x = (x \wedge y) \Rightarrow (x \wedge y) \le x$, e analogamente $x \wedge y \le y$.
    \item $z \le x, y \Rightarrow z \le (x \wedge y)$. La tesi mi dice che $z \wedge x = z$ e che $z \wedge y = z$. Devo dimostrare quindi $z \wedge (x \wedge z) = z \wedge z = z$.
\end{enumerate}
\end{proof}

\begin{lem}\label{dual}
$ x \le y \Leftrightarrow x \vee y = y $
\end{lem}
\begin{proof}[Dimostrazione del lemma \ref{dual}]
Sostituendo in $x \wedge y = x$ in $ x \vee y $ otteniamo $ (x \wedge y) \vee y = y $, per la propriet\`a \ref{itm:r4}.
\end{proof}
Per dimostrare $x \vee y = \sup(x,y)$ basta sfruttare il lemma \ref{dual}, ripercorrendo le stesse tappe e sostituendo $\sup$ al posto di $\inf$.

I reticoli possono quindi essere visti come strutture algebriche in cui le operazioni sono l'$\inf$ e il $\sup$, o come insiemi parzialmente ordinati.

Ci sono altre due propriet\`a dei reticoli:
\begin{description}
  \item [R5\label{itm:isotoniche}] Le relazioni sono \textit{isotoniche}.
  \[
  x \le y, z \in L \Rightarrow 
  \begin{cases}
  x \wedge z \le y \wedge z \\ 
  x \vee z \le y \vee z
  \end{cases}
  \]
  \item [R6\label{itm:disuguaglianza_distributiva}] Disuguaglianza distributiva. In ogni reticolo $(L, \le) \ \forall \ x, y, z \in L $ si ha:
  \[
  x \vee (y \wedge z) \le (x \vee y) \wedge (x \vee z)
  \]
\end{description}

\begin{proof}[Dimostrazione di \ref{itm:isotoniche}]
Devo dimostrare che $ (x \wedge z) \wedge (y \wedge z) = x \wedge z $.
Per ipotesi $x \le y \Rightarrow x \wedge y = x$.
\begin{multline*}
(x \wedge z) \wedge (y \wedge z) = \\
(x \wedge z) \wedge (z \wedge y) = \\
x \wedge (z \wedge z) \wedge y = \\
(x \wedge z) \wedge y = \\
(x \wedge y) \wedge z = \\
 x \wedge z
\end{multline*}
\end{proof}
\begin{proof}[Dimostrazione di \ref{itm:disuguaglianza_distributiva}]
Per definizione di $\inf$ bisogna dimostrare:
\[
x \vee (y \wedge z) \le 
\begin{cases}
(x \vee y) \\
(x \vee z)
\end{cases}
\]
Poich\`e $\vee$ \`e un'operazione isotonica:
\[
(y \wedge z) \le y \Rightarrow x \vee (y \wedge z) \le x \vee y
\]
\end{proof}

\subsection{Teorema di dualit\`a}
Data una stringa $E(\wedge, \vee)$ contenente gli elementi del reticolo, $\wedge, \vee, (, )$, posso creare la sua stringa duale $E^{\ast} (\vee, \wedge)$ in cui ogni $\wedge$ \`e sostituita da $\vee$, e le relazioni d'ordine sono scambiate.
\begin{gather*}
(x \wedge y) \vee y = E(\wedge, \vee) \\
(x \vee y) \wedge y = E^{\ast} (\vee, \wedge)
\end{gather*}
Vediamo che le propriet\`a dei reticoli sono formule duali.

Perch\`e bisogna scambiare le relazioni d'ordine? Dato un reticolo $(L, \le)$ e il suo duale $(L, \ge)$, il $\sup^{\ast} (x,y) = \inf(x,y)$, e l'$\inf^{\ast}(x,y) = \sup(x,y)$.

\begin{theorem}
Se in un reticolo $(L, \le)$ \`e vero un enunciato $\xi$ relativo a stringhe del tipo $E(\wedge, \vee) \Rightarrow $ \`e vero l'enunciato duale $\xi^{\ast}$ di $\xi$ che si ottiene da $\xi$ sostituendo ad ogni stringa $E(\wedge, \vee)$ la stringa $E^{\ast}(\vee, \wedge)$, e se $E_i (\wedge, \vee) \le E_2 (\wedge, \vee)$, allora $E_i^{\ast} (\vee, \wedge) \ge E_2^{\ast} (\vee, \wedge)$.
\end{theorem}
Ogni volta che dimostro un teorema sui reticoli, \`e vero anche il teorema duale. 

La duale della disuguaglianza distributiva \`e:
\[
x \wedge (y \vee z) \ge (x \wedge y) \vee (x \wedge z)
\]
Che \`e anche la propriet\`a di disuguaglianza distributiva in $(L^{\ast}, \ge)$.

\textbf{Esercizio:} $\xi$: in un reticolo $(L, \le)$ si ha che $x \le z, y \in L \Rightarrow x \vee (y \wedge z) \le (x \vee y) \wedge z$ (disuguaglianza modulare). Dimostrarlo.

\vspace{5cm}

\begin{defn}[Reticolo distributivo]
Se in $(L, \le)$ vale l'identit\`a $x \wedge (y \vee z) = (x \wedge y) \vee (x \wedge z)$, il reticolo si dice distributivo. Vale anche la duale.
\end{defn}

Ogni reticolo distributivo \`e modulare. Esistono reticoli modulari ma non distributivi.

\begin{figure}[ht]
\centering
\begin{tikzpicture}
  \node (1) {1};
  \node (y) [below of=1] {$y$};
  \node (x) [left of=y] {$x$};
  \node (z) [right of=y] {$z$};
  \node (0) [below of=y] {0};
  \path[-]  (1) edge node {} (x)
            (1) edge node {} (y)
            (1) edge node {} (z)
            (x) edge node {} (0)
            (y) edge node {} (0)
            (z) edge node {} (0)
            ;
\end{tikzpicture}
\caption{\label{fig:mod_not_distr}Reticolo modulare ma non distributivo}
\end{figure}
Prendiamo ad esempio il reticolo in figura \ref{fig:mod_not_distr}: $(x \wedge y) \vee z = 0 \vee z = z$, $(x \vee z) \wedge (y \vee z) = 1 \wedge 1 = 1 \neq z$, quindi non \`e distributivo. 

Mentre \`e modulare: prendiamo la coppia $0 \le x$ in relazione e $y \in L$ non in relazione con $x$ (lo scegliamo non in relazione con $x$ perch\'e altrimenti sarebbe banale la dimostrazione). $ 0 \vee (y \wedge x) = 0 \vee 0 = 0$, e $(0 \vee y) \wedge x = 0 \wedge x = 0$. Quindi, \`e modulare.

L'insieme delle parti \`e un reticolo distributivo, e quindi anche modulare. Dimostrarlo.

\vspace{5cm}

\textbf{Esercizio:} $(\mathbb{N}, |)$ \`e un reticolo distributivo?

\vspace{5cm}

\section{Partizioni di $A$}

\begin{defn}[Partizione]\label{partizione}
Una partizione di $A$ \`e un insieme di sottoinsiemi di $A$, indicato con $\pi$, tale che:
\begin{description}
  \item[P1\label{itm:P1}] $\forall \ B \in \pi , \ B \neq \emptyset$
  \item[P2\label{itm:P2}] $B, C \in \pi , \ B \cap C \neq \emptyset \Rightarrow B = C$
  \item[P3\label{itm:P3}] $\forall \ a \in A \ \exists \ B \in \pi : a \in B$
\end{description}
\end{defn}
Una partizione $\pi$ \`e un insieme di sottoinsiemi non vuoti di $A$ t.c. hanno intersezione disgiunta e la loro unione \`e $A$. In altre parole, ogni elemento di $A$ appartiene ad un solo elemento di $\pi$.

Gli elementi di una partizione sono chiamati ``blocchi''.

Prendiamo $A = \mathbb{R}^2$. Posso pensare due partizioni banali: $\pi_0 = \left\{ B \subseteq \mathbb{R}^2 : |B| = 1 \right\}$, $\pi_1 = \left\{ \mathbb{R}^2 \right \}$.

\textbf{Esercizio:} indicare altre partizioni su $\mathbb{R}^2$.
\vspace{5cm}

\subsection{Classi di equivalenza}
\begin{defn}[Classe di equivalenza]
Prendiamo una relazione di equivalenza $\varepsilon \subseteq A \times A$. Definisco le classi di equivalenza come, dato un $a \in A$:
\[
[a] = \left \{ x \in A : a \varepsilon x \right\}
\]
$[a]$ si chiama ``classe di equivalenza rappresentata da $a$''.
\end{defn}
\begin{prop}\label{insieme_quoziente}
Se considero l'insieme delle classi di equivalenza $A / \varepsilon$, chiamato ``insieme quoziente'', questo insieme \`e una partizione di $A$.
\[
A/\varepsilon = \left\{ [a] \subseteq \mathbb{P}(A) : [a] \text{ \`e una classe di equivalenza} \right\}
\]
\end{prop}
\begin{proof}[Dimostrazione di \ref{insieme_quoziente}]
Bisogna dimostrare le tre propriet\`a specificate in \ref{partizione}.
\begin{itemize}
  \item $[a] \neq \emptyset$ \`e vero perch\'e la relazione \`e riflessiva, ed $a$ \`e in relazione almeno con s\'e stesso.
  \item $[a] \cap [b] \neq \emptyset \Rightarrow [a] = [b]$. Supponiamo esista $z \in [a] \cap [b] \Rightarrow a \varepsilon z$ e $b \varepsilon z$. Essendo la relazione simmetrica e transitiva, $z \varepsilon b \Rightarrow a \varepsilon b$. Per la propriet\`a simmetrica, di nuovo, $b \varepsilon a$. Quindi $ \forall \ x \in [a] \Rightarrow a \varepsilon x $ e per transitivit\`a $ b \varepsilon x$.
  \item $\forall \ a \in A \ \exists \ B \in \pi : a \in B$. Banalmente, ogni $\forall \ a \in A : a \in [a]$ con $[a] \in \mathbb{P}(A)$ per la propriet\`a riflessiva.
\end{itemize}
\end{proof}

\subsection{Congruenza modulo $n$ su $\mathbb{Z}$}

Consideriamo l'insieme quoziente $\mathbb{Z} / \equiv_{n}$, con $\equiv_{n} $ simbolo per la relazione di congruenza modulo $n$.

$\equiv_{n}$ con $n \in \mathbb{N}, n \ge 2$, \`e definita, con $ k \in \mathbb{Z}$, come:
\[
\forall \ a, b \in \mathbb{Z}, a \equiv_{n} b \Leftrightarrow n | (a - b) \Leftrightarrow (a - b) = k n 
\]
Posso definire quindi l'insieme quoziente sulla relazione di congruenza modulo $n$:
\begin{gather*}
\mathbb{Z} / \equiv_{n} \\
a \in \mathbb{Z} \\
[a] = \left \{ z \in \mathbb{Z} : a \equiv_{n} z \right \}
\end{gather*}

\begin{theorem}[Teorema di divisione su $\mathbb{Z}$]
$a, n \in \mathbb{Z}$ con $n > 0$, esiste un'unica coppia di interi $q, r \in \mathbb{Z}$ tale che:
\begin{itemize}
  \item $a = n q + r$
  \item $0 \le r < n$
\end{itemize}
\end{theorem}
Cosa significa $z \in [a]$, con $a$ e $z$ interi? Per il teorema di divisione, $a = n q + r$ e $z = n p + r'$. $a \varepsilon z $ significa che $ (a - z) = k n \Rightarrow n (q - p) + (r - r') = kn $. Essendo $r - r' < n$, necessariamente $r - r' = 0 \Rightarrow r = r'$. Ossia, se la differenza fra $a$ e $z$ \`e multiplo di $n$, devono avere lo stesso resto nella divisione per $n$.

Le classi di equivalenza hanno come rappresentante il resto della divisione modulo $n$. $Z / \equiv_{2} = \left \{ [0], [1] \right \}$. $Z / \equiv_{3} = \left \{ [0], [1], [2] \right \}$. In generale, $Z / \equiv_{n}$ ha $n$ classi di equivalenza.

\subsection{Relazioni di equivalenza e partizioni}

Il concetto di relazione di equivalenza \`e analogo al concetto di partizione. Data una relazione di equivalenza \`e data una partizione, e una partizione individua una relazione di equivalenza.

\begin{prop}
Sia $A \neq \emptyset$ e $\varepsilon$ una relazione di equivalenza su $A$, allora $\varepsilon$ individua una partizione di $A$ data da $A / \varepsilon = \left \{ [a] :  \text{ \`e una classe di equivalenza}  \right \}$. Viceversa, data una partizione $\pi$ di $A$, $\pi$ individua una relazione di equivalenza $\varepsilon_{\pi}$ su $A$, tale che $A / \varepsilon_{\pi} = \pi$, ossia \`e possibile stabilire una biezione $F: \mathcal{E}(A) \to \Pi(A)$ dove $\mathcal{E}(A)$ \`e l'insieme delle relazioni di equivalenza su $A$ e $\Pi(A)$ \`e l'insieme delle partizioni di $A$.
\end{prop}
\begin{proof}
Definisco $F$ come $\forall \ \varepsilon \in \mathcal{E}(A) \ F(\varepsilon) = A / \varepsilon$. Esiste la funzione $G$ inversa di $F$, $G : \Pi(A) \to \mathcal{E}(A)$, $G(\pi) = \varepsilon_{\pi}$ con $a \varepsilon_{\pi} b \Leftrightarrow a,b \in B \in \pi$.
\begin{gather*}
\varepsilon \xrightarrow{F} A / \varepsilon \xrightarrow{G} \varepsilon \\
\pi \xrightarrow{G} \varepsilon_{\pi} \xrightarrow{F} \pi = A / \varepsilon_{\pi} 
\end{gather*}

\textbf{Esercizio:} dimostrare che $F$ \`e iniettiva e suriettiva.

\vspace{5cm}
\end{proof}

Una semplice biezione non implica l'equivalenza. Prendo un insieme $E = \{e_1, e_2, \dots e_n\}$ di cardinalit\`a $|E| = n$, l'insieme delle relazioni d'ordine $L(E)$, e l'insieme delle sue permutazioni $S(E)$ di cardinalit\`a $n!$. Una permutazione \`e una biezione di un insieme finito. I due insiemi sono in corrispondenza biunivoca. Ho un'applicazione $F : S(E) \to L(E)$ che data un'occupazione $\sigma$ associa ad ogni elemento di $E$ il suo ordine associato $\sigma(e_1), \sigma(e_2), \dots \sigma(e_n)$. 

Ad ogni biezione posso associare un ordine. Ma i due concetti non sono analoghi (ossia equivalenti). Non ho un'unica biezione: per definire la biezione devo stabilire un ordine degli elementi di $E$. A seconda di come li ordino ho una biezione diversa.

\begin{defn}
Le biezioni naturali non dipendono dall'ordine lineare degli insiemi.
\end{defn}

Come \`e fatta la classe $[(a,b)] = \left \{ (c,d) \in \mathbb{N} \times \mathbb{N} : a + d = b + c \right\}$?

Se $a = b \Rightarrow [(a,a)] = \left \{ (c,c) \in \mathbb{N} \times \mathbb{N} : c \in \mathbb{N} \right\} = [(0,0)]$ (la coppia $(0,0)$ la scelgo come ``rappresentante standard'' o ``rappresentante canonico'').

Se $a < b \Rightarrow b = a + m \Rightarrow [(a,a+m)] = \left \{ (c, c + m) \in \mathbb{N} \times \mathbb{N} : c \in \mathbb{N}\right\} = [(0,m)]$.

Se $a > b \Rightarrow a = b + m \Rightarrow [(b + m, b)] = \left \{ (c + m, c) \in \mathbb{N} \times \mathbb{N} : c \in \mathbb{N}\right\} = [(m,0)]$.

Ciascuno dei tre tipi di classi di equivalenza ha un rappresentante canonico.

$\mathbb{N} \times \mathbb{N} / \rho$ \`e in corrispondenza biunivoca con $\mathbb{Z}$, associando $[(0,0)]$ a 0, $[(0,m)]$ a $-m$ e $[(m,0)]$ a $m$. Posso quindi costruire $\mathbb{Z}$ da $\mathbb{N}$.

\textbf{Esercizio:} $(\mathbb{N} \times \mathbb{N}, \rho)$ con $(a,b) \rho (c,d) \Leftrightarrow a + d = b + c$. Dimostrare che $\rho$ \`e una relazione di equivalenza.

\vspace{5cm}

\subsubsection{Proiezione di $A$ sul suo insieme quoziente, e sezione}

\begin{defn}[Proiezione]
Dato un insieme $A$, una relazione di equivalenza $\varepsilon$ e l'insieme quoziente $A / \varepsilon$, definiamo $p : A \to A / \varepsilon $ come $ p(a) = [a]$.
\end{defn}
$p$ \`e detta proiezione canonica, o naturale, ed \`e necessariamente suriettiva, ma in generale non \`e iniettiva (a meno che le classi abbiano tutte cardinalit\`a 1). Ad ogni elemento di $A$ associa la sua classe di equivalenza $[a]$.

Posso definire una famiglia di applicazioni $s : A / \varepsilon \to A $ ``contrarie'' a $p$ tali che $s([a]) = a$, chiamate sezioni.
\begin{defn}[Sezione]
$\forall \ B \in A / \varepsilon , s(B) = a$ con $a \in B$. La sezione sceglie un elemento ``rappresentante'' da ogni classe di equivalenza. 
\end{defn}
In generale non \`e suriettiva (a meno che le classi abbiano tutte cardinalit\`a 1), ma \`e necessariamente iniettiva.

\`E possibile comporre le due applicazioni:
\begin{itemize}
  \item $p \circ s: A / \varepsilon \to A / \varepsilon$. Questa composizione mi restituisce l'identit\`a dell'elemento che ``passo'' a $s$.
  \item $s \circ p : A \to A$. Questa composizione mi restituisce il rappresentante dell'elemento $a$ nella sua classe di equivalenza, ossia manda tutti gli elementi di una classe di equivalenza in un unico elemento di quella classe.
\end{itemize}

\begin{theorem}[Assioma della scelta]
Per ogni partizione \`e possibile scegliere un rappresentante in ogni classe, ossia si pu\`o definire una sezione. Equivale a dire che ogni applicazione suriettiva ha un'inversa destra, che equivale a dire che ogni applicazione iniettiva ha un'inversa sinistra, che equivale a dire che dato $A$ insieme infinito \`e possibile ordinare totalmente $A$.
\end{theorem}

Una conseguenza dell'assioma della scelta \`e che tutte le applicazioni suriettive hanno un'inversa destra, e tutte le applicazioni iniettive hanno un'inversa sinistra.
\begin{itemize}
  \item $f$ \`e iniettiva $\Leftrightarrow f$ ha un'inversa sinistra.
  \item $f$ \`e suriettiva $\Leftrightarrow f$ ha un'inversa destra.
\end{itemize}
Se una funzione ha sia un'inversa sinistra sia un'inversa destra, \`e biunivoca.

L'assioma della scelta permette di ordinare totalmente un insieme infinito. Scelgo un elemento $x_1 \in A$, dopodich\'e scelgo un elemento $x_2 \in A \setminus \{x_1\}$, e via dicendo.

\subsubsection{Nuclei di funzioni}

Ogni funzione $f$ definisce una relazione di equivalenza $\varepsilon_f$. 

\begin{defn}
Data $f : A \to B$, definiamo $\varepsilon_f$ su $A$. $\varepsilon_f$ \`e definita da:
\[
\forall \ x, y \in A , \ x \ \varepsilon_f \ y \Leftrightarrow f(x) = f(y)
\]
\end{defn}

\begin{defn}[Nucleo di una funzione]
L'insieme quoziente $A / \varepsilon_f$ \`e chiamato ``nucleo di $f$'', e si indica con $\ker f$.
\end{defn}

\begin{prop}
Ogni funzione $f : A \to B$ si pu\`o esprimere come composta di una funzione suriettiva con una iniettiva.
\end{prop}

\begin{proof}
Considero la funzione $F : \ker f \to Im_f$, che ad un elemento del nucleo di $f$ associa l'immagine del rappresentante di quell'elemento, ossia $b = F([a]) = f(c) \ \forall \ c \in [a] = f(a)$.

La funzione $F$ viene poi composta con $i : Im_f \to B $, detta ``immersione''. Da wikipedia:
\begin{defn}[Immersione]
Una struttura $A$  si dice immersa nella struttura $B$ se esiste una funzione iniettiva $f: A \to B$ tale che l'immagine $f(A)$ conserva tutte o parte delle strutture matematiche presenti in $A$, ereditandole da quelle di $B$. La funzione prende anch'essa il nome di \textit{immersione}.
\end{defn} 
In questo caso la funzione di immersione $i$ \`e la funzione identit\`a $ \forall \ b \in Im_f \ i(b) = b$, poich\'e $Im_f \subseteq B$.

La funzione identit\`a composta $F$, composta con la proiezione $p$ di $f$, d\`a proprio $f$.

\begin{figure}[ht]
\centering
\begin{tikzpicture}
  \node (A) {$A$};
  \node (B) [right of=A, node distance=4cm] {$B$};
  \node (ker) [below of=A, node distance=4cm] {$\ker f$};
  \node (Im) [right of=ker, node distance=4cm] {$Im_f$};
  \path[->]  (A) edge [left] node {$p$} (ker)
            (A) edge [above] node {$f$} (B)
            (ker) edge [below] node {$F$} (Im)
            (Im) edge [right] node {$i$} (B)
            (ker) edge [above left] node {$i \circ F$} (B)
            ;
\end{tikzpicture}
\caption{$f$ come composizione di $p$ e di $i \circ F$}
\end{figure}
\end{proof}

\begin{exmp}
Consideriamo la funzione $f : \mathbb{R}^2 \to \mathbb{R}^3$ definita da $f(x,y) = (x,x,0)$.

Le classi di equivalenza individuate in $\ker f$ saranno $[(x,y)] = \{ (z,t) \in \mathbb{R}^2 | z = x\}$. Per ogni classe possiamo individuare un rappresentante canonico del tipo $[(x,0)]$ con $x \in \mathbb{R}$.

Ad ogni classe devo associare la sua immagine, quindi a $[(x,y)]$ \`e associato $(x,x,0)$.

\begin{figure}[ht]
\centering
\begin{tikzpicture}
  \node (A) {$\mathbb{R}^2$};
  \node (B) [right of=A, node distance=4cm] {$\mathbb{R}^3$};
  \node (ker) [below of=A, node distance=4cm] {$\{[(x,0)]\}$};
  \node (Im) [right of=ker, node distance=4cm] {$\{(x,x,0)\}$};
  \path[->]  (A) edge [left] node {$p$} (ker)
            (A) edge [above] node {$f$} (B)
            (ker) edge [below] node {$i$} (Im)
            (Im) edge [right] node {$F$} (B)
            (ker) edge [above left] node {$i \circ F$} (B)
            ;
\end{tikzpicture}
\caption{$f(x,y) = (x,x,0)$ come composizione di due funzioni}
\end{figure}
\end{exmp}

\subsection{Teorema di omomorfismo per gli insiemi}
\begin{prop}
Per ogni $f : A \to B$ si ha $|Im_f| = |\ker f|$, ossia esiste una biezione $F : \ker f \to Im_f$.
\end{prop}
\begin{proof}
Per le definizioni di immagine di una funzione e nucleo di una funzione:
\[
Im_f = \{ b \in B : \exists \ a \in A \text{ t.c. } f(a) = b \}\subseteq B
\]
\[
\ker f = \{ [a] : \forall \ x \in [a] \ f(x) = f(a) \}
\]
$ \forall $ classe di $ \ker f \ F([a]) = f(a) = F([b]) \Rightarrow [a] = [b]$ e $\forall \ b \in Im_f \ b = f(a) \Rightarrow F([a]) = b$. $F$ \`e una funzione suriettiva e iniettiva, quindi gli elementi del $\ker f$ sono tanti quanti gli elementi di $Im_f$.
\end{proof}

\begin{figure}[ht]
\centering
\begin{tikzpicture}
  \node (A) {$A$};
  \node (f) [right of=A, node distance=2cm] {$f$};
  \node (B) [right of=f, node distance=2cm] {$B$};
  \node (1) [below of=A, node distance=1cm] {$1$};
  \node (2) [below of=1, node distance=1cm] {$2$};
  \node (3) [below of=2, node distance=1cm] {$3$};
  \node (4) [below of=3, node distance=1cm] {$4$};
  \node (5) [below of=4, node distance=1cm] {$5$};
  \node (6) [below of=5, node distance=1cm] {$6$};
  \node (7) [below of=6, node distance=1cm] {$7$};
  \node (a) [below of=B, node distance=1cm] {$a$};
  \node (b) [below of=a, node distance=1cm] {$b$};
  \node (c) [below of=b, node distance=1cm] {$c$};
  \node (d) [below of=c, node distance=1cm] {$d$};
  \node (e) [below of=d, node distance=1cm] {$e$};
  \node (f) [below of=e, node distance=1cm] {$f$};
  \path[->]  (1) edge node {} (b)
            (2) edge node {} (a)
            (3) edge node {} (a)
            (4) edge node {} (b)
            (5) edge node {} (d)
            (6) edge node {} (d)
            (7) edge node {} (d)
            ;
\end{tikzpicture}
\caption{$\ker f = \{ [1], [2], [5]\}$, $Im_f = \{a, b, d\}$ }
\end{figure}

\subsubsection{Partizioni come reticoli}

$\Pi (A)$ \`e insieme delle partizioni di $A$, $\subseteq $ \`e una relazione di raffinamento.

Dati $\pi, \sigma \in \Pi(A)$, $\pi \subseteq \sigma \Leftrightarrow $ $\forall \ B \in \pi$ si ha che $B$ \`e contenuto in un blocco di $\sigma \Leftrightarrow $ ogni blocco di $\sigma$ \`e unione di blocchi di $\pi \Leftrightarrow \forall \ x, y \in A , \ x \ \varepsilon_{\pi} \ y \Rightarrow x \ \varepsilon_{\sigma} \ y$.

\begin{prop}
$\left( \Pi(A) , \subseteq \right)$ \`e un insieme parzialmente ordinato ed \`e un reticolo, ossia presi comunque due elementi c'\`e l'$\inf$ e il $\sup$.
\end{prop}
\begin{proof}
$(\pi \wedge \sigma)$ \`e dato dalla relazione $ x \ \varepsilon_{(\pi \wedge \sigma)} \ y \Leftrightarrow x \ \varepsilon_{\pi} \ y $ e $ x \ \varepsilon_{\sigma} \ y $. Quindi per definizione $\pi \wedge \sigma $ \`e $\inf( \pi, \sigma)$.
\end{proof}

Verifichiamone le propriet\`a:
\begin{itemize}
  \item $\pi \wedge \sigma \le \pi, \sigma$, vero per definizione di $\le$.
  \item $\pi \le \sigma \Leftrightarrow $ se $ x \ \varepsilon_{\pi} \ y $ allora $x \ \varepsilon_{\sigma} \ y$
  \item $\tau \le \pi, \sigma \Rightarrow \tau \le (\pi \wedge \sigma)$, infatti:
   \[
   x \ \varepsilon_{\tau} \ y \Rightarrow 
   x \ \varepsilon_{\pi} \ y \text{ e }
   x \ \varepsilon_{\sigma} \ y 
   \Rightarrow x \ \varepsilon_{(\pi \wedge \sigma)} \ y
   \]
\end{itemize}

\begin{defn}
$x \ \varepsilon_{(\pi \vee \sigma)} \ y \Leftrightarrow \exists \ x = a_1, a_2 \dots a_n = y $ tale che $a_i \ \varepsilon_{\pi} \ a_j$ oppure $a_i \ \varepsilon_{\sigma} \ a_j$ con $i \in [1, n-1]$, ossia ho una catena che mette in relazione $a_i$ con $a_j$ passando per $\pi$ o $\sigma$.
\end{defn}

Poich\'e $(\pi \vee \sigma) = \sup (\pi, \sigma)$, allora:
\begin{itemize}
  \item $\pi, \sigma \le (\pi \vee \sigma)$, ossia $\pi \le (\pi \vee \sigma)$ e $\sigma \le (\pi \vee \sigma)$, vero per definizione perch\'e $x \ \varepsilon_{\pi} \ y \Rightarrow x \ \varepsilon_{\pi \vee \sigma} \ y$.
  \item $\pi, \sigma \le \tau \Rightarrow (\pi \vee \sigma) \le \tau$, ossia $x \ \varepsilon_{\pi} \ y \Rightarrow x \ \varepsilon_{\tau} \ y$ e $x \ \varepsilon_{\sigma} \ y \Rightarrow x \ \varepsilon_{\tau} \ y$. Quando dico $x \ \varepsilon_{(\pi \vee \sigma)} \ y$, allora $x \ \varepsilon \ a_2 \ \varepsilon \dots \varepsilon \ y$, in cui ogni $\varepsilon$ \`e o $\varepsilon_{\pi}$ o $\varepsilon_{\sigma}$, quindi in ogni caso $a_i \ \varepsilon_{\pi} \ a_j$ o $a_i \ \varepsilon_{\sigma} \ a_j$ per cui necessariamente $a_i \ \varepsilon_{\tau} \ a_j$. Segue per transitivit\`a che $x \ \varepsilon_{\tau} \ y$.
\end{itemize}

\textbf{Esercizio:} $(\Pi(A), \subseteq)$ \`e un reticolo e $\subseteq$ \`e una relazione d'ordine. Dimostrare che non \`e modulare.
\vspace{5cm}

\section{Morfismi}

Sono delle funzioni $f : A \to B$ che partono da $A$ con una struttura e arrivano a $B$ con la stessa struttura.

\subsection{Morfismi di insiemi parzialmente ordinati}

Detti anche ``morfismi d'ordine'' o funzioni monotone. 
\begin{defn}
Un morfismo di un insieme parzialmente ordinato \`e un'applicazione $f : P_1 \to P_2$ dove $(P_1, \le_{1})$ e $(P_2, \le_{2})$ sono insiemi particolarmente ordinati, tale che $\forall \ x, y \in P_1$ con $x \le_1 y \Rightarrow f(x) \le_2 f(y)$. $f$ \`e una funzione monotona (ossia $f$ conserva l'ordine). 
\end{defn}
Quindi un morfismo va da un'insieme con una struttura ad un altro insieme con la sua struttura. Lo indicheremo con questa notazione:
\[
f : (P_1, \le_{1}) \to (P_2, \le_{2})
\]
\begin{exmp}
Dati $P_1 = P_2 = \mathbb{P}(\Gamma)$ e la relazione $(\mathbb{P}(\Gamma), \subseteq)$, definisco $f : \mathbb{P}(\Gamma) \to \mathbb{P}(\Gamma)$ come, fissato $A$ sottoinsieme finito di $\Gamma$, $\forall \ X \in \mathbb{P}(\Gamma) f(X) = A \cap X$.

$f$ \`e un morfismo $f : (\mathbb{P}(\Gamma), \subseteq) \to (\mathbb{P}(\Gamma), \subseteq)$ poich\'e $\forall \ X, Y \in \mathbb{P}(\Gamma) \ X \subseteq Y \Rightarrow f(X) \subseteq f(Y)$ visto che $X \cap A \subseteq Y \cap A$.
\end{exmp}
Questa propriet\`a si chiama \textbf{isotonia} di $\subseteq$. Anche l'unione ha questa propriet\`a: $g : (\mathbb{P}(\Gamma), \subseteq) \to (\mathbb{P}(\Gamma), \subseteq)$ con $g(X) = A \cup X$

\begin{exmp}
Sia $\Gamma$ un insieme finito, definisco una funzione su $\mathbb{P}(\Gamma)$ e $(\mathbb{N}, \le)$ $f : \mathbb{P}(\Gamma) \to \mathbb{N}$ come $\forall \ X \in \mathbb{P}(\Gamma) \ f(X) = |X|$. Anche questa \`e una funzione monotona.
\end{exmp}

\subsection{Morfismi di reticoli}

$(L, \le)$ \`e un reticolo se $\forall \ x, y \in L \ \exists \ \inf(x, y) $ e $ \exists \ \sup(x,y)$. Ogni reticolo individua una struttura algebrica con due operazioni, $(L, \wedge, \vee)$, e inoltre dalla struttura algebrica posso definire il reticolo.

\begin{defn}
Dati due reticoli $(L_1, \le_1)$ e $(L_2, \le_2)$, un morfismo di reticoli \`e una funzione $f : L_1 \to L_2$ che verifica le seguenti propriet\`a:
\begin{description}
  \item[M1\label{itm:M1}] $\forall \ x, y \in L_1 \ f(\inf(x,y)) = \inf(f(x),f(y))$, e analogamente $\forall \ x, y \in L_1 \ f(\sup(x,y)) = \sup(f(x),f(y))$.
  \item[M2\label{itm:M2}] Deve rispettare l'ordine. $\forall \ x, y \in L_1 $ con $x \le_1 y \Rightarrow f(x) \le_2 f(y)$.
\end{description}
\end{defn}

\begin{exmp}
Riprendendo l'esempio precedente, $P_1 = P_2 = \mathbb{P}(\Gamma)$ e la relazione $(\mathbb{P}(\Gamma), \subseteq)$, definisco $f : \mathbb{P}(\Gamma) \to \mathbb{P}(\Gamma)$ come, fissato $A$ sottoinsieme finito di $\Gamma$, $\forall \ X \in \mathbb{P}(\Gamma) \ f(X) = A \cap X$.

In questo reticolo, $\inf(X,Y) = X \cap Y$, quindi $f(\inf(X,Y)) = f(X \cap Y) = (X \cap Y) \cap A$. Per verificare \ref{itm:M1}, $(X \cap Y) \cap A = \inf(f(X,Y)) = \inf(X \cap A, Y \cap A) = (X \cap A) \cap (Y \cap A)$, vero per la propriet\`a commutativa, la propriet\`a associativa e l'idempotenza degli insiemi.
\[
(X \cap A) \cap (Y \cap A) = (X \cap A) \cap (A \cap Y) = X \cap (A \cap A) \cap Y = X \cap A \cap Y
\]
\end{exmp}

\begin{exmp}\label{morfismo_no_reticoli}
Sia $\Gamma$ un insieme finito, definisco una funzione su $\mathbb{P}(\Gamma)$ e $(\mathbb{N}, \le)$ $f : \mathbb{P}(\Gamma) \to \mathbb{N}$ come $\forall \ X \in \mathbb{P}(\Gamma) f(X) = |X|$.

Devo verificare \ref{itm:M1}. $f(\inf(X,Y)) = \inf(f(X),f(Y))$. $f(X \cap Y) = |X \cap Y|$, mentre $\inf(f(X),f(Y)) = \inf(|X|,|Y|)$, ossia il minore fra i due numeri. Non \`e vero in generale, quindi questa funzione non \`e un morfismo di reticoli.
\end{exmp}

In realt\`a \ref{itm:M1} implica \ref{itm:M2}, a differenza di quanto visto nelle relazioni di equivalenza e di ordine.

\begin{prop}
$(L, \le_1)$ e $(L, \le_2)$ due reticoli. Sia $f : L_1 \to L_2$ tale che $\forall \ x, y \in L_1 \ f(\inf(x,y)) = \inf (f(x),f(y))$ e $\forall \ x, y \in L_1 \ f(\sup(x,y)) = \sup (f(x),f(y))$, allora $f$ \`e una funzione monotona, e quindi \`e un morfismo di reticoli.
\end{prop}
\begin{proof}
Bisogna dimostrare che conserva l'ordine, ossia che se $x \le_1 y \Rightarrow f(x) \le_2 f(y)$. Per l'ipotesi, $\inf(x,y) = x$ e $\sup(x,y)= y$. Quindi $f(\inf(x,y)) = f(x)$. Per \ref{itm:M1} $f(\inf(x,y)) = \inf(f(x),f(y))$, quindi $f(x) = \inf(f(x),f(y)) \Rightarrow f(x) \le_2 f(y)$.
\end{proof}
Stessa dimostrazione vale per il $\sup$. Abbiamo dimostrato gi\`a che il contrario non vale, nell'esempio \ref{morfismo_no_reticoli}.

\subsection{Isomorfismi}
Un isomorfismo \`e un morfismo biunivoco.

Dato l'insieme delle parti di $\Gamma$, $(\mathbb{P}(\Gamma), \subseteq)$, e l'insieme $2 = \{ 0, 1 \}$, $2^{\Gamma}$ \`e l'insieme delle funzioni $\Gamma \to \{0,1\}$. Date due funzioni $f, g \in 2^{\Gamma}$, $f \le g \Leftrightarrow \forall \ x \in 2 $ ho che $ f(x) \le g(x)$ (come da  definizione \ref{ordine_funzioni}). 

$(2^{\Gamma}, \le)$ \`e un reticolo. Infatti $\inf(f,g) : \Gamma \to 2$ \`e:
\[
\inf(f,g)(x) = 
\begin{cases}
f(x) & \text{ se } f(x) \le g(y) \\
g(x) & \text{ altrimenti}
\end{cases}
\]
Analogamente, $\sup(f,g)(x) : \Gamma \to 2$ \`e:
\[
\sup(f,g)(x) = 
\begin{cases}
g(x) & \text{ se } f(x) \le g(y) \\
f(x) & \text{ altrimenti}
\end{cases}
\]

Un esempio di isomorfismo \`e quindi $F : \mathbb{P}(\Gamma) \to 2^{\Gamma}$. Prendo un sottoinsieme $A \subseteq \Gamma$ (ossia un elemento $A \in \mathbb{P}(\Gamma)$) e definisco la ``funzione caratteristica di $A$'' che indicher\`o con $\varphi_{A} : \Gamma \to 2$ che dice se un elemento di $\Gamma$ \`e o non \`e in $\Gamma$, ossia:
\[
\forall \ x \in \Gamma \ \varphi_A (x) = 
\begin{cases}
0 \text{ se } x \notin A \\
1 \text{ se } x \in A
\end{cases}
\]
$F$ \`e biunivoca. Dimostriamo che \`e iniettiva e suriettiva.
\begin{prop}
Se $A, B \in \mathbb{P}(\Gamma)$ e $\varphi_A = \varphi_B \Rightarrow A = B$, ossia per doppia inclusione $A \le B$ e $B \le A$.
\end{prop}
\begin{proof}
Per definizione di funzione caratteristica, $\forall \ x \in A \Rightarrow \varphi_A(x) = 1 = \varphi_B(x) \Rightarrow x \in B$.
\end{proof}
\begin{prop}
$\forall \ f \in 2^{\Gamma} \ \exists \ A \in \mathbb{P}(\Gamma)$ t.c. $\varphi_A = f$.
\end{prop}
\begin{proof}
Prendo $\Gamma = \{ 1, 2, 3, 4, 5, 6, 7\}$, e rappresento $f$ dal punto di vista dell'occupazione.

\begin{tabular}{ccccccc}
1 & 2 & 3 & 4 & 5 & 6 & 7 \\
0 & 0 & 1 & 1 & 0 & 1 & 0
\end{tabular}

Quindi $A = \{ 3, 4, 6 \} = f^{-1} (1)$, ossia la controimmagine di $\varphi_A$.
\end{proof}

Poich\'e $F$ \`e un isomorfismo, $A \subseteq B \Rightarrow F(A) \le F(B) = \varphi(A) \le \varphi(B)$, infatti $\forall \ x \in \Gamma$:
\[
\varphi_A(x) = 
\begin{cases}
0 \text{ se } x \notin A \Rightarrow \varphi_A(x) \le \varphi_B(x) \\
1 \text{ se } x \in A \Rightarrow \text{ poich\'e } A \subseteq B, \ \varphi_B(x) = 1 \Rightarrow \varphi_A(x) \le \varphi_B(X)
\end{cases}
\]
Dimostriamo che si tratta di un morfismo di reticoli.
\begin{prop}
$F(\inf(A,B)) = \inf(F(A), F(B))$, ossia $F(\inf(A,B)) = F(A \cap B) = \varphi_{(A \cap B)}$ e $\inf(F(A), F(B)) = \inf(\varphi_A, \varphi_B)$.

Quindi $\inf(\varphi_A, \varphi_B) = \varphi_{(A \cap B)}$.
\end{prop}
\begin{proof}
Per le propriet\`a dei reticoli:
\begin{itemize}
  \item $\varphi_{A \cap B} \le \varphi_A, \varphi_B$
  \[
  \forall \ x \in \Gamma \varphi_{A \cap B} (x) =
  \begin{cases}
  0 \text{ se } x \notin (A \cap B) \\
  1 \text{ altrimenti}
  \end{cases}
  \]
  \[
  \varphi_{A \cap B} (x) = 
  \begin{cases} 
  0 \le \varphi_A \\
  1 \Rightarrow x \in (A \cap B) \subseteq A \Rightarrow \varphi_A(x) = 1 \Rightarrow \varphi_{A \cap B} (x) \le \varphi_A (x)
  \end{cases}
  \]
  \item $f \le \varphi_A, \varphi_B \Rightarrow f \le \varphi_{A \cap B}$
  \begin{gather*}
  f \in 2^{\Gamma} \\
  f \le \varphi_A, \varphi_B \Rightarrow
  \forall \ x \in \Gamma \ f(x) \le 
  \varphi_A(x) 
  \text{ e } 
  \varphi_B(x) \\
  f(x) =
  \begin{cases}
  0 \le \varphi_{A \cap B} (x) \\
  1 \Rightarrow \varphi_A (x) = \varphi_B (x) = 1 \Rightarrow x \in (A \cap B) \Rightarrow \varphi_{A \cap B} (x) = 1
  \end{cases}
  \end{gather*}
\end{itemize}
\end{proof}

L'insieme delle parti di un insieme $\Gamma$ \`e in corrispondenza biunivoca con l'insieme delle funzioni da $\Gamma$ in 2, quindi hanno la stessa cardinalit\`a.

\subsubsection{Considerazioni generali sugli isomorfismi}

\begin{enumerate}
  \item Sia $F : (P, \le) \to (Q, \le)$ un isomorfismo da $P$ in $Q$, posso prendere la sua inversa $G : (Q, \le) \to (P, \le)$ che \`e a sua volta un isomorfismo da $Q$ in $P$.
  \item Se prendo tre strutture $P, G, R$ e due isomorfismi $F : (P, \le) \to (Q, \le)$ e $G : (Q, \le) \to (R, \le)$, $G \circ F : (P, \le) \to (R, \le)$.
  \item Le relazioni di equivalenza sono una generalizzazione dell'uguaglianza. La relazione di uguaglianza \`e la pi\`u piccola relazione di equivalenza, in cui tutte le classi sono costituite da un solo elemento.
  \item Se prendo tutte le possibili strutture d'ordine $(P, \le)$, posso creare una relazione di equivalenza per cui due strutture sono equivalenti se c'\`e un isomorfismo.

  Ossia, prendo due strutture $(P, \le)$ e $(Q, \le)$, e dico che $(P, \le)$ \`e equivalente a $(Q, \le)$ se esiste un isomorfismo da $P$ a $Q$. Lo indico con $(P, \le) \cong (Q, \le)$.
  \item Se due strutture sono isomorfe (come in figura \ref{fig:strutture_isomorfe}), posso studiare una sola struttura e le propriet\`a di quella struttura valgono su tutte le strutture isomorfe.
\end{enumerate}

\begin{figure}
\centering
\begin{tikzpicture}
  \node (6) {6};
  \node (5) [left of=6, node distance=1cm] {5};
  \node (4) [below right of=5, node distance=1cm] {4};
  \node (3) [below right of=4, node distance=1cm] {3};
  \node (2) [below of=4, node distance=1cm] {2};
  \node (1) [below of=2, node distance=1cm] {1};
  \path[-]  (6) edge node {} (4)
            (5) edge node {} (4)
            (4) edge node {} (2)
            (2) edge node {} (3)
            (1) edge node {} (2)
            ;
\end{tikzpicture}
\begin{tikzpicture}
  \node (6) {e};
  \node (5) [left of=6, node distance=1cm] {f};
  \node (4) [below right of=5, node distance=1cm] {d};
  \node (3) [below right of=4, node distance=1cm] {c};
  \node (2) [below of=4, node distance=1cm] {b};
  \node (1) [below of=2, node distance=1cm] {a};
  \path[-]  (6) edge node {} (4)
            (5) edge node {} (4)
            (4) edge node {} (2)
            (2) edge node {} (3)
            (1) edge node {} (2)
            ;
\end{tikzpicture}
\caption{\label{fig:strutture_isomorfe}Esempio di strutture isomorfe}
\end{figure}

\textbf{Esercizio:} trovare tutti i reticoli (a meno di isomorfismi) con 4 elementi.

\vspace{5cm}

\section{Numeri Naturali}

$\mathbb{N}$ \`e l'insieme dei numeri naturali. Seguiamo la definizione di Peano.
\begin{defn}[Numeri naturali]
L'insieme dei numeri naturali \`e un insieme non vuoto tale che:
\begin{description}
  \item[N1\label{itm:N1}] Esiste una funzione $\sigma : \mathbb{N} \to \mathbb{N}$ (una endofunzione) iniettiva detta ``successore'';
  \item[N2\label{itm:N2}] Esiste un elemento di $\mathbb{N}$ chiamato ``zero'' ed indicato con 0 tale che $0 \notin Im_{\sigma}$, ossia che non \`e il successore di nessun numero naturale;
  \item[N3\label{itm:N3}] Principio di induzione: se $U \subseteq \mathbb{N}$ tale che:
  \begin{itemize}
    \item $0 \in U$;
    \item $n \in U \Rightarrow \sigma(n) \in U$;
  \end{itemize}
  allora $U = \mathbb{N}$.
\end{description}
\end{defn}
Dire che un insieme \`e finito vuol dire che ogni iniezione sull'insieme \`e pure una suriezione. Quindi da \ref{itm:N1} e \ref{itm:N2} segue che $\mathbb{N}$ \`e infinito, poich\'e esiste una funzione iniettiva che non \`e suriettiva.
\begin{prop}
L'immagine di $\sigma$ \`e tutto $\mathbb{N}$ escluso lo 0. $Im_{\sigma} = \mathbb{N} \setminus \{ 0 \}$
\end{prop}
\begin{proof}
Sia $U = Im_{\sigma} \cup 0$. $U$ coincide con $\mathbb{N}$ per \ref{itm:N3}. Infatti:
\begin{itemize}
  \item $0 \in U$ per definizione di $U$;
  \item $\forall \ n \in U, \ \sigma(n) \in U$.
\end{itemize}
\end{proof}
L'elemento $\sigma(0)$ si indica con 1.

\subsection{Iterazioni}

Dato un insieme $A \neq \emptyset$ ed una funzione $f : A \to A$, con $\circ : A^A \times A^A \to A^A$ a indicare la composizione, $(A^A, \circ)$ \`e un monoide. Infatti $f \circ id_A = f = id_A \circ f$ \`e l'identit\`a.

\begin{defn}[Iterazioni]
Le iterazioni di $f : A \to A$ sono definite da:
\begin{itemize}
  \item $f^0 = id_A$;
  \item $\forall \ n \in \mathbb{N}$ definisco $f^{\sigma(n)} = f \circ f^n$.
\end{itemize}
\end{defn}
\begin{prop}
$f^n$ \`e ben definita per ogni $n \in \mathbb{N}$.
\end{prop}
\begin{proof}
Sia $U = \{ n \in \mathbb{N} : f^n $ \`e ben definita $\}$. $0 \in U$ per definizione. Se $n \in U f^n$ \`e definita, quindi \`e definita anche $f^{\sigma(n)} = f \circ f^n$, essendo $f$ data. Quindi $U = \mathbb{N}$.
\end{proof}
\subsubsection{Iterazioni di $\sigma$}
\begin{defn}
Definiamo le iterazioni di $\sigma$ come segue:
\begin{itemize}
  \item $\sigma^0 = id_{\mathbb{N}}$;
  \item $\sigma^{\sigma(n)} = \sigma \circ \sigma^n \ \forall \ n \in \mathbb{N}$.
\end{itemize}
\end{defn}

\begin{prop}\label{iterazione_nesima}
$\sigma^n(0) = n$
\end{prop}
\begin{proof}
Si dimostra per induzione. $U = \{ n \in \mathbb{N} : \sigma^n(0) = n \}$. 

Lo 0 \`e gi\`a $\in U$, poich\'e $\sigma^0 (0) = id_{\mathbb{N}}(0) = 0$.

$\sigma(n) \in U \Rightarrow \sigma^{\sigma(n)} (0) = \sigma(n)$. Per definizione delle iterazioni su $\sigma$, $\sigma^{\sigma(n)}(0) = \sigma \circ \sigma^{n} (0) = \sigma(n)$ poich\'e $\sigma(n) \in U$ per ipotesi.
\end{proof}

\begin{defn}[Potenze]
Ho un monoide $(M, \cdot)$. $\cdot : M \times M \to M $ associativa e con unit\`a $1_M$. $a \in M$ $a \cdot 1_M = a = 1_M \cdot a$
\begin{itemize}
  \item $a^0 = 1_M$
  \item $a^{\sigma(n)} = a \cdot a^n$
\end{itemize}
\textit{Sinceramente non ho capito che c'entra col resto e perch\'e sta qui.}
\end{defn}

\begin{prop}
$\forall \ n \in \mathbb{N} \setminus \{ 0 \}$, $0 \notin Im_{\sigma^n}$, ossia lo 0 non \`e nell'immagine di nessuna iterazione di $\sigma$, ossia comunque itero $\sigma$ non ottengo mai lo 0.
\end{prop}
\begin{proof}
Lo dimostriamo per assurdo. Supponiamo che $\exists \ n \in \mathbb{N}$ t.c. $0 \in Im_{\sigma^n} \Rightarrow \exists \ x \in \mathbb{N}$ t.c. $\sigma^n (x) = 0$. Poich\'e $n \neq 0 \Rightarrow n \in Im_{\sigma} = \mathbb{N} \setminus \{ 0 \} \Rightarrow n = \sigma(t)$. Quindi $0 = \sigma^n(x) = \sigma^{\sigma(t)}(x) = (\sigma \circ \sigma^t)(x) = \sigma(\sigma^t(x)) \Rightarrow 0 \in Im_{\sigma}$. Ho ottenuto l'assurdo con \ref{itm:N2}.
\end{proof}
Attraverso le tre propriet\`a che definiscono $\mathbb{N}$ si possono ritrovare tutte le altre propriet\`a di $\mathbb{N}$. 

\subsection{Operazioni su $\mathbb{N}$}

Possiamo definire il monoide $(\mathbb{N}, +)$. L'operazione di somma $+ : \mathbb{N} \times \mathbb{N} \to \mathbb{N}$ \`e definita cos\`i:
\[
m + n = \sigma^{n}(m)
\]
\textbf{Esercizio:} dimostare che \`e commutativa, e che quindi $\sigma^n(m) = \sigma^m(n)$. \vspace{5cm}

L'elemento neutro \`e lo 0, infatti: $ m + 0 = \sigma^0 (m) = id_{\mathbb{N}} (m) = m $ e $ 0 + m = \sigma^m (0) = m $ per la proposizione \ref{iterazione_nesima}.

\begin{oss}
$n + 1 = n + \sigma(0) = \sigma(n)$
\end{oss}

L'operazione + \`e associativa, oltre che commutativa. 

La somma in $\mathbb{N}$ ha la ``regola di cancellazione''. $\forall \ m, n, k \in \mathbb{N}$, se $m + k = n + k \Rightarrow m = n$. In realt\`a in $\mathbb{N}$ non esiste l'inverso, quindi questa regola ha poco a che vedere con la ``cancellazione''.

Tutte le propriet\`a sono pi\`u facili da dimostare con il seguente lemma.
\begin{lem}\label{successore_somma}
\[
m + \sigma (n) = \sigma(m + n)
\]
\end{lem}
\begin{proof}
\[
m + \sigma(n) = \sigma^{\sigma(n)} (m) = (\sigma \circ \sigma^n) (m) = \sigma ( \sigma^n (m) ) = \sigma(m + n)
\]
\end{proof}
\begin{defn}[Prodotto su $\mathbb{N}$]
$(\mathbb{N}, \cdot)$ \`e un'operazione $\cdot : \mathbb{N} \times \mathbb{N} \to \mathbb{N}$ tale che $\forall \ (m, n) \in \mathbb{N} \times \mathbb{N}$, $m \cdot n = (\sigma^m)^n(0)$.
\end{defn}
Il prodotto \`e un'operazione associativa e distributiva.

\begin{prop}
$\sigma(0) = 1$ \`e l'elemento neutro.
\end{prop}
\begin{proof}
$m \cdot \sigma(0) = (\sigma^m)^{\sigma(0)} (0) = \sigma^m (0) = m$

$\sigma(0) \cdot m = (\sigma^{\sigma(0)})^m (0) = \sigma^m (0) = m$
\end{proof}

\begin{defn}[Legge di annullamento del prodotto]
$m \cdot n = 0 \Leftrightarrow m = 0 $ oppure $ n = 0 $.
\end{defn}

Sul prodotto e sulla somma valgono le leggi distributive: $k \cdot (m+n) = k \cdot m + k \cdot n$ e, per commutativit\`a, $(m + n) \cdot k = m \cdot k + n \cdot k$.
\begin{lem}\label{prodotto_successore}
$m \cdot \sigma(n) = m \cdot n + m$
\end{lem}
\begin{proof}
$m \cdot \sigma(n)$ per definizione del prodotto \`e $ (\sigma^m)^{\sigma(n)} (0) $ che per definizione di iterazione \`e $ (\sigma^m \circ (\sigma^m)^n) (0) = \sigma^m ( (\sigma^m)^n (0)) = \sigma^m (m \cdot n) = m \cdot n + m$
\end{proof}
Usando i due lemmi (lemma \ref{successore_somma} e lemma \ref{prodotto_successore}) si ottengono tutte le propriet\`a delle operazioni su $\mathbb{N}$.

\subsection{Ordine naturale su $\mathbb{N}$}

Possiamo definire una relazione d'ordine totale (e naturale) su $\mathbb{N}$ usando le due operazioni.
\begin{defn}[Ordine naturale]
$m \le n \Leftrightarrow \exists \ k \in \mathbb{N} : m + k = n$
\end{defn}
\`E una relazione d'ordine totale (ossia l'insieme $\mathbb{N}$ \`e linearmente ordinato), ossia $\forall \ m, n \in \mathbb{N}$ si ha che $ m \le n$ oppure $n \le m$.

L'ordine naturale su $\mathbb{N}$ \`e un ``ordine buono'', ossia $(\mathbb{N}, \le)$ si dice ``bene ordinato''. Il ``buon ordinamento'' \`e equivalente all'assioma della scelta. Vuol dire che ogni suo sottoinsieme non vuoto ha un primo elemento. $\mathbb{R}$, ad esempio, non \`e bene ordinato.
\begin{proof}[Dimostrazione del buon ordinamento]
Dimostriamolo per assurdo. Suppongo esista un sottoinsieme $V \neq \emptyset$ che non ha un primo elemento. Ossia, $\forall \ v \in V \ \exists \ w \in V$ t.c. $w \le v$.

Consideriamo la proposizione $P_n$: ``$\forall \ n \in \mathbb{N} \land \ \forall \ v \in V, \ n \le v$'' e dimostriamo che \`e vera $\forall \ n \in \mathbb{N}$.

\`E vero con 0: $\forall \ v \in V 0 \le v$.

Supponiamo sia vero per un $n \in \mathbb{N}$, e dimostriamo che \`e vero per $\sigma(n) $ che $ \forall \ v \in V \sigma(n) \le v$.

Per ipotesi di induzione so che $\forall \ v \in V, \ n \le v \Rightarrow \exists \ x \in \mathbb{N}$ t.c. $n + x = v$, per la definizione di $\le$. $x$ non pu\`o essere 0, altrimenti $n$ sarebbe il primo elemento di $V$. Quindi $x \neq 0 \Rightarrow x = \sigma(y) = y + 1$, e quindi $n + y + 1 = v $ ma per la propriet\`a commutativa $n + 1 + y = v \Rightarrow \sigma(n) + y = v \Rightarrow \sigma(n) \le v$.

Siamo giunti all'assurdo. Se prendo $n = v + 1$ avrei che $v + 1 \le v$.
\end{proof}

\subsection{Altre forme del principio di induzione}

Se $n_0 \in U$ (base dell'induzione) e $n_0 \le n \in U \Rightarrow n+1 = \sigma(n) \in U$, quindi $U = \{ n \in \mathbb{N}, n_0  \le n\}$.

$\mathbb{Z} \simeq \mathbb{N} \times \mathbb{N} / \rho$ con $(m, n) \rho (p, q) \Leftrightarrow m + q = n + p$. Le classi di equivalenza sono $[(0,0)], [(m,0)]$ e $[(0,m)]$. Posso chiamarle anche 0, $m$ e $-m$

\section{Principi del calcolo combinatorio}

Enumerare corrisponde a mettere in fila una serie di elementi. Contare significa saper stabilire una corrispondenza biunivoca fra due insiemi, ossia saper dire che gli elementi di $A$ sono quanti gli elementi di $B$, e saper poi dire che l'insieme $A$ ha $n$ elementi, ossia ha tanti elementi quanti sono i primi $n$ numeri naturali, ossia tanti quanti gli elementi di $[n] = \{1, 2, \dots n \}$.

Dati due insiemi $A$ e $B$, se dico $A \rho B \Leftrightarrow $ esiste una corrispondenza biunivoca da $A$ a $B$, verifico che $\rho$ \`e:
\begin{itemize}
   \item riflessiva;
   \item simmetrica;
   \item transitiva.
 \end{itemize} 
 Quindi $\rho$ \`e una relazione di equivalenza nell'insieme degli insiemi finiti. $A \rho B \Leftrightarrow |A| = |B| \Leftrightarrow A \leftrightarrow B \Leftrightarrow$ la cardinalit\`a di $A$ \`e uguale alla cardinalit\`a di $B$.

\subsection{Principio della somma}

\begin{prop}[Principio della somma]
 Dati $A$, $B$ tali che $A \cap B = \emptyset$, segue che $|A \cup B| = |A| + |B|$.
\end{prop}
\begin{proof}
$|A| = m$, $|B| = n$. Devo quindi poter realizzare una corrispondenza biunivoca $F$ fra $|A \cup B|$ e $[m + n]$. Ossia, dati $A = \{ a_1, a_2, \dots a_m\}$ e $B = \{ b_1, b_2 \dots b_n\}$, ho che $F(a_i) = i$ e $F(b_i) = m + i$.
\end{proof}

\`E possibile generalizzare la regola della somma.
\begin{prop}[Principio della somma 2]
Dati $t$ insiemi $A_1, \dots A_t$ tali che $A_i \cap A_j = \emptyset$ se $i \neq j$, ho che:
\[
\left| \bigcup_{i = 1}^{t} A_i\right| = \sum_{i = 1}^{t} |A_i|
\]
\end{prop}

$+ \mapsto A \dotcup B$ con $ A \cap B = \emptyset$.

\subsection{Principio del prodotto}

\begin{prop}[Principio del prodotto]
Dati $A$, $B$, ho che $|A \times B| = |A| \cdot |B|$.
\end{prop}
\begin{proof}
C'\`e una corrispondenza biunivoca $F : (A \times B) \to [m \cdot n]$. Quanto vale $F (a_i, b_j)$? In generale, contando con il metodo della diagonale (figura \ref{fig:metodo_diagonale}), ho che sulla diagonale $k$-esima trovo tutte le coppie tali per cui $i + j = k + 1$.

\begin{figure}
\centering
\begin{tikzpicture}
\begin{axis}[
    % graph options
]
\addplot[dashed,-] coordinates {(0,0) (0,5)};
\addplot[dashed,-] coordinates {(1,0) (1,5)};
\addplot[dashed,-] coordinates {(2,0) (2,5)};
\addplot[dashed,-] coordinates {(3,0) (3,5)};
\addplot[dashed,-] coordinates {(4,0) (4,5)};
\addplot[dashed,-] coordinates {(5,0) (5,5)};
\addplot[dashed,-] coordinates {(0,0) (5,0)};
\addplot[dashed,-] coordinates {(0,1) (5,1)};
\addplot[dashed,-] coordinates {(0,2) (5,2)};
\addplot[dashed,-] coordinates {(0,3) (5,3)};
\addplot[dashed,-] coordinates {(0,4) (5,4)};
\addplot[dashed,-] coordinates {(0,5) (5,5)};
\addplot[very thick, -] coordinates {(1,1) (2,1)};
\addplot[very thick, -] coordinates {(2,1) (1,2)};
\addplot[very thick, -] coordinates {(1,2) (1,3)};
\addplot[very thick, -] coordinates {(1,3) (3,1)};
\addplot[very thick, -] coordinates {(3,1) (4,1)};
\addplot[very thick, -] coordinates {(4,1) (1,4)};
\addplot[very thick, -] coordinates {(1,4) (1,5)};
\addplot[very thick, -] coordinates {(1,5) (5,1)};
\addplot[very thick, -] coordinates {(5,1) (5,2)};
\addplot[very thick, -] coordinates {(5,2) (2,5)};
\addplot[very thick, -] coordinates {(2,5) (3,5)};
\addplot[very thick, -] coordinates {(3,5) (5,3)};
\addplot[very thick, -] coordinates {(5,3) (5,4)};
\addplot[very thick, -] coordinates {(5,4) (4,5)};
\addplot[very thick, -] coordinates {(4,5) (5,5)};
\end{axis}
\end{tikzpicture}
\caption{\label{fig:metodo_diagonale}Metodo della diagonale}
\end{figure}
\end{proof}

\`E possibile generalizzare la regola del prodotto.
\begin{prop}[Principio del prodotto 2]
Se prendo un insieme finito di insiemi $A_1, \dots a_t$, ho che:
\[
\left| A_1 \times \dots \times A_t \right| = |A_1| \cdot \ldots \cdot |A_t|
\]
\end{prop}
\begin{proof}
Si dimostra per induzione su $t$.
\end{proof}

\begin{defn}[Principio del prodotto 3]
Dato un sottoinsieme $S$ del prodotto cartesiano $X^n$, $S \subseteq X^n$, ossia un insieme di $n$-uple, se verifico che:
\begin{itemize}
  \item al primo posto di una $n$-upla in $S$ ci sono $i_1$ elementi;
  \item $\forall \ j = 2 \dots n-1$ ci sono $i_j$ $n$-uple che hanno le prime $j-1$ coordinate uguali;
\end{itemize}
allora $|S| = i_1 \cdot i_2 \cdot \dots i_n$.
\end{defn}

\subsection{Dimostrazioni biettive}

Se voglio dimostrare che $a = b$, posso trovare due insiemi $A$ e $B$ tali che $|A| = a$ e $|B| = b$, e dimostrare che sono in biezione.

Due insiemi con intersezione disgiunta possono essere i due blocchi di una partizione con due soli blocchi.

\begin{prop}
Dato l'insieme delle funzioni da $A$ in $R$ $R^A = \{ f : A \to R\}$, se $|A| = n$ e $|R| = r$, allora $|R^A| = r^n = |R^n|$. Quindi esiste una biezione fra $R^A$ e $R^n$. $R^n$ \`e il prodotto di $R$ con s\'e stesso $n$ volte. Quindi le funzioni da $A$ in $R$ sono quante le $R$-uple di $n$ elementi.
\end{prop}
\begin{proof}
Data una funzione $f : A \to R$, definisco $F : R^A \to R^n$ tale che $(f, A = \{ a_1, \dots a_n\}) \mapsto (f(a_1), \dots f(a_n))$, che dal punto di vista dell'occupazione \`e:

\begin{tabular}{cccc}
$a_1$ & $a_2$ & $\dots$ & $a_n$ \\
$f(a_1)$ & $f(a_2)$ & $\dots$ & $f(a_n)$ 
\end{tabular}

Ossia, ad ogni funzione $f : A \to R$ associo la $n$-upla dei valori di $f$ ordinata seguendo un ordine lineare degli elementi di $A$.
\end{proof}
\begin{prop}
Dati $A, B$ disgiunti, $R^A \times R^B$ ha cardinalit\`a $|R^A \times R^B| = |R^{A \cup B}|$, dato che $|R^A| = r^n$ e $|R^B| = r^m$ e che quindi $r^n \cdot r^m = r^{n + m}$.

Quindi $|R^A \times R^B| $ e $ |R^{A \cup B}|$ sono in corrispondenza biunivoca.
\end{prop}
\begin{proof}
Esiste una biezione $F : R^A \times R^B \to R^{A \cup B}$. Questa funzione prende una coppia di funzioni e ne restituisce una terza $(f : A \to R, g: B \to R) \mapsto h : (A \cup B) \to R$ definita come, $\forall \ x \in A \cup B$:
\[
h(x) = 
\begin{cases}
f(x) \text{ se } x \in A\\
g(x) \text{ se } x \in B
\end{cases}
\]
\end{proof}
Dati due insiemi $R$, $S$ ed un insieme $A$:
\[
|R^A \times S^A| = |(R \times S)^A|
\]
\[
|(R^A)^B| = |R^{A \cdot B}|
\]
\textbf{Esercizio:} definire la biezione che dimostra le due propriet\`a precedenti.
\vspace{5cm}

La maggior parte dei numeri positivi che compaiono in combinatoria hanno un'interpretazione in teoria degli insiemi.

\subsection{Fattoriale decrescente}

\begin{defn}
Il fattoriale decrescente \`e una successione:
\begin{equation}
\{[r]_n\}_{n \in \mathbb{N}} \text{ : }
\begin{cases}
[r]_0 = 1 \\
[r]_{n+1} = [r]_n \cdot (r - n)
\end{cases}
\end{equation}
\end{defn}
\`E evidente che:
\[
[r]_n = \frac{r!}{(r-n)!}
\]

\begin{oss}
$[r]_n = 0$ se $r < n$. Inoltre, $[r]_n = n!$ se $r = n$.
\end{oss}

\begin{prop}
Dato $In(A,R)$, l'insieme delle funzioni iniettive da $A$ in $R$, se $|A| = n$ e $|R| = r$, allora $|In (A,R)| = [r]_n$.

In altre parole, il fattoriale decrescente $[r]_n$ corrisponde all'insieme delle funzioni iniettive da $A$ in $R$ $In(A,R)$.

\[
\left| In(A,R) \right| = \left| In (A- \{x\},R) \right| \cdot \left| R - Im_f \right|
\]

\end{prop}
\begin{proof}
Ricordiamo la definizione di iniettivit\`a: $\forall \ x, y \in A$ se $f(x) = f(y) \Rightarrow x = y$, $f$ \`e iniettiva.

Dimostriamo ora la proposizione per induzione. Se $A = \emptyset$, $|In(\emptyset, R)| = 1$.

Supponendo di aver calcolato il numero di funzioni inettive da un insieme con $n$ elementi ad un insieme con $r$ elementi (ossia $[r]_n$), dobbiamo determinare $[r]_{n+1}$.

Se $|A| = (n+1)$, sapendo che tutti gli elementi di $A$ sono distinti, prendo $x \in A$.

Data una funzione $f : A \to R$, questa individua una coppia $(f(x), f^{-} : A \setminus \{ x \} \to R)$ dove $f^{-}$ \`e definita come $f$ ossia $f^{-} (a) = f(a)$.

In quanti modi si pu\`o scegliere $f(x)$? In $|R \setminus Im_{f^-}|$ modi (ossia, $r - n$). Le funzioni iniettive da $A \setminus \{ x \}$ in $R$ sono $[r]_n$ per definizione, quindi per il principio del prodotto il numero di coppie che posso creare sono $[r]_{n+1} = [r]_n \cdot (r - n)$.
\end{proof}

\begin{prop}
Sappiamo che $|\mathbb{P}(A)| = |2^A| = 2^n$ con $|A| = n$. Quanti sono i sottoinsiemi pari di $A$, ossia $S \subseteq A $ t.c. $|S| = 2k$?

Gli insiemi di cardinalit\`a pari sono tanti quanti quelli di cardinalit\`a dispari, quindi sono $2^{n-1}$.
\end{prop}
\begin{proof}
So che $2^{n-1} = |\mathbb{P}(A \setminus \{x\})|$, quindi l'insieme dei sottoinsiemi pari di $A$ \`e in biezione con l'insieme delle parti di $A$ a cui ho tolto un elemento. Indicando con $P$ l'insieme dei sottoinsiemi pari di $A$, ho una biezione $F : \mathbb{P}(A \setminus \{x\}) \to P$:
\[
F(S) = 
\begin{cases}
S \text{ se } S \text{ \`e pari} \\
S \cup \{ x \} \text{ se } S \text{ \`e dispari} \\
\end{cases}
\]
\end{proof}

\subsection{Coefficienti  binomiali}
Indichiamo con $\binom{n}{k}$ il numero di sottoinsiemi con $k$ elementi di un insieme con $n$ elementi. Vediamo subito che $\binom{n}{k}$ per $k > n$ \`e 0. $\binom{n}{0} = 1$, $\binom{n}{1} = n$. Inoltre $\binom{n}{k} = \binom{n}{n-k}$, ossia il numero di sottoinsiemi con $k$ elementi \`e uguale al numero dei loro insiemi complementari.
\begin{prop}[Definizione ricorsiva di Pascal dei coefficienti binomiali]
\begin{equation}
\binom{n}{k} = \binom{n-1}{k} + \binom{n-1}{k-1}
\end{equation}
$\binom{n}{k}$ \`e il numero di sottoinsiemi con $k$ elementi, indicato con $\mathbb{P}_{k}(A)$. Lo divido in due sottoinsiemi disgiunti, con cardinalit\`a rispettivamente $\binom{n-1}{k}$ e $\binom{n-1}{k-1}$.

Fissato un $x \in A$, $\binom{n-1}{k}$ \`e l'insieme di tutti i sottoinsiemi con $k$ elementi di $A \setminus \{ x \}$, $\binom{n-1}{k-1}$ \`e l'insieme di tutti i sottoinsiemi con $k - 1$ elementi di $A \setminus \{x \}$.
\end{prop}
\begin{proof}
$\forall \ S \subseteq A$ con $|S| = k$ e $x \in A$ ho due casi: $x \in S$ o $x \notin S$. Definisco la biezione $F : \mathbb{P}_k (A) \to \mathbb{P}_k (A \setminus \{ x \}) \cup \mathbb{P}_{k - 1} (A \setminus \{ x \})$ come:
\[
F (S) =
\begin{cases}
S \text{ se } x \notin S \\
S \setminus \{ x \} \text{ se } x \in S
\end{cases}
\]
Ho quindi che $\mathbb{P}_k (A) = X \cup Y$ con $X = \{ S \subseteq A : |S| = k \land x \in S \}$ e $Y = \{ S \subseteq A : |S| = k \land x \notin S\}$. Se tolgo $x$ da $A$, le cardinalit\`a dei due insiemi diventano $|X| = \binom{n - 1}{k - 1}$ e $|Y| = \binom{n - 1}{k}$.
\end{proof}

% TODO che vuol dire questo
% $\{ r^n \}_{n \in \mathbb{N}} \leftrightarrow \{ f : A_n \to R\} = \{ |R^{A_n}| \}$ con $|A_n| = n$ e $|R| = r$.

\begin{prop}
Quanto valgono i coefficienti binomiali? Diciamo questo:
\[
[r]_n = n! \cdot \binom{r}{n}
\]
da cui segue che:
\[
\binom{r}{n} = \frac{[r]_n}{n!} = \frac{r!}{n! \cdot {r-n}!}
\]
Stiamo dicendo quindi che $[r]_n$ (il numero di funzioni iniettive da $A$ in $R$ con $|A| = n$ e $|R| = r$) \`e uguale alla cardinalit\`a di un insieme di coppie $|E|$. $n!$ \`e il numero di elementi al secondo posto della coppia, $\binom{r}{n}$ \`e il numero di elementi al primo posto.
\end{prop}
\begin{proof}
$\binom{r}{n}$ \`e il numero di sottoinsiemi con $n$ elementi di un insieme con $r$ elementi.
\[
E = \{ (S, ?) \text{ con } |S| = n \land S \subseteq R \}
\]
Abbiamo visto che ogni funzione $f : A \to R$ individua un'insieme immagine $Im_f \subseteq R$ e una partizione $\ker f \in \Pi(A)$, e che questi due insiemi sono in corrispondenza biunivoca $|Im_f| = |\ker f |$.

Se prendo una funzione iniettiva, il nucleo della funzione \`e la partizione pi\`u piccola, avendo per elementi le classi rappresentate dai singoli elementi del dominio. Quante sono le iniezioni $In(A, S)$ con $S = Im_f$? $[n]_n = n!$

Quindi, $ E = \{ (S, \bar f : A \to S) $, con $ |S| = n \land S \subseteq R $ e $\bar f : A \to S$ insieme delle biezioni da $A$ in $S$ $\}$
\end{proof}

\subsection{Teorema binomiale}
\begin{prop}[Teorema binomiale\label{teorema_binomiale}]
\[
\forall \ n \in \mathbb{N} \text{ con } n > 0 \ 
(x + y)^n = 
\sum_{k = 0}^n \binom{n}{k} \ x^k y^{n - k}
\]
\end{prop}
Si possono dimostrare anche identit\`a polinomiali con le biezioni. Quando sono uguali due polinomi? Quando nel loro sviluppo hanno tutti i coefficienti uguali.
\begin{theorem}[Identit\`a polinomiale]
Dati due polinomi in $n$ variabili $p(x_1, \ \dots \ x_n)$ e $q(x_1, \ \dots \ x_n)$, sono uguali se coincidono su insiemi infiniti, ossia se $\forall \ (a_1, \ \dots \ a_n) \in I^n$ con $I \subseteq \mathbb{R} $ e $ |I| = \infty$, $p(a_1, \ \dots \ a_n) = q(a_1, \ \dots \ a_n)$, ossia i due polinomi coincidono ($p = q$).
\end{theorem}
Al posto di $x$ e di $y$ posso quindi mettere due interi $r, s \in \mathbb{N}$, e dimostrare che $(r + s)^n = \sum_{k = 0}^n \binom{n}{k} r^k s^{n - k}$.
\begin{proof}
Il membro sinistro della proposizione equivalente alla definizione \ref{teorema_binomiale}, $(r + s)^n$, \`e la cardinalit\`a dell'insieme di funzioni da un insieme con $n$ elementi ad uno con cardinalit\`a $r + s$, ossia all'unione disgiunta di due insiemi con cardinalit\`a rispettivamente $r$ e $s$. Dicendo quindi che gli insiemi $A$, $R$ e $S$ hanno cardinalit\`a $|A| = n$, $|R| = r $ e $|S| = s$, e che $T = R \cup S$, ho che $(r + s)^n = \left| T^A \right|$.

Il membro destro della proposizione \`e una somma, quindi corrisponde all'unione disgiunta di $n$ insiemi.
\[
\left| \bigsqcup_{k = 0}^{n} E_k \right| = \sum_{k = 0}^{n} \left| E_k \right|
\text{ con }
|E_k| = \binom{n}{k} \ r^k s^{n - k}
\]
Devo definire l'insieme $E_k$. Per il principio del prodotto, \`e una terna:
\[
\left( X, \ f_k : X \to R, \ g_{(n - k)} : (A \setminus X) \to S \right)
\]
con $X \subseteq A$ tale che $|X| = k$, $f_k$ l'insieme delle funzioni da $X$ in $R$, e $g_{(n-k)}$ l'insieme delle funzioni da $A \setminus X$ (ossia il complementare di $X$) in $S$.

Dobbiamo mostrare ora che $T^A \leftrightarrow \bigsqcup E_k$, ossia che c'\`e una biezione. La funzione biettiva $F$ associa ad una funzione $f : A \to T$ una terna:
\[
\left( f^{-1} (R), \ f_k : X \to R, \ g_{(n - k)} : (A \setminus X) \to S \right) \in E_k
\]
con $f^{-1}(R) = X$ a indicare la controimmagine di $R$ e $|f^{-1}(R)| = k$.
\end{proof}

\subsection{Coefficienti multinomiali}

Il coefficiente multinomiale, indicato con:
\[
\binom{n}{k_1 \dots k_t}
\]
\`e il numero delle $t$-scomposizioni di un insieme $A$ con $n$ elementi nella forma $(E_1, \dots, E_t)$ tali che $|E_1| = k_1, \dots |E_t| = k_t$.

\begin{defn}[$t$-scomposizioni]
Sia $A$ un insieme con $n$ elementi, una $t$-scomposizione di $A$ \`e una $t$-upla $(E_1, \dots E_t)$ tale che:
\begin{description}
  \item[1S] $\bigsqcup_{i = 1}^{t} E_i = A$
  \item[2S] $E_1 \cap E_j = \emptyset$ se $i \neq j$ 
\end{description}
$E_i$ \`e detto blocco.
\end{defn}
Ci sono due differenze fra una $t$-scomposizione e una partizione.
\begin{itemize}
  \item Un blocco di una scomposizione pu\`o anche essere vuoto, un blocco di una partizione \`e necessariamente $\neq \emptyset$.
  \item Considero un insieme $A = \{ a, b, c, d, e, f, g\}$. Posso avere una partizione:
  \[
  \pi = \{ \{a\}, \{b, g, f\}, \{c, d\}, \{e\} \}
  \]
  e due 4-scomposizioni distinte:
  \[
  (\{a\}, \{b, g, f\}, \{c, d\}, \{e\}) \neq 
  (\{a\}, \{c, d\}, \{b, g, f\}, \{e\})
  \]
  poich\'e le $t$-scomposizioni sono ordinate.
\end{itemize} 

Il coefficiente multinomiale \`e una generalizzazione del coefficiente binomiale.
\[
\binom{n}{k} = \binom{n}{k, n-k}
\]
Partendo da questa uguaglianza, so che:
\[
\binom{n}{k} = \binom{n}{k, n-k} = \frac{n!}{k! \ (n - k)!}
\Rightarrow
\binom{n}{k_1, \dots k_t} = \frac{n!}{k_1! \ \dots \ k_t!}
\]

\begin{prop}
Il numero delle $t$-scomposizioni di $A$ $(E_1, \dots E_t)$ di $A$ (indicato anche con $E_1 + \dots + E_t = A$) tali che $|E_1| = k_1, \dots |E_t| = k_t$ \`e:
\begin{equation}
\binom{n}{k_1, \dots k_t} = \frac{n!}{k_1! \ \dots \ k_t!}
\end{equation}
\end{prop}
\begin{proof}
Per $t = 2$ \`e vero. Supponiamo sia vero per $t > 2$. Prendiamo una $(t+1)$-scomposizione $(E_1, \dots, E_t, E_{t+1})$ di $A$. Ogni scomposizione individua una coppia:
\[
\left( E_{t+1}, (E_1, \dots E_t) \right)
\]
in cui $E_{t+1} \subseteq A$ e $(E_1, \dots E_t)$ \`e una $t$-scomposizione di $A \setminus E_{t+1}$, poich\'e $E_1 + \dots + E_t + E_{t+1} = A$.

In quanti modi posso scegliere un sottoinsieme $|E_{t+1}| = k_{t+1}$? In $\binom{n}{k_{t+1}}$ modi. Quindi quante coppie $(E_{t+1}, (E_1, \dots E_t))$ ho? \`E un prodotto, e per il principio del prodotto:
\[
\binom{n}{k_{t+1}} \cdot \frac{(n - k_{t+1})!}{k_1! \dots k_{t}!}
\]
Semplificando, si ottiene la tesi.
\end{proof}

\subsection{Teorema multinomiale}
\begin{prop}
\begin{equation}
(x_1 + \dots + x_t)^n = \sum_{(k_1, \dots k_t) | k_1 + \dots k_t = n}
\binom{n}{k_1, \dots, k_t} \ x_1^{k_1} \dots x_t^{k_t}
\end{equation}
Dove la sommatoria indica la somma di tutte le $t$-uple $(k_1, \dots k_t)$ tali che $k_1 + \dots + k_t = n$.
\end{prop}
\begin{proof}
Si dimostra di nuovo per biezione. Prendo $x_i = r_i \in \mathbb{N}$. Quindi:
\[
(r_1 + \dots + r_t)^n = \sum_{(k_1, \dots k_t) | k_1 + \dots k_t = n}
\binom{n}{k_1, \dots, k_t} \ r_1^{k_1} \dots r_t^{k_t}
\]
Il membro sinistro \`e uguale alla cardinalit\`a dell'insieme delle funzioni da $A$ in $T$ tali che $T = R_1 \cup \dots \cup R_t$, in cui $R_i $ con $i \in [1, t]$ insiemi a due a due disgiunti tali che $| R_i | = r_i$.

Ogni funzione individua una scomposizione di $A$ in $t$-blocchi, in cui ogni blocco \`e dato dalla controimmagine $f^{-1} (R_i)$ ed ha cardinalit\`a $k_i$. 

Il membro destro \`e dato quindi dall'unione disgiunta di una serie di insiemi di $(t+1)$-tuple. Ciascuna $(t+1)$-upla ha al primo posto una $t$-scomposizione di $A$, e all'$i+1$-esimo posto una funzione dal blocco $K_i$ nell'insieme $R_i$. Per il principio del prodotto, il numero di queste $(t+1)$-uple, per ogni possibile $t$-scomposizione, \`e proprio:
\[
\binom{n}{k_1, \dots, k_t} \ r_1^{k_1} \dots r_t^{k_t}
\]
Come nel caso del teorema binomiale, per dimostrare la biezione basta definire la funzione biettiva $F : T^A \to$ l'insieme di queste $(t+1)$-uple, che a ciascuna funzione $f : A \to T$ associa la $t$-scomposizione di $A$ in cui ogni blocco \`e definito dalla controimmagine $f^{-1} (R_i)$ e tutte $t$ funzioni che da un blocco $K_i$ vanno in $R_i$.
\end{proof}

\begin{cor}[Corollario del teorema multinomiale\label{corollario_multinomiale}]
\[
\sum_{k = 0}^{n} (-1)^k \binom{n}{k} = 0
\]
\end{cor}
\begin{proof}
\[
(x - 1)^n = \sum_{k = 0}^{n} \binom{n}{k} (-1)^k x^{n-k}
\]
Ponendo $x = 1$, per il teorema binomiale, ho la tesi.
\end{proof}

\subsection{Principio dei cassetti}

$R^A$ \`e l'insieme delle funzioni da $A$ in $R$. $|R^A| = r^n$. Una funzione $f : A \to R \in R^A$ la posso interpretare in due modi:
\begin{description}
  \item[Punto di vista dell'occupazione] $A$ viene interpretato come un insieme di oggetti. Il codominio $R$ viene interpretato come un insieme di cassetti. Quindi una funzione da $A$ in $R$ \`e un modo di disporre gli oggetti di $A$ nei cassetti di $R$.
  \item[Punto di vista della distribuzione] Rappresento l'insieme $A$ come un insieme di posti, e l'insieme $R$ come un insieme di lettere (ossia come un alfabeto). Una funzione $f : A \to R$ mi d\`a una parola, ossia un modo di disporre le lettere di $R$ nei posti di $A$.
\end{description}
\begin{prop}[Principio dei cassetti]
Se $|A| > |R|$ non esistono funzioni iniettive da $A$ in $R$.
\end{prop}
Dal punto di vista dell'occupazione, se si mettono $n$ oggetti in $r$ cassetti con $n > r$, allora almeno un cassetto conterr\`a pi\`u di un oggetto. 

Dal punto di vista della distribuzione, ogni parola di lunghezza $n$ con lettere in un alfabeto di cardinalit\`a $r$, con $n > r$, contiene almeno una lettera ripetuta.

I due punti di vista hanno una formalizzazione matematica distinta.

\begin{prop}
Dato un triangolo equilatero di lato 1, e dati 5 punti interni, almeno 2 di essi hanno distanza minore di $\frac{1}{2}$.
\end{prop}
\begin{proof}
Si pu\`o partizionare il triangolo in 4 blocchi come in figura \ref{triforza}, in modo che dati due punti qualsiasi in uno stesso blocco questi siano a meno di $\frac{1}{2}$ di distanza.
\begin{figure}[ht]
\centering
\begin{tikzpicture}
\draw (0,0) -- (2,0) -- (1,1.73)-- (0,0);
\draw[dashed] (1,0) -- (0.5,0.866) -- (1.5,0.866) -- (1,0);
\end{tikzpicture}
\caption{\label{triforza}Le quattro partizioni possibili}
\end{figure}
\end{proof}

\begin{prop}
Se si colorano i punti del piano con due colori (R,B) esisterà un rettangolo con tutti i vertici dello stesso colore.
\end{prop}
\begin{proof}
Se prendo tre punti allineati, so che almeno due hanno lo stesso colore. Se prendo nove terne di punti allineati, a loro volta allineate in modo da poter formare un rettangolo con ciascuna coppia di terne, so che almeno due terne hanno la stessa combinazione di colori.
\end{proof}
Prendo l'insieme $I = \{ 0, 1, 2 \}$ e definisco $X = I \times I \times I$. Definisco la relazione di equivalenza $\rho$ come:
\[
(x, y, z) \ \rho \ (x', y', z') \Leftrightarrow \{ x, y, z \} = \{ x', y', z'\}
\]
$\rho$ divide $X$ in classi di equivalenza. Quali sono gli elementi della classe $[(0,1,0)]$? Posso indicarla con $[\{0,1\}]$, ed \`e evidente che gli elementi della classe sono 6.

\subsection{Principio di inclusione/esclusione}

\begin{prop}
Sia $\Gamma$ un insieme e siano $A_1, \dots, A_t$ sottoinsiemi finiti di $\Gamma$.
\begin{equation}
| A_1 \cup A_2 \cup \ldots \cup A_t |  = 
\sum_{\emptyset \neq I \subseteq \{ 1, \ldots, t \}} (-1)^{|I| - 1} \left| \bigcap_{i \in I} A_i \right|
\end{equation}
\end{prop}
\begin{proof}
Si pu\`o dimostrare per induzione, o si pu\`o fare una dimostrazione combinatoria. La dimostrazione combinatoria calcola il contributo alla somma di ogni elemento dell'unione.

Prendiamo $x \in A_1 \cup \dots \cup A_t$ e siano $A_{j_1} \dots A_{j_n}$ i sottoinsiemi che contengono $x$, ossia se $i \neq j_1 , \dots j_n$ allora $x \notin A_i$. Il contributo che $x$ porta alla somma \`e evidentemente 1, perch\'e la cardinalit\`a dell'unione aumenta di uno.

Per $|I| = 1$ aggiunger\`o 1 per ogni sottoinsieme $A_{j_1} \dots A_{j_n}$, ossia per ogni sottoinsieme contenente $x$ poich\'e ogni altro $A_i$ non aggiunger\`a niente alla somma. Per $k > 1$ il contributo alla somma sar\`a 1 solo per le intersezioni fra i sottoinsiemi $A_{j_1} \dots A_{j_n}$, e 0 per ogni intersezione contenente un altro $A_i$. Quindi il contributo di $x$ alla somma sar\`a:
\[
\sum_{k = 1}^{n} (-1)^{k - 1} \ \binom{n}{k}
\]
So che, per il corollario \ref{corollario_multinomiale}:
\[
\sum_{k = 0}^{n} (-1)^k \ \binom{n}{k} = 0
\]
Quindi:
\[
1 + \sum_{k = 1}^{n} (-1)^{k} \ \binom{n}{k} = 
1 - \sum_{k = 1}^{n} (-1)^{k - 1} \ \binom{n}{k} \Rightarrow
\sum_{k = 1}^{n} (-1)^{k - 1} \ \binom{n}{k} = 1
\]
\end{proof}

\subsection{Numeri di Stirling di prima e seconda specie}

% Teorema di omomorfismo per gli insiemi.

% Qualunque funzione $f : A \to R$ individua una relazione di equivalenza, e quindi una partizione $\ker f$ chiamata ``nucleo di $f$''. Dati $x, y \in A$, ho che $x \ \varepsilon_f \ y \Leftrightarrow f(x) = f(y)$.

% Sappiamo che nucleo e immagine di $f$ sono in corrispondenza biunivoca. $|\ker f| = |Im_f|$.

Se indico con $S(n, k)$ (numero di Stirling di II specie) il numero delle partizioni in $k$-blocchi \textit{non vuoti} di un insieme con $n$ elementi, indicando con $Sur(A, R)$ l'insieme delle funzioni suriettive da $A$ in $R$, posso esprimere il numero delle funzioni suriettive $|Sur(A,R)| $ come $S(n,r) \cdot r!$, con $|A| = n$ e $|R| = r$.

Ogni funzione suriettiva $f : A \to R $ individua biunivocamente una coppia $(\pi, F)$ dove $\pi$ \`e una partizione di $A$ in $|R|$-blocchi, in cui ogni blocco (necessariamente non vuoto) \`e individuato da $f^{-1}(r) \forall \ r \in R$, ed $F$ \`e una biezione da $\pi \to R$. Quindi se voglio costruire una funzione suriettiva da $A$ in $R$, partiziono $A$ in $r$ blocchi \textit{non vuoti} e creo una biezione dai blocchi individuati in $A$ ad $R$.

Ora troviamo $|Sur(A,R)|$ usando il principio di inclusione/esclusione.

Possiamo trovare il numero delle funzioni suriettive dall'insieme di tutte le funzioni $R^A$ togliendo le funzioni non suriettive.

Supponiamo per semplicit\`a che $R = \{ 1, \ldots , r\} = [r]$. $f$ non \`e suriettiva $\Leftrightarrow \exists \ i = 1, \ldots, r$ tale che $i \notin Im_f$. Se indico con $A_i = \{ f : A \to R : i \notin Im_f \}$, l'unione $A_1 \cup \ldots \cup A_r$ \`e l'insieme delle funzioni non suriettive.
\[
Sur(A,R) = |R^A| - |A_1 \cup \dots \cup A_r|
\]
Calcoliamo $|A_1 \cup \ldots \cup A_r|$ con il principio di inclusione/esclusione.
\[
| A_1 \cup \ldots \cup A_r |  = 
\sum_{\emptyset \neq I \subseteq \{ 1, \ldots, r \}} (-1)^{|I| - 1} \left| \bigcap_{i \in I} A_i \right|
\]
Se prendo $I$ costituito da un solo elemento, $|A_1| = \left| (R \setminus \{ 1 \})^A \right| = (r - 1)^n$, quindi la cardinalit\`a del generico $A_i = (r-1)^n$.

Con $|I| = 2$, la cardinalit\`a dell'intersezione $|A_1 \cap A_2| = (r - 2)^n$. In generale $|A_{i_1} \cap \dots \cap A_{i_r}| = (r - k)^n$.

Quindi generalizzando:
\[
(-1)^0 \cdot \binom{r}{1} \cdot (r-1)^n + 
(-1)^1 \cdot \binom{r}{2} \cdot (r-2)^n \dots =
\sum_{k = 1}^{n} (-1)^{k - 1} \binom{r}{k} (r - k)^n
\]
Sapendo che $r^n$ \`e $(-1)^0 \binom{r}{0} \ (r - 0)^n$:
\begin{align*}
|Sur(A,R)| = 
r^n - \sum_{k = 1}^{n} (-1)^{k - 1} \binom{r}{k} (r - k)^n =  \\
r^n + \sum_{k = 1}^{n} (-1)^{k} \binom{r}{k} (r - k)^n =  \\
\sum_{k = 0}^{n} (-1)^{k} \binom{r}{k} (r - k)^n
\end{align*}
Quindi il numero di Stirling di II specie \`e:
\begin{equation}
S(n,r) = \frac{|Sur(A,R)|}{r!} = \frac{1}{r!} \sum_{k = 0}^{n} (-1)^{k} \binom{r}{k} (r - k)^n
\end{equation}

% SONO ARRIVATO QUI

Per il teorema di omomorfismo, data una funzione $f : A \to R$, $|\ker f| = |Im_f|$. Data la funzione $F : \ker f \to Im_f$ che ad ogni blocco associa l'immagine degli elementi del dominio nel bocco ($F[x] = f(x)$).

Prendo la funzione $\bar F : \ker f \to R$ tale che $\bar F [x] = f(x)$.

In genere non \`e biunivoca, a meno che la funzione $f$ sia suriettiva. Di norma \`e solo iniettiva.
\[
f : A \to R \leftrightarrow ( \ker f , \bar F : \ker f \to R \text{ iniettiva})
\]
Sono in corrispondenza biunivoca.
\[
r^n = \sum_{k = 1}^{r} S(n, k) \cdot [r]_k
\]
Come polinomio diventa:
\[
x^n = \sum_{k = 1}^{r} S(n, k) \cdot [x]_k
\]
Dall'algebra lineare si impara che la successione delle potenze e la successione dei fattoriali decrescenti si possono esprimere l'uno in funzione dell'altra, ed i coefficienti di cui sopra si possono invertire.
\[
[x]_n = \sum_{k = 1}^{r} s(n, k) x^k
\]
Con il coefficiente $s(n, k)$ ad indicare il numero di Stirling di prima specie.

\textbf{Esercizio:} 
\begin{enumerate}
  \item determinare il numero degli ``anagrammi'' (anche privi di senso) di MATEMATICA;
  \item determinare il numero degli anagrammi che contengono almeno una di queste sequenze:
  \begin{itemize}
    \item MATE;
    \item ATI;
    \item MAC.
  \end{itemize}
\end{enumerate}

Dal punto di vista della distribuzione, il dominio \`e un insieme di posti di cardinalit\`a 10, mentre il codominio \`e l'insieme delle lettere $\{ $A, C, E, I, M, T$\}$. La parola MATEMATICA \`e una funzione che associa ad ogni posto una lettera del codominio. Gli anagrammi di MATEMATICA sono tutte quelle funzioni tali per cui la controimmagine $|f^{-1}(A)| = 3$, $|f^{-1}(M)| = 2$ e $|f^{-1}(T)| = 2$.

Quale scomposizione dell'insieme dei posti \`e associata alla funzione MATEMATICA?

\begin{tabular}{cccccc}
A & C & E & I & M & T \\
$E_1$ & $E_2$ & $E_3$ & $E_4$ & $E_5$ & $E_6$ \\
$\{2, 6, 10\}$ & $\{9\}$ & $\{4\}$ & $\{8\}$ & $\{1,5\}$ & $\{3,7\}$ 
\end{tabular}

Quindi per trovare gli anagrammi devo trovare le scomposizioni tali che $|E_1| = 3$, $|E_2| = 1$, $|E_3| = 1$, $|E_4| = 1$, $|E_5| = 2$, $|E_6| = 2$.

La scomposizione $( \{ 7, 1, 2\}, \{ 10\}, \{3\}, \{ 6\}, \{4, 5\}, \{8,9\})$, ad esempio, individua l'anagramma AAEMMIATTC.

\[
\frac{10!}{3! 2! 2!}
\]

$A_1 = $ insieme degli anagrammi che contengono la sequenza MATE.
$A_2 = $ insieme degli anagrammi che contengono la sequenza ATI.
$A_3 = $ insieme degli anagrammi che contengono la sequenza MAC. $|A_1 \cup A_2 \cup A_3|$ \`e la risposta al secondo punto, e posso calcolarlo con il principio di inclusione ed esclusione.

Considero ogni sequenza come un'unica lettera. Nel primo caso il mio codominio adesso \`e $\{ $(MATE), A, C, I, M, T$\}$, ed ho 7 posti. $|E_2| = 2$.
\[
|A_1| = \frac{7!}{2!}
\]
Nel secondo caso il mio codominio adesso \`e $\{ $(ATI), A, C, I, M, T$\}$, ed ho 8 posti. $|E_2| = 2$ e $|E_5| = 2$.
\[
|A_1| = \frac{8!}{2! \ 2!}
\]
Nel terzo caso il mio codominio adesso \`e $\{ $(MAC), A, C, I, M, T$\}$, ed ho 8 posti. $|E_2| = 2$ e $|E_6| = 2$.
\[
|A_1| = \frac{8!}{2! \ 2!}
\]
Ora devo trovare le cardinalit\`a delle intersezioni $|A_1 \cap A_2|$, $|A_1 \cap A_3|$ e $|A_2 \cap A_3|$.
Nel primo caso il mio codominio adesso \`e $\{ $(MATE), (ATI), A, C, M$\}$, ed ho 5 posti.
\[
|A_1 \cap A_2| = 5!
\]
Nel secondo caso il mio codominio adesso \`e $\{ $(MATE), (MAC), A, I, T$\}$, ed ho 5 posti.
\[
|A_1 \cap A_3| = 5!
\]
Nel terzo caso il mio codominio adesso \`e $\{ $(ATI), (MAC), A, E, M, T$\}$, ed ho 6 posti.
\[
|A_2 \cap A_3| = 6!
\]
Ora devo trovare la cardinalit\`a dell'intersezione $|A_1 \cap A_2 \cap A_3|$. Il mio codominio adesso \`e $\{ $(MATE), (MAC), (ATI)$\}$, ed ho 3 posti.
\[
|A_1 \cap A_2 \cap A_3| = 3!
\]
Adesso possiamo applicare il principio di inclusione/esclusione.
\[
|A_1 \cup A_2 \cup A_3| = 
\frac{7!}{2!} + 2 \cdot \frac{8!}{2! 2!} - 2 \cdot 5! - 6! + 3!
\]

\textbf{Esercizio:} 
\begin{enumerate}
  \item determinare il numero degli ``anagrammi'' (anche privi di senso) di TRATTARE;
  \item determinare il numero degli anagrammi che contengono almeno una di queste sequenze:
  \begin{itemize}
    \item TRA;
    \item ATTA;
    \item ARE.
  \end{itemize}
\end{enumerate}

Abbiamo 8 posti. Il codominio \`e $\{$A, E, R, T$\}$. Il numero di anagrammi \`e:
\[
\frac{8!}{2! 2! 3!}
\]
$A_1 = $ insieme degli anagrammi che contengono TRA. 
$A_2 = $ insieme degli anagrammi che contengono ATTA. 
$A_3 = $ insieme degli anagrammi che contengono ARE. 
Nel primo caso il mio codominio adesso \`e $\{ $(TRA), A, E, R, T$\}$, ed ho 6 posti. $|E_4| = 2$.

Ma posso ottenere l'anagramma TRATRAET in due modi: (TRA), T, R, A, E, T e T, R, A, (TRA), E, T. Quindi devo togliere il numero di parole formate da $\{ $(TRA),E,T$\}$ con $|E_\text{TRA}| = 2$.
\[
|A_1| = \frac{6!}{2!} - \frac{4!}{2!}
\]
Nel secondo caso il mio codominio adesso \`e $\{ $(ATTA), E, R, T$\}$, ed ho 5 posti. $|E_3| = 2$.
\[
|A_2| = \frac{5!}{2!}
\]
Nel terzo caso il mio codominio adesso \`e $\{ $(ARE), A, R, T$\}$, ed ho 6 posti. $|E_4| = 3$.
\[
|A_3| = \frac{6!}{3!}
\]
Adesso dobbiamo procedere con le intersezioni.

Nel caso dell'intersezione $A_1 \cap A_2$, non posso avere la sequenza TRA e la sequenza ATTA separate, avendo solo due A nella parola TRATTARE. Quindi devo considerare TRATTA come una sola sequenza. Il codominio adesso \`e $\{ $(TRATTA), E, R$\}$. Ho 3 posti.
\[
|A_1 \cap A_2| = 3!
\] 
Per $A_1 \cap A_3$, il codominio \`e $\{ $(TRA), (ARE), T$\}$, con $|E_\text{T}| = 2$. Ho 4 posti. Ma posso anche avere la A in comune fra le due sequenze, e quindi avrei 4 posti ed un codominio $\{ $(TRARE), A, T$\}$ sempre con $|E_\text{T}| = 2$. Quindi:
\[
|A_1 \cap A_3| = \frac{4!}{2!} + \frac{4!}{2!} = 4!
\]
Per $A_2 \cap A_3$, come con $A_1 \cap A_2$, ATTA e ARE devono essere considerate come una sequenza sola. Il codominio \`e $\{$(ATTARE), T, R$\}$, con 3 posti. Quindi:
\[
|A_2 \cap A_3| = 3!
\]
L'unico modo per intersecare tutte e tre le sequenze \`e la sequenza TRATTARE. Quindi $|A_1 \cap A_2 \cap A_3| = 1$.

% coefficienti multinomiali

Dato un insieme $E$ ed una partizione tale per cui ogni classe ha lo stesso numero di elementi:
\[
\frac{|E|}{\text{numero delle classi}} = \text{ numero degli elementi di ciascuna classe}
\]
Viceversa:
\[
\frac{|E|}{\text{numero degli elementi di ciascuna classe}} = \text{ numero delle classi}
\]
Quindi una divisione pu\`o essere interpretata come il numero delle classi di equivalenza o come il numero degli elementi di ciascuna classe. So che il numero degli anagrammi di una parola di $n$ lettere, con ciascuna lettera ripetuta $k_i$ volte:
\[
\frac{n!}{k_1! \dots k_t!}
\]
$S_n$ \`e l'insieme delle permutazioni di $E$, ed ha cardinalit\`a $|S_n| = n!$. Negli anagrammi $n$ \`e il numero di posti. Quindi $n!$ \`e la cardinalit\`a dell'insieme delle permutazioni dei posti. Definisco una relazione di equivalenza sull'insieme delle permutazioni dei posti.

Ad ogni permutazione di posti corrisponde un anagramma, ma la relazione non \`e biunivoca. Devo definire una relazione di equivalenza sull'insieme delle permutazioni dei posti. 

Definisco una relazione di equivalenza $\rho$ sull'insieme delle permutazioni e dico che $\sigma \rho \tau \Leftrightarrow \sigma$ e $\tau$ individuano lo stesso anagramma.

Quindi $k_1! \dots k_t!$ \`e il numero di elementi in ogni classe di equivalenza.

\textbf{Esercizio:} $d_n = $ numero delle permutazioni di $E$ ($|E| = n$) senza punti fissi, ossia $\forall \ x \in E $ $ f(x) \neq x$. Supponiamo $E = \{1, 2, \dots n \} = [n]$.

Diciamo che $A_i$ \`e l'insieme delle permutazioni che fissano $i$ ($f(i) = i$). Quindi $A_1 \cup \dots \cup A_n$ \`e l'insieme delle permutazioni con almeno un punto fisso.

Quindi $d_n = |S_n| - |A_1 \cup \dots \cup A_n|$.

La cardinalit\`a di ogni $|A_i|$ \`e $(n - 1)!$. La cardinalit\`a di $A_i \cap A_j $ \`e $(n - 2)!$. In generale, la cardinalit\`a di $k$ insiemi distinti $|A_{i_1} \cap \dots \cap A_{i_k}| = (n - k)!$.

Andiamo a sostituire nella formula del principio di inclusione/esclusione.
\[
\sum_{\emptyset \neq I \subseteq \{1, \dots n \}} (-1)^{|I| - 1} \binom{n}{|I|} \cdot (n - |I|)! =
\binom{n}{1} \cdot (n-1)! - \binom{n}{2} \cdot (n - 2)! \dots
\]











\chapter{Strutture algebriche}


\begin{center}
\indent
\textit{Gruppi, anelli, campi. In particolare, anello degli interi modulo $m$ intero, anello dei polinomi.}
\end{center}

\section{Strutture algebriche con un'operazione}

Una struttura algebrica \`e una coppia $(A, \cdot)$ dove A \`e un'insieme e $\cdot$ \`e un'operazione $\cdot : A \times A \to A$. Ad esempio $(\mathbb{N}, +)$ \`e una struttura algebrica.

Le operazioni sono funzioni definite su prodotti cartesiani a valori in un insieme. Un'operazione binaria \`e definita sul prodotto cartesiano fra due insiemi.

Riprendendo la composizione, dati tre insiemi $A, B, C$, $B^A$ \`e l'insieme delle funzioni da $A$ in $B$, $C^B$ \`e l'insieme delle funzioni da $B$ in $C$. La composizione $\circ$ \`e un'operazione definita sul prodotto cartesiano degli insiemi $B^A \times C^B \to C^A$.

Posso rappresentare un'operazione come funzione $(\circ \left( f, g \right))$ o inserendo l'operatore fra i due operandi $ (g \circ f) $.
\begin{gather*}
f: \mathbb{R} \times \mathbb{R} \to \mathbb{R} ; \
f(x,y) = \sqrt{2} \cdot x + y \\
g: \mathbb{R} \to \mathbb{R} \times \mathbb{R} ; \
g(z) = (0,z) \\
g \circ f : \mathbb{R} \times \mathbb{R} \to \mathbb{R} \times \mathbb{R} \\
(x,y) \xrightarrow{f} \sqrt{2} \cdot x + y = z \xrightarrow{g} \left( 0, \sqrt{2}x + y \right) \\
f \circ g : \mathbb{R} \to \mathbb{R} \\
z \xrightarrow{g} (0,z) \xrightarrow{f} \sqrt{2} \cdot 0 + z = z
\end{gather*}

Una struttura algebrica \`e un'insieme su cui \`e definita un'operazione che prende due elementi di quell'insieme e gliene associa un terzo.

Le strutture vengono classificate in base alle loro propriet\`a:
\begin{description}
    \item[Propriet\`a associativa\label{itm:strutture_associativa}] $\forall \ a, b, c \in A : a \cdot (b \cdot c) = (a \cdot b) \cdot c$
    \item[Elemento neutro\label{itm:strutture_neutro}] Esistenza di un'elemento neutro, o elemento identit\`a. $1 \in A : \forall \ a \in A , a \cdot 1 = a = 1 \cdot a$
    \item[Propriet\`a commutativa\label{itm:strutture_commutativa}] $ \forall a, b \in A , a \cdot b = b \cdot a $. In una struttura algebrica commutativa in genere l'identit\`a si indica con 0.
    \item[Inverso\label{itm:strutture_inverso}] Esistenza dell'inverso. $ \forall a \in A \ \exists \ b \in A $ t.c. $a \cdot b = 1 = b \cdot a $.
\end{description}

\subsection{Classificazione delle strutture algebriche con una operazione}

Per essere studiabile, una struttura algebrica deve essere quantomeno associativa.

\begin{description}
    \item[Semigruppo] struttura algebrica associativa.
    \item[Monoide] struttura algebrica associativa con elemento identit\`a.
    \item[Gruppo] struttura algebrica associativa con elemento identit\`a e con inverso (ossia, monoide con inverso).
    \item[Gruppo abeliano] struttura algebrica che presenta tutte e quattro le propriet\`a: associativa, elemento neutro, commutativa, inverso.
\end{description}

La struttura algebrica $\left( \mathbb{N}, + \right)$ \`e un monoide commutativo. Anche $\left( \mathbb{N}, \cdot \right)$ \`e un monoide commutativo. $\left( \mathbb{Z}, + \right)$ \`e un gruppo perch\`e esiste l'inverso. $\left( \mathbb{Z}, \cdot \right)$ \`e un monoide, perch\'e non ha l'inverso.

\subsection{Gruppo simmetrico}

\begin{defn}[Gruppo simmetrico]
Il prototipo di tutti i gruppi \`e il gruppo simmetrico su $n$ elementi, il cui insieme \`e indicato con $S_n$. Prendiamo un insieme $E = \left\{ e_1, \dots, e_n \right\}$.
\[
S_n = \left\{ f : E \to E \text{ t.c. $f$ \`e biunivoca} \right\}
\]
Quindi $S_n$ \`e l'insieme di tutte le permutazioni degli elementi di $E$. Il gruppo simmetrico \`e definito sull'insieme $S_n$ e l'operazione \`e la composizione: $\left( S_n, \circ \right)$.
\begin{enumerate}
    \item $f \circ \left( g \circ h \right) = \left( f \circ g \right) \circ h$ 
    \item L'unit\`a \`e la funzione identica (o identit\`a) $i_E : E \to E$ tale che $\forall \ e \in E $ $i_E(e) = e$. $f \circ i_E = f = i_E \circ f$ 
    \item Una funzione biunivoca $f$ ha una funzione inversa $g$. $g : E \to E $ t.c. $ g(f(e)) = e$.
\end{enumerate}
\end{defn}

Una funzione $f : E \to E $ iniettiva su un insieme finito $E$ \`e necessariamente suriettiva e quindi biunivoca. Un insieme \`e finito se non pu\`o essere messo in corrispondenza biunivoca con un suo sottoinsieme proprio.

% \subsection{Punto di vista dell'occupazione}

% $f : \left \{ 1, \dots, 6 \right \} \to \left \{ 1, \dots, 6 \right \}$. Penso il dominio come degli oggetti. Il codominio come dei ``cassetti''. La funzione \`e un modo di mettere gli oggetti del dominio nei ``cassetti''.

% \begin{tabular}{cccccc}
% 1 & 2 & 3 & 4 & 5 & 6 \\
% 2 & 3 & 5  &1 & 6 & 4
% \end{tabular}
% \`E un'occupazione.

% \begin{tabular}{cccccc}
% 1 & 2 & 3 & 4 & 5 & 6 \\
% 6 & 6 & 3 & 5 & 5 & 5
% \end{tabular}
% Non \`e un'occupazione.


\section{Monoidi}

Un monoide $(M, \cdot)$ \`e una struttura algebrica con un'operazione $\cdot : M \times M \to M$ tale che:
\begin{description}
    \item[1M] L'operazione $\cdot$ \`e associativa;
    \item[2M] $\exists \ 1_M $ t.c. $ \forall \ a \in M$ $ 1_M \cdot a = a = a \cdot 1_M$, ossia esiste l'elemento identit\`a. 
\end{description}

Un sottomonoide $(S, \cdot)$ con $S \subseteq M$ \`e un monoide in cui esiste l'operazione $\cdot : S \times S \to S$, ossia $S$ \`e chiuso, cio\`e $\forall s, s' \in S $, $s \cdot s' \in S$ e $1_M \in S$.

Ad esempio, considerando $(\mathbb{N}, +)$, $(P, +)$ con $P = \{ m \in \mathbb{N} : \exists \ k $ t.c. $m = 2k \}$ \`e un sottomonoide di $(\mathbb{N},+)$ perch\`e la somma di due pari \`e pari e lo 0 appartiene ai pari.

$k \mathbb{N} = \{ m \in \mathbb{N} : \exists \ t \in \mathbb{N} $ t.c. $ m = k t\}$ \`e la ``versione generale'' dell'insieme dei numeri pari.

Considerando $(\mathbb{N}, \cdot)$ e l'elemento neutro 1, i pari non sono un sottomonoide perch\'e non hanno l'elemento neutro, ma i dispari s\`i.

\subsection{Morfismi di monoidi}

\begin{defn}
Dati i monoidi $(M, \cdot)$ e $(A, \ast)$, un morfismo di monoidi \`e un'applicazione $f : M \to A$ che conserva le strutture, ossia tale che $\forall \ x,y \in M $ ho che $f(x \cdot y) = f(x) \ast f(y)$. Inoltre, $f(1_M) = 1_A$.
\end{defn}

\subsection{Teorema di omomorfismo per i monoidi\label{omomorfismo_monoidi}}

\begin{prop}
Sia $f : (M, \cdot) \to (A, \ast)$ un morfismo di monoidi, $f$ definisce una relazione di equivalenza $\varepsilon_f$ tale che $\ker f = M / \varepsilon_f \cong Im_f$, ossia il quoziente \`e isomorfo all'immagine. Inoltre il quoziente ha una struttura di monoide:
\[
(M / \varepsilon_f , \cdot) \cong (Im_f, \ast)
\]
con $(Im_f, \ast)$ sottomonoide di $(A, \ast)$.
\end{prop}
\begin{proof}
Ogni morfismo di monoidi $f : (M, \cdot) \to (A, \ast)$ individua un sottomonoide di $(A, \ast)$, che \`e $(Im_f, \ast)$. 

Essendo $f$ un morfismo di monoidi, $\forall f(x), f(y) \in Im_f$, $f(x) \ast f(y) = f(x \cdot y) \in Im_f$, quindi $Im_f $ \`e chiuso rispetto a $\ast$. Devo poi verificare che $1_A \in Im_f \Leftarrow f(1_M) = 1_A$.

Anche $(\ker f, \cdot)$ \`e un monoide. Dobbiamo dimostrare l'esistenza dell'isomorfismo con $Im_f$.

$\ker f$ \`e l'insieme delle classi di equivalenza $[x] = \{ y \in M : x \ \varepsilon_f \ y \Leftrightarrow f(x) = f(y) \} \in \ker f$.

Definiamo l'operazione di prodotto fra classi come la classe del prodotto di due rappresentanti $[x] \cdot [z] = [x \cdot z]$ qualsiasi rappresentante scelgo della classe. Bisogna verificare che questa definizione sia indipendente dai rappresentnati! Lo faremo nella sezione \ref{congruenze}, in particolare nella dimostrazione \ref{congruenza_monoidi}. 

L'operazione fra classi \`e associativa, perch\'e \`e associativa l'operazione fra rappresentanti. Inoltre ho l'unit\`a $[1_M]$ in $\ker f$.

L'isomorfismo fra $(Im_f, \ast)$ e $(\ker f, \cdot)$ segue naturalmente dal fatto che $\ker f $ e $Im_f$ sono in biezione. Inoltre, essendo entrambi dei monoidi, la funzione $f$ \`e un isomorfismo di monoidi.
\end{proof}

\subsection{Potenze (iterazioni sui monoidi)}

\begin{defn}[Potenze]
A partire dal monoide $(M, \cdot)$ possiamo definire le iterazioni dell'operazione $\cdot$, ossia le potenze.

Sia $ a \in M$, si definisce:
\begin{enumerate}
    \item $a^0 = 1_M$
    \item $a^{n+1} = a \cdot a^{n}$
\end{enumerate}
\end{defn}
\begin{prop}[Commutativit\`a della potenza]
$\forall \ n \in \mathbb{N}$, $a \cdot a^n = a^n \cdot a$, ossia la potenza \`e commutativa.
\end{prop}
\begin{proof}
Si dimostra per induzione. Si vede subito che con $n = 0$, per definizione $a \cdot a^0 = a = a^0 \cdot a$.

Per definizione di potenza $a \cdot a^{n+1} = a \cdot a \cdot a^{n} $, per ipotesi induttiva $ a \cdot a^n \cdot a $ che di nuovo per definizione di potenza \`e $ a^{n+1} \cdot a$.
\end{proof}
\begin{prop}
Valgono tutte le propriet\`a tipiche delle potenze:
\[
a^{m + n} = a^m \cdot a^n
\]
\end{prop}
\begin{proof}
Si dimostra anche questo per induzione su $n$. Con $n = 0$, $a^{m+0} = a^m = a^m \cdot 1_M = a^m \cdot a^0$.

Passo induttivo. $a^{m + n + 1} = a \cdot a^{m + n}$ per definizione di potenze. Applicando l'ipotesi induttiva, $a \cdot a^{m + n} = a \cdot a^m \cdot a^n$. Per commutativit\`a $a \cdot a^m \cdot a^n = a^m \cdot a \cdot a^n = a^m \cdot a^{n+1}$.
\end{proof}

\begin{theorem}
Dato un monoide $(M, \cdot)$ ed un elemento $a \in M$, esiste un solo morfismo di monoidi $f : (\mathbb{N}, +) \to (M, \cdot)$ tale che $f(1) = a$, ed \`e $f(n) = a^n$.
\end{theorem}
\begin{proof}
$f$ \`e un morfismo di monoidi, quindi deve verificare che $f(m+n) = f(m) \cdot f(n)$ e che $f(0) = 1_M$. 

Per le propriet\`a delle potenze dimostrate precedentemente, $f(m+ n) = a^{m+n} = a^m \cdot a^n = f(m) \cdot f(n)$, e $f(0) = a^0 = 1_M$ per la definizione delle potenze. 

Inoltre verifica la condizione $f(1) = a$, infatti $f(1) = a^1 = a \cdot a^0 = a \cdot 1_M = a$.

Dobbiamo dimostrare l'unicit\`a di $f$. Sia $g : (\mathbb{N}, +) \to (M, \cdot )$ un morfismo tale che $g(1) = a$, dimostriamo che $\forall n \in \mathbb{N} $ $ g(n) = f(n) = a^n$.

Dimostriamolo per induzione su $n$. Per definizione di morfismo di monoidi, $g(0) = 1_M = f(0) = a^0$.

Supponiamo che $g(n) = f(n)$, per definizione di morfismo di monoidi $g(n+1) = g(1) \cdot g(n) = a \cdot g(n) = a \cdot f(n) = a \cdot a^n = a^{n+1}$.
\end{proof}

\begin{exmp}
Sia $\Gamma$ un insieme, la struttura algebrica $(\mathbb{P}(\Gamma), \cup)$ \`e in particolare un monoide. L'unione \`e associativa ($(A \cup B) \cup C = A \cup (B \cup C)$), ed esiste l'elemento neutro $\emptyset$.

Anche $(\mathbb{P}(\Gamma), \cap)$ \`e un monoide, con $\Gamma$ come elemento neutro, poich\'e $\forall \ A$, $\Gamma \cap A = A$. Abbiamo quindi due esempi di monoidi commutativi.

Fissato un insieme $S \subseteq \Gamma$ diverso da $\emptyset$, possiamo considerare il suo insieme delle parti $\mathbb{P}(S)$ e definire l'applicazione $f : \mathbb{P}(\Gamma) \to \mathbb{P}(S)$ tale che $f(A) = A \cap S$. 

Verifichiamo che questa applicazione \`e un morfismo di monoidi rispetto a $(\mathbb{P}(\Gamma), \cup)$ e $(\mathbb{P}(S), \cup)$. Dobbiamo dimostrare che $f( A \cup B) = f(A) \cup f(B)$. Infatti $f(A \cup B) = (A \cup B) \cap S = (A \cap S) \cup (B \cap S)$ per la propriet\`a distributiva, che \`e proprio $f(A) \cup f(B)$.

Inoltre l'applicazione conserva l'elemento neutro, poich\'e $f(\emptyset) = \emptyset \cap S = \emptyset$.

Verifichiamo che \`e un morfismo di monoidi anche rispetto $(\mathbb{P}(\Gamma), \cap)$ e $(\mathbb{P}(S), \cap)$. $f(A \cap B) = f(A) \cap f(B)$, infatti $(A \cap B) \cap S = (A \cap S) \cap (B \cap S)$ sempre per la propriet\`a distributiva. E anche in questo caso l'applicazione conserva l'elemento neutro, poich\'e $f(\Gamma) = \Gamma \cap S = S$, ed $S$ \`e proprio l'elemento neutro di $(\mathbb{P}(S), \cap)$.
\end{exmp}

\begin{exmp}
Sia $f : \mathbb{P}(\Gamma) \to \mathbb{P}(\Gamma)$ un'applicazione tale che $f(A) = \overline{A} = \{ x \in \Gamma : x \notin A\}$, ossia che associa ad $A$ il suo complementare $\overline{A}$. L'applicazione $f : (\mathbb{P}(\Gamma), \cup) \to (\mathbb{P}(\Gamma), \cap)$ \`e un morfismo di monoidi visto che verifica $f(A \cup B) = f(A) \cap f(B)$ per le leggi di De Morgan ($\overline{A \cup B} = \overline{A} \cap \overline{B}$) e $f(\emptyset) = \overline{\emptyset} = \Gamma$.

Possiamo considerare la stessa applicazione come un morfismo di monoidi da $f : (\mathbb{P}(\Gamma), \cap) \to (\mathbb{P}(\Gamma), \cup)$. Infatti $f(A \cap B) = \overline{A \cap B} = \overline{A} \cup \overline{B} = f(A) \cup f(B)$ e $f(\Gamma) = \overline{\Gamma} = \emptyset$.
\end{exmp}

\subsection{Congruenze\label{congruenze}}

\begin{defn}
Le congruenze sono relazioni d'equivalenza definite sulle strutture algebriche. Sia $\varepsilon$ una relazione d'equivalenza su $A$, con $(A, \cdot)$ monoide, si dice che $\varepsilon$ \`e una congruenza rispetto all'operazione $\cdot$ se, dati $a \ \varepsilon \ b$ e $c \ \varepsilon \ d$, ho che $a \cdot c \ \varepsilon \ b \cdot d$.
\end{defn}

Vuol dire che, dati $a \in [a]$ e $c \in [c]$, se $a \cdot c \in [a \cdot c]$ e ho una congruenza, allora $b \in [a]$ e $d \in [c]$ sono tali che $b \cdot d \in [a \cdot c]$. La congruenza fa s\`i che io possa definire operazioni sulle classi.

\begin{prop}
Riprendendo il teorema \ref{omomorfismo_monoidi}, abbiamo che considerato un morfismo di monoidi $f : (M, \cdot) \to (A, \ast)$, se definisco la relazione di equivalenza $\varepsilon_f$ tale che $x, y \in M$ sono $x \ \varepsilon_f \ y \Leftrightarrow f(x) = f(y)$, questa relazione di equivalenza \`e una congruenza.
\end{prop}
\begin{proof}\label{congruenza_monoidi}
$x \ \varepsilon_f \ y$, $z \ \varepsilon_f \ w \Rightarrow (x \cdot z) \ \varepsilon_f \ (y \cdot w)$.

Infatti $f(x \cdot z) = f(x) \ast f(z)$ e $f(y \cdot w) = f(y) \ast f(w)$. 
\end{proof}

Avevamo definito $\rho$ su $\mathbb{N} \times \mathbb{N}$, come $(a, b) \ \rho \ (c, d) \Leftrightarrow a+d = b+c$. Quindi a partire da $(M, \cdot) $ possiamo creare altri monoidi $(M^n, \cdot)$, ad esempio su $M^2 = M \times M$ in cui $(x, y) \cdot (z, t) = (x \cdot z, y \cdot t)$.

Ad esempio $(\mathbb{N}, +) \to (\mathbb{N} \times \mathbb{N}, +)$ in cui $(m, n) + (a, b) = (m+a, n+b)$ con l'elemento neutro $(0,0)$.

$\rho$ \`e una congruenza rispetto a + in $\mathbb{N} \times \mathbb{N}$. Inoltre, avendo visto che $\mathbb{N} \times \mathbb{N} / \rho = \mathbb{Z}$, abbiamo che $(\mathbb{Z}, +)$ \`e un monoide.

% WEB

\section{Gruppi}

\begin{defn}
$(G, \cdot)$ \`e un gruppo se:
\begin{description}
    \item[1G] $(G, \cdot)$ \`e un monoide
    \item[2G] $\forall \ a \in G $, $ \exists \ b \in G $ tale che $a \cdot b = 1_G$, con $b$ comunemente indicato come $a^{-1}$ e detto inverso di $a$.
\end{description}
\end{defn}

Possiamo definire un morfismo di gruppi. Un morfismo conserva strutture e propriet\`a, deve quindi essere un morfismo di monoidi che manda l'inverso nell'inverso.
\begin{defn}[Morfismo di gruppi]
Quindi $f : (G, \cdot) \to (G', \ast)$ \`e un morfismo di gruppi se:
\begin{enumerate}
    \item \`e un morfismo di monoidi
    \item $\forall a \in G f(a^{-1}) = (f(a))^{-1}$
\end{enumerate}
\end{defn}
\begin{prop}
Queste due propriet\`a sono la conseguenza di una sola, ossia che il morfismo conserva le operazioni. Infatti se $f(a \cdot b) = f(a) \ast f(b)$ allora sono vere tutte le propriet\`a.
\end{prop}
\begin{proof}
Un morfismo che conserva le operazioni manda le unit\`a nelle unit\`a: $f(1_G) = f(1_G \cdot 1_G) = f(1_G) \ast f(1_G)$. Essendo entrambi gruppi hanno l'inverso, quindi moltiplicando entrambi i lati per l'inverso di $f(1_G)$ abbiamo $(f(1_G))^{-1} \ast f(1_G) = (f(1_G))^{-1} \ast f(1_G) \ast f(1_G) \Rightarrow 1_{G'} = f(1_G)$.

Inoltre, se il morfismo conserva le operazioni manda gli inversi negli inversi, ossia $f(a^{-1}) = (f(a))^{-1}$. Infatti $f(a) \ast f(a^{-1}) = f(a \cdot a^{-1}) = f(1_G) = 1_{G'} = f(a) \ast (f(a))^{-1}$.
\end{proof}

\subsection{Sottogruppi}

\begin{defn}[Sottogruppo]
Partendo da $(G, \cdot)$ e scegliendo $S \subseteq G$ diverso da $\emptyset$, un sottogruppo $(S, \cdot)$ deve essere:
\begin{itemize}
    \item Chiuso: $\forall \ s, s' \in S$, $s \cdot s' \in S$
    \item Deve contenere l'unit\`a: $1_G \in S$ (quindi $S$ \`e un sottomonoide di $G$)
    \item Per essere anche un sottogruppo, $S$ deve essere chiuso rispetto agli inversi: $s \in S \Rightarrow s^{-1} \in S$.
\end{itemize}
\end{defn}
\begin{prop}
Condizione necessaria e sufficiente affinch\'e $(S, \cdot)$ con $S \neq \emptyset$ sia un sottogruppo del gruppo $(G, \cdot)$ \`e:
\[
a, b \in S \Rightarrow a^{-1} \cdot  b \in S
\]
\end{prop}
\begin{proof}
Dimostrare che \`e condizione necessaria \`e banale. Per definizione di sottogruppo $a^{-1}$ \`e in $S$, ed essendo chiuso $a^{-1} \cdot b \in S$.

Dobbiamo dimostrare che \`e sufficiente. $S \neq \emptyset$, quindi ha almeno un elemento $a \in S$. Prendiamo $b = a $, per la propriet\`a indicata sopra $a \cdot a^{-1} \in S \Rightarrow 1_G \in S$. Quindi $S$ contiene almeno l'elemento neutro.

Contiene l'inverso: $\forall \ x \in S $, $ x^{-1} \in S$ sempre per la propriet\`a sopra. Infatti prendendo $a = x$ e $b = 1_G$, $a^{-1} \cdot b \in S$ ossia $x^{-1} \cdot 1_G \in S \Rightarrow x^{-1} \in S$.

\`E chiuso: $\forall \ s, s' \in S \Rightarrow s \cdot s' \in S$. Abbiamo appena visto che $s \in S \Rightarrow s^{-1} \in S$, quindi per la solita propriet\`a ho che $(s^{-1})^{-1} \cdot s' \in S \Rightarrow s \cdot s' \in S$.
\end{proof}

% SONO ARRIVATO QUI

% $(\mathbb{Z}, +) \to (\mathbb{N} \times \mathbb{N}, +) / \rho$.

I sottogruppi di $(\mathbb{Z}, +)$ sono tutti e solo i gruppi $(k \mathbb{Z}, +)$. $(\mathbb{Z}, \cdot)$ non \`e un gruppo perch\'e non ha l'inverso per ogni elemento.

Dati $a, b \in G$, voglio conoscere l'inverso del prodotto $a \cdot b$, ossia $(a \cdot b)^{-1}$. Solitamente un gruppo $(G, \cdot)$ non \`e commutativo. Solo se il gruppo \`e commutativo ho che $(a \cdot b)^{-1} = a^{-1} \cdot b^{-1}$.

Consideriamo il gruppo simmetrico $(S_n, \circ)$, definito sull'insieme delle funzioni iniettive da un insieme con $n$ elementi in s\'e stesso con l'operazione di composizione. Non \`e un gruppo commutativo.

Prendiamo le due funzioni $\sigma$ e $\tau$ dal punto di vista dell'occupazione in figura \ref{fig:gruppo_simmetrico}. Se voglio trovare l'inverso $pi$ di $\sigma \circ \tau$ tale che $(\sigma \circ \tau) \circ \pi = i$ devo usare $\pi = (\tau^{-1} \circ \sigma^{-1})$.

\begin{figure}[ht]
\centering
\begin{tabular}{cccc}
\multicolumn{4}{c}{$\sigma$} \\
\hline
1 & 2 & 3 & 4 \\
2 & 3 & 1 & 4
\end{tabular} \qquad
\begin{tabular}{cccc}
\multicolumn{4}{c}{$\tau$} \\
\hline
1 & 2 & 3 & 4 \\
1 & 2 & 4 & 3
\end{tabular}

\begin{tabular}{cccc}
\multicolumn{4}{c}{$\sigma \circ \tau$} \\
\hline
1 & 2 & 3 & 4 \\
1 & 2 & 4 & 3 \\
2 & 3 & 4 & 1
\end{tabular} \qquad
\begin{tabular}{cccc}
\multicolumn{4}{c}{$\tau \circ \sigma$} \\
\hline
1 & 2 & 3 & 4 \\
2 & 3 & 1 & 4 \\
2 & 4 & 1 & 3
\end{tabular}
\caption{\label{fig:gruppo_simmetrico}Il gruppo simmetrico non \`e commutativo}
\end{figure}

Quindi, l'inverso del prodotto \`e il prodotto degli inversi scambiati di posto:
\[
(a \cdot b) \cdot (b^{-1} \cdot a^{-1}) = a \cdot (b \cdot b^{-1}) 
\cdot a^{-1} = a \cdot 1_G \cdot a^{-1} = 1_G
\]

Un'applicazione $f : G \to G'$ \`e un morfismo di gruppi se $\forall a, b \in G$ ho che $f(a \cdot b) = f(a) \ast f(b) \Rightarrow f(1_G) = 1_{G'}$ e $f(a^{-1}) = (f(a))^{-1}$.

\begin{exmp} 
La funzione $\log : (\mathbb{R}^{+}, \cdot) \to (\mathbb{R}, +)$ \`e un morfismo di gruppi, perch\'e $\log(a \cdot b) = \log(a) + \log(b)$.

Anche la funzione $\exp : (\mathbb{R}, +) \to (\mathbb{R}^{+}, \cdot)$ \`e un morfismo di gruppi, infatti  $\exp(a + b) = \exp(a) \cdot \exp(b)$.

Anche l'iterazione della somma \`e un morfismo di gruppi. $f_n : (\mathbb{Z}, +) \to (\mathbb{Z}, +) $, infatti $\forall \ z \in \mathbb{Z}$, $f_n(z) = n \cdot z$
\end{exmp}

\subsection{Nucleo di un morfismo di gruppi}

Ogni morfismo di gruppi $f$ individua due sottogruppi:
\begin{itemize}
    \item $Im_f \subseteq G'$
    \item $\ker f \subseteq G$. Diversamente dalle definizioni gi\`a viste, in questo caso il nucleo $\ker f = \{ u \in G : f(u) = 1_{G'}\}$ \`e una classe, ossia $\ker f \in G / \varepsilon_f$
\end{itemize}

Con i monoidi avevamo una struttura associativa $(M, \cdot)$ contenente l'unit\`a $1_M$. Il $\ker f$ l'abbiamo chiamato ``quoziente'', ossia:
\[
\ker f = M / \varepsilon_f
\]

\begin{figure}[ht]
\centering
\begin{tikzpicture}
  \node (A) {$A$};
  \node (f) [right of=A, node distance=2cm] {$f$};
  \node (B) [right of=f, node distance=2cm] {$B$};
  \node (a) [below of=A, node distance=1cm] {$a$};
  \node (b) [below of=a, node distance=1cm] {$b$};
  \node (c) [below of=b, node distance=1cm] {$c$};
  \node (d) [below of=c, node distance=1cm] {$d$};
  \node (e) [below of=d, node distance=1cm] {$e$};
  \node (1) [below of=B, node distance=1cm] {$1$};
  \node (2) [below of=1, node distance=1cm] {$2$};
  \node (3) [below of=2, node distance=1cm] {$3$};
  \node (4) [below of=3, node distance=1cm] {$4$};
  \node (5) [below of=4, node distance=1cm] {$5$};
  \node (6) [below of=5, node distance=1cm] {$6$};
  \node (7) [below of=6, node distance=1cm] {$7$};
  \path[->]  (a) edge node {} (3)
            (b) edge node {} (3)
            (c) edge node {} (4)
            (d) edge node {} (4)
            (e) edge node {} (7)
            ;
\end{tikzpicture}
\caption{\label{fig:esempio_funzione}$\ker f = \{ \{a, b \}, \{ c, d \}, \{ e \}\}$ }
\end{figure}

Nel caso in figura \ref{fig:esempio_funzione} $\ker f = \{ \{a, b \}, \{ c, d \}, \{ e \}\}$. Ho \textit{bisogno} di sapere quali classi ci sono per ricostruire la funzione. Per conoscere la $f$ devo sapere tutti i blocchi della partizione, non posso ricostruire gli altri blocchi da un blocco solo.

Con i gruppi non \`e cos\`i. Mi basta la classe degli elementi che vanno nell'unit\`a. Se conosco questa classe le conosco tutte. Infatti fissato il $\ker f$ conosco tutti gli elementi che hanno la stessa immagine di $a$, ossia $a \cdot \ker f = [a]$.

Abbiamo un morfismo di gruppi $ f : (G, \cdot ) \to (G', \ast)$. Dimostriamo intanto che il nucleo \`e un sottospazio, ossia un sottogruppo.

$\ker f = [1_G]$ conosco tutti gli elementi che finiscono nell'unit\`a di $G'$

CNES afficnhe un sottoinsieme sia un sottogruppo \`e che se $a, b \in S \Rightarrow a^{-1} b \in S$.

Prendiamo $u, v \in \ker f$. La tesi \`e che $u^{-1} \cdot v \in \ker f$. Quindi $f ( u^{-1} \cdot v ) = 1_{G'}$.
\[
f( u^{-1} \cdot v ) = f(u^{-1}) \ast f(v) = f(u)^{-1} \ast f(v)
\]
Ma $f(u)^{-1} = 1_{G'}$, quindi ho:
\[
f(u)^{-1} \ast f(v) = 1_{G'} \ast 1_{G'} = 1_{G'}
\]

\subsection{Teorema di omomorfismo per i gruppi}

Sia $f : (G, \cdot) \to (G', \ast)$ un morfismo di gruppi, allora $\varepsilon_f$ \`e una congruenza e il gruppo $G / \varepsilon_f$ \`e isomorfo al gruppo $(Im_f, \ast)$. Ogni elemento $[a] \in G / \varepsilon_f$ \`e del tipo $a \cdot \ker f$.
\[
F : (G / \varepsilon_f, \cdot) \to (Im_f, \ast)
\]
\[
\forall [a] \in G / \varepsilon_f [a] = a \cdot \ker f
\]
Dimostrazione:
$b \in [a] \Rightarrow f(a) = f(b)$ per definizione hanno la stessa immagine tramite il morfismo.

Quindi moltiplico entrambi i membri per $f(a)^{-1}: f(a)^{-1} \ast f(b) = 1_{G'} \Rightarrow 1_{G'} = f(a)^{-1} \ast f(b) = f(a^{-1} \cdot b)$

Quindi $u = (a^{-1} \cdot b ) \in \ker f e b = a \cdot u = a \cdot (a^{-1} \cdot b)$

Viceversa, dobbiamo prendere $b \in a \cdot \ker f \Rightarrow b = a \cdot u$. Ha per forza la stessa immagine di $a$, infatti $f(b) = f(a \cdot u) = f(a) \cdot f(u) = f(a) \cdot 1_G = f(a)$. Segue che $b$ \`e nella classe di $a, b \in [a]$.

Non essendo commutativo ho che $b = a \cdot u = v \cdot a$. Posso vedere $b \in a \cdot \ker f$ sia in $\ker f \cdot a$.

La prima \`e la classe laterale sinistra, la seconda \`e la classe laterale destra.
\[
\forall a \in G a \cdot ker f = \ker f \cdot a
\]
\[
b = u \cdot a
\]
\[
f(b) = f(u) \cdot f(a) = 1_{G'} \cdot f(a)
\]
Proiezioni. La seguente \`e la prima proiezione
\[
p_1 : \mathbb{R} \times \mathbb{R} \to \mathbb{R}
\]
\[
p_1 (x, y) = x
\]
\[
p_1 : (\mathbb{R} \times \mathbb{R}, +) \to (\mathbb{R}, +)
\]
Qual \`e l'immagine $Im_{p_1}$? $Im_{p_1} = \{ r \in \mathbb{R} : \exists (x, y) p_1 (x, y) = r \}$. In questo caso l'immagine \`e tutto $\mathbb{R}$. Infatti $\forall r \in \mathbb{R} p_1(r, 0) = r$.

Adesso troviamo il nucleo.
\[
\ker f = \{ (x, y) \in \mathbb{R} \times \mathbb{R} : p_1(x, y) = 0 \} = \{ (0, y) : y \in \mathbb{R} \}
\]
L'unit\`a di $G 1_G = (0, 0) \in \ker p_1$

0 \`e l'elemento neutro di $(\mathbb{R}, +)$.
\[
(2, 3) \in \mathbb{R} \times \mathbb{R}, p_1(2, 3) = 2
\]
\[
[(2, 3)] = \{ (x, y) \in \mathbb{R} \times \mathbb{R} : f(x, y) = 2\}
\]
Applicando il teorema di omomorfismo visto prima vediamo subito che \`e:
\[
[(2,3)] = (2, 3) + \ker p_1 = \{ (2, 3) + (0, y) = (2, y + 3) \}
\]
Altro esempio:
$\mathbb{R}[x] =$ insieme dei polinomi in una indeterminata $x$

un polinomio \`e una espressione formale del tipo $a_0 + a_1 x + \dots a_n x^n$ dove $a_n \neq 0$ e $n$ si dice grado del polinomio.

C'\`e una differenza fra $x \to$ indeterminata e $x \to $ variabile. L'indeterminata fa parte dei polinomi, e vuol dire che $x$ \`e un simbolo. Variabile vuol dire che \`e un elemento di un insieme, ossia $x \in E$. In genere si confonde indeterminata con variabile, perch\'e quando si parla di polinomi la $x$ \`e indeterminata, ma ogni polinomio individua una funzione polinomiale. $p : R \to R a \mapsto p(a)$
\[
p(x) = 1 + 2x
\]
individua la funzione polinomiale $p : R \to R \forall a \in R$ associa $1 + 2 \cdot a = p(a)$. Nel caso dei numeri reali, questa funzione \`e una biezione. Se due polinomi hanno la stessa funzione polinomiale, allora sono lo stesso polinomio. Non \`e vero se prendo altri insiemi.

Per fare i polinomi bisogna avere un campo.

Se prendo $Z_2 = {0,1}$, ossia i resti della divisione per 2. Possiamo definire due operazioni, di somma e di prodotto.

\begin{tabular}{c|cc}
+ & 0 & 1 \\
\hline
0 & 0 & 1 \\
1 & 1 & 0
\end{tabular}

$Z / \equiv_2$, congruenza modulo 2.

\begin{tabular}{c|cc}
$\cdot$ & 0 & 1 \\
\hline
0 & 0 & 0 \\
1 & 0 & 1
\end{tabular}

Un campo \`e un gruppo rispetto a $+$ e un gruppo rispetto a $\cdot$ senza elemento neutro (0).

Posso creare i polinomi a coefficienti in $Z_2$, ossia $Z_2[x]$, ad esempio $1 + x$. La funzione polinomiale rispetto a questo polinomio \`e:
\begin{itemize}
    \item per $x = 0 \to p(0) = 1$
    \item per $x = 1 \to p(1) = 0$
\end{itemize}
Ma dato il polinomio $1 + x^2$ ho:
\begin{itemize}
    \item per $x = 0 \to p(0) = 1$
    \item per $x = 1 \to p(1) = 0$
\end{itemize}
Sono quindi due polinomi diversi che hanno la stessa funzione polinomiale.

Due polinomi sono uguali se hanno tutti i coefficienti uguali.

$(R[x], +)$ \`e un gruppo commutativo. Come funziona il +? Sommo i coefficienti. L'elemento neutro \`e il polinomio nullo, ossia il polinomio con tutti i coefficienti uguali a 0. Si indica con $\underline{0}$. Il polinomio nullo ha grado -1.

Posso definire un morfismo di gruppi su tutta sta merda.

Prendo tutti i polinomi di grado minore o uguale a due, e li indico $R_2[x]$.

Prendo l'applicazione $f : (R_2[x], +) \to (R^2, +)$ definito come segue:
\[
f(a_0 + a_1 \cdot x + a_2 \cdot x^2) = (a_2, a_1 + a_2)
\]
Quindi $1 + 2x \mapsto (0, 2)$.

Qual \`e l'immagine di questa $f$? 
\[
Im_f = \{ (r, s) \in R^2 : \exists p(x) t.c. f(p(x)) = (r,s)\} = R^2
\]
\[
(r, s) = f(0 + (s - r) x + r x^2)
\]
Troviamo il nucleo.
\[
\ker f = \{ p(x) \in R^2[x] : f(p(x)) = (0, 0)\}
\]
Quindi sono tutti i polinomi di grado 0 pi\`u il polinomio nullo, dovendo avere $a_2 = 0$ e $a_1 = 0$.

Tutti i polinomi $[p(x)] = a_0 + a_1 x + a_2 x^2$ devono potersi scrivere come $p(x) + \ker f$.
\[
f(p(x)) = (a_2, a_1 + a_2)
\]
\[
p(x) + \ker f
\]
\[
q(x) \in [p(x)] sono q(x) = p(x) + a_0
\]
con $a_0 \in \ker f$

Esercizio:

$(G, \cdot)$, $(G', \ast)$, poi abbiamo il morfismo $f : (G, \cdot) \to (G', \ast)$
\[
\ker f \subseteq G
\]
\[
Im_f \subseteq G'
\]
Caratterizzano il morfismo, questi due gruppi. f \`e un morfismo iniettivo (monomorfismo) $\Leftrightarrow \ker f = \{ 1_G\}$. Il morfismo \`e suriettivo (epimorfismo) $\Leftrightarrow$ solo se $Im_f = G'$.

Il primo caso \`e evidente, il $\ker f$ ha solo un elemento quindi la classe degli elementi con la stessa immagine $[a] = a \cdot \ker f$ ha un solo elemento, ossia $a$. Si pu\`o anche dimostrare direttamente.

$f$ \`e iniettiva per ipotesi. La tesi \`e che il nucleo \`e costituito da un solo elemento $\ker f = \{ 1_G \} $ ossia $f(1_G) = 1_{G'}$

Viceversa se $\ker f = \{ 1_G \} \Rightarrow f(a) = f(b) \Rightarrow f(a) \cdot f(b)^{-1} = 1_{G'} = f(a \cdot b^{-1}) = 1_{G'} \Rightarrow a \cdot b^{-1} = 1_G \Rightarrow a = b$

Se prendiamo $S$ sottogruppo di $G$ possiamo considerare la classe laterale destra $a S$ e la classe laterale sinistra $S a$. Lo vediamo la prossima volta. 

\subsection{Teorema di omomorfismo per i gruppi}

Dato un morfismo $f : (G, \cdot) \to (G', \ast)$, allora 
\begin{enumerate}
    \item $\varepsilon_f$ individuata da $f$ ($x \ \varepsilon_f \ y \Leftrightarrow f(x) = f(y)$) \`e una congruenza
    \item Il gruppo $(G / \varepsilon_f, \cdot)$ \`e isomorfo al sottogruppo $(Im_f, \ast)$ di $(G', \ast)$. Questo vale per ogni struttura algebrica.
\end{enumerate}

$f$ individua due sottogruppi, $\ker f = \{ u \in G : f(u) = 1_{G'} \} \le G$, $Im_f = \{ x' \in G' : \exists x \in G f(x) = x'\} \le G'$. 
\[
\forall a \in G [a] = a \cdot \ker f = \ker f \cdot a
\]
\begin{prop}
Tutte le classi hanno la stessa cardinalit\`a $|[a]| = |a \cdot \ker f| = |\ker f|$
\end{prop}
\begin{proof}
Bisogna far vedere che esiste una corrispondenza biunivoca.
\[
\forall a \in G \varphi_a : \ker f \to a \cdot \ker f
\]
\[
\forall u \in \ker f \varphi_a (u) = a \cdot u \in a \cdot \ker f
\]
$\varphi_a$ \`e biunivoca. 

$\varphi_a$ \`e iniettiva, infatti $\forall u$, $v \in \ker f $ tali che $\varphi_a (u) = \varphi_a (v)$, ho che a $\cdot u = a \cdot v \Rightarrow$ essendo in un gruppo $a$ ha l'inverso, quindi $a^{-1} \cdot a \cdot u = a^{-1} \cdot a \cdot v \Rightarrow u = v$.
\end{proof}

Vediamo cosa succede se prendiamo un sottogruppo qualunque.

Sia $S \le G$ un sottogruppo qualunque di $G$. Prendo $a \in G$ e moltiplico tutti gli elementi di $S$ per $a$, ossia faccio $a \cdot S$ e $S \cdot a$.

$a \cdot S = \{ x \in G : x = a \cdot s con s \in S \}$ \`e la classe laterale sinistra di $S$

$S \cdot a = \{ x \in G : x = s \cdot a con s \in S \}$ \`e la classe laterale destra di $S$

Per lo stesso motivo di prima tutte le classi laterali hanno la stessa cardinalit\`a di $S |a \cdot S| = |S| = |S \cdot a| \forall a \in G$

Prendiamo l'insieme di sottoinsiemi sinistri
\[
\{ a \cdot S\}_{a \in G} a \cdot S \subseteq G
\]
$a \cdot S$ non \`e un sottogruppo perch\'e non contiene l'unit\`a, ma \`e un sottoinsieme. L'insieme delle classi sinistre \`e una partizione di $G$.
\begin{enumerate}
    \item $\bigcup_{a \in G} a \cdot S = G$
    \item $a \cdot S \neq \emptyset$, infatti necessariamente $a \in a \cdot S$, essendo $a = a \cdot 1_G$ e $1_G \in S$
    \item $a \cdot S \cap a \cdot S \neq \emptyset \Rightarrow a \cdot s = b \cdot S$
\end{enumerate}
Dimostriamo il punto tre.

$a \cdot S \cap b \cdot S \neq \emptyset \Rightarrow a \cdot S = b \cdot S$

\[
c \in a \cdot S \cap b \cdot S
\]
\[
c = a \cdot s = b \cdot v con s, v \in S
\]
\`e l'ipotesi

La tesi \`e:
$x \in a \cdot S \Rightarrow = a \cdot u$ con $u \in S$
Per ipotesi ho $a = c \cdot s^{-1} \Rightarrow x = c \cdot s^{-1} \cdot u$, ma sempre per ipotesi ho che $x = b \cdot v \cdot s^{-1} \cdot u$. Quindi $x \in b \cdot S$, avendo che $(v \cdot s^{-1} \cdot u) \in S$.

Se $\{a \cdot S\}_{a \in G}$ \`e una partizione di $G$, individua in $G$ una relazione di equivalenza che indichiamo con $\lateralsx{S}$ (perch\'e che $S$ \`e una classe laterale sinistra).

Dico che due elementi sono equivalenti se sono nello stesso blocco. Ossia dati $x, y \in G$ dico $x \lateralsx{S} y \Leftrightarrow \exists a \in G x, y \in a \cdot S \Leftrightarrow x = a \cdot s$ e $y = a \cdot v$ con $s, v \in S$. Si pu\`o semplificare ulteriormente questa definizione, perch\'e se un elemento $x$ \`e nella classe posso prendere $x$ come rappresentante, e quindi dire che $x \in y \cdot S$ o che $y \in x \cdot S$ o che $x \cdot S = y \cdot S$ (sono tutte definizioni equivalenti).

Quindi posso scrivere $x \in y \cdot S$ come $x = y \cdot s$ con $s \in S$.
\[
s = y^{-1} \cdot x \in S
\]
Qual \`e la differenza con il nucleo? Nel nucleo le classi laterali coincidono, in generale no. Le classi laterali hanno la stessa cardinalit\`a ma non sono identiche.

Le due relazioni di equivalenza destra e sinistra (simboli qui) non sono uguali, e non sono congruenze.
\[
a \cdot S \neq S \cdot a
\]
Nel caso del nucleo invece abbiamo che $a \cdot \ker f = \ker f \cdot a$, e la relazione di equivalenza destra e sinistra \`e una sola, ossia $\varepsilon_f$, ed \`e una congruenza.

Facciamo un esempio. Consideriamo $S_4$

$(G, \cdot)$ e $a \in G$, possiamo indicare con $< a > =$ sottogruppo generato da $a \in G$ costitiuto da tutte le potenze generate da $a$. Per le propriet\`a delle potenze \`e un sottogruppo, infatti contiene $1_G = a^{0}$.
\[
S_4
\]
\[
< a > = \{ a^{z} : z \in \mathbb{Z} \}
\]
Contiene anche l'inverso di $a^{n}$, ossia $a^{-n}$

Prendiamo il sottogruppo generato da questa permutazione, e tutte le potenze generate da questa $\sigma$:

\begin{tabular}{cccc}
\multicolumn{4}{c}{$\sigma$} \\
1 & 2 & 3 & 4 \\
2 & 3 & 1 & 4 
\end{tabular}

\[
\sigma^{0} = id
\]
\[
\sigma^{1} = \sigma
\]
$\sigma^{2}$ cosa \`e?

\begin{tabular}{cccc}
\multicolumn{4}{c}{$\sigma^{2}$} \\
1 & 2 & 3 & 4 \\
2 & 3 & 1 & 4 \\
3 & 1 & 2 & 4
\end{tabular}

Se poi faccio $\sigma^{3}$ riottengo l'identit\`a.

\begin{tabular}{cccc}
\multicolumn{4}{c}{$\sigma^{3}$} \\
1 & 2 & 3 & 4 \\
3 & 1 & 2 & 4 \\
1 & 2 & 3 & 4 
\end{tabular}

Quindi il gruppo $H = < \sigma > = \{ 1 , \sigma, \sigma^{2} \}$. L'inversa di $\sigma$ \`e $\sigma^{2}$. $H$ \`e un gruppo finito di ordine 3.

Facciamo la relazione di equivalenza e la classe laterale.

Prendiamo due elementi equivalenti, $\mu$  e $\tau \Leftrightarrow \sigma \tau^{-1} \in H$. Quindi $\mu \tau^{-1} = \rho \in H$, quindi o l'identit\`a, o $\sigma$ o $\sigma^{2}$.

\begin{tabular}{cccc}
\multicolumn{4}{c}{$\tau$} \\
1 & 2 & 3 & 4 \\
2 & 1 & 4 & 3
\end{tabular}

\begin{tabular}{cccc}
\multicolumn{4}{c}{$\sigma \cdot \tau$} \\
1 & 2 & 3 & 4 \\
2 & 1 & 4 & 3 \\
3 & 2 & 4 & 1
\end{tabular}

Quindi $\mu$ \`e:

\begin{tabular}{cccc}
\multicolumn{4}{c}{$\mu$} \\
1 & 2 & 3 & 4 \\
3 & 2 & 4 & 1
\end{tabular}

Prendiamo $\tau'$

\begin{tabular}{cccc}
1 & 2 & 3 & 4 \\
4 & 3 & 2 & 1
\end{tabular}

Applichiamo $\sigma$

\begin{tabular}{cccc}
\multicolumn{4}{c}{$\sigma \cdot \tau'$} \\
1 & 2 & 3 & 4 \\
4 & 3 & 2 & 1 \\
4 & 1 & 3 & 2
\end{tabular}

Otteniamo $\mu'$ equivalente a $\tau'$

\begin{tabular}{cccc}
\multicolumn{4}{c}{$\mu'$} \\
1 & 2 & 3 & 4 \\
4 & 1 & 3 & 2
\end{tabular}

% Dobbiamo far vedere ora che \mu, \tau, \mu' e \tau' sono nella stessa classe.

Siamo nella classe destra, avendo fatto $\sigma \tau$ (va vista come composizione).

La congruenza mi d\`a modo di definire il prodotto fra classi. 

Dati $a, b$ nella classe 1, dati $a', b'$ nella classe 2. Le classi sono definite sulla struttura $(A, \cdot)$. Vogliamo definire il prodotto fra classi, ossia $[1] \cdot [2] = [3]$. 

La congruenza fa in modo che comunque prendo i rappresentanti i prodotti vanno sempre nella stessa classe. Quindi $a \cdot a'$ e $b \cdot b' \in [3]$.

Possiamo vedere che $\tau$ per $\tau'$ e $\mu$ per $\mu'$ non vanno nella stessa classe.

Che ordine ha il sottogruppo $H$? Lo possiamo dire per il teorema di Lagrange.

Teorema di Lagrange
$(G, \cdot)$ gruppo finito, la cardinalit\`a di $G$ si dice ordine.

Prendiamo $|G| = n$, allora se $H$ \`e un sottogruppo di $G$, l'ordine di $G$ \`e diviso dall'ordine di $H$
\[
\frac{|G|}{|H|}
\]
Questo intero \`e detto indice di $H$.

Un gruppo di ordine un numero primo ha due sottogruppi.

$G \to \{ a H\}_{a \in G}$
\`e una partizione di $G$
\[
|a H| = |H|
\]
\[
|G / \lateralsx{H}| = \frac{|G|}{|H|}
\]
\begin{defn}
un sottogruppo $N$ di $G$ si dice normale se $\forall a \in G$ $a N = N a$, ossia ogni classe laterale destra \`e uguale alla classe laterale sinistra. I nuclei dei morfismi sono sottogruppi normali.
\end{defn}
CNES affinch\'e $N$ sia normale \`e che $\forall a \in G$ e $\forall u \in N$, $a \cdot u \cdot a^{-1} \in G$. Deriva banalmente da $a \cdot N = N \cdot a$

CNES affinch\'e $N$ sia normale \`e che $\lateraldx{N}$ (o $\lateralsx{N}$) \`e una congruenza.

Tutti i nuclei dei morfismi sono sottogruppi normali.

Ogni permutazione \`e una biezione che pu\`o essere indicata sia dal punto di vista dell'occupazione sia dal punto di vista della distribuzione (come parola).

\[
\sigma \in S_8
\]

\begin{tabular}{*{8}{c}}
\multicolumn{8}{c}{$\sigma$} \\
1 & 2 & 3 & 4 & 5 & 6 & 7 & 8 \\
1 & 7 & 4 & 6 & 5 & 2 & 3 & 8
\end{tabular}

O come parola:
\[
17465238
\]
Possiamo indicare le permutazioni come composte di cicli.
\[
\sigma = (1) (2 7 3 4 6) (5) (8)
\]
$(2 7 3 4 6)$ significa che il 2 va nel 7, il 7 nel 3, il 3 nel 4 e il 4 nel 6.

$\mu (3 1) (5 4 2) (8 7 6)$ \`e un prodotto di cicli. Corrisponde alla permutazione:

\begin{tabular}{*{8}{c}}
\multicolumn{8}{c}{$\mu$} \\
1 & 2 & 3 & 4 & 5 & 6 & 7 & 8 \\
3 & 5 & 1 & 2 & 4 & 8 & 6 & 7
\end{tabular}

Le permutazioni vengono rappresentate come prodotti di cicli. I cicli di lunghezza 1 non vengono scritti, visto che ogni elemento va a finire in s\'e stesso. Quindi $\sigma = (1) (2 7 3 4 6) (5) (8)$ posso scriverla come $\sigma = (2 7 3 4 6)$.

Data una permutazione
\[
\sigma \in S_n 
\]
Definita su
\[
[n] = \{1 \dots n\}
\]
Ho la relazione di equivalenza sui cicli
$x equiv_sigma y \Leftrightarrow \exists n \in \mathbb{N} $ t.c. $ y = \sigma^{n} (x)$

\textbf{Esercizio:} questa \`e una relazione di equivalenza che divide $[n]$ in classi di equivalenza. Le classi sono i cicli.

Ad esempio, nel caso del $\sigma$ di prima, $7 = \sigma(2)$, $3 = \sigma^{2}(2)$, $4 = \sigma^{3} (2)$, $6 = \sigma^{4} (2)$.

$x$ \`e equivalente a tutti gli elementi $\sigma (x), \sigma^{2} (x), \dots \sigma^{t}(x)$ fino alla $t$-esima permutazione che torna in $x$ (altrimenti il ciclo sarebbe infinito).
\[
\mu_x : [n] \to [n]
\]
Definita come:
\[
\mu_x (y) =
\begin{cases}
y se y \notin [x] \\
\sigma(y) se y \in [x]
\end{cases}
\]
Quindi $\mu_x$ si comporta come $\mu$ nella partizione individuata da $x$, e tutti gli altri restano fissi.

Quindi $\mu_3$ \`e:

\begin{tabular}{*{8}{c}}
\multicolumn{8}{c}{$\mu_3$} \\
1 & 2 & 3 & 4 & 5 & 6 & 7 & 8 \\
3 & 2 & 1 & 4 & 5 & 6 & 7 & 8
\end{tabular}
\[
\mu_3 = \mu_1
\]
$\mu_5 = \mu_4 = \mu_2$ si comporta come $\mu$ solo sulla partizione individuata da 5:

\begin{tabular}{*{8}{c}}
\multicolumn{8}{c}{$\mu_5$} \\
1 & 2 & 3 & 4 & 5 & 6 & 7 & 8 \\
1 & 5 & 3 & 2 & 4 & 6 & 7 & 8
\end{tabular}

$\mu_6 = \mu_7 = \mu_8$ si comporta come $\mu$ solo sulla partizione individuata da 6:

\begin{tabular}{*{8}{c}}
\multicolumn{8}{c}{$\mu_6$} \\
1 & 2 & 3 & 4 & 5 & 6 & 7 & 8 \\
1 & 2 & 3 & 4 & 5 & 8 & 6 & 7
\end{tabular}

$\mu$ come prodotto di cicli si scrive come prodotto di cicli, ossia posso comporre $\mu_3$, $\mu_5$ e $\mu_6$ per ottenere $\mu$.

\begin{tabular}{*{9}{c}}
\multicolumn{9}{c}{$\mu = \mu_3 \circ \mu_5 \circ \mu_6$} \\
1 & 2 & 3 & 4 & 5 & 6 & 7 & 8 & \\
3 & 2 & 1 & 4 & 5 & 6 & 7 & 8 & $\mu_3$ \\
3 & 5 & 1 & 2 & 4 & 6 & 7 & 8 & $\mu_5$ \\
3 & 5 & 1 & 2 & 4 & 8 & 6 & 7 & $\mu_6$
\end{tabular}

Questo si chiama rappresentazione delle permutazioni come cicli disgiunti.

Non si possono omettere le parentesi, o si avrebbe una parola.

Con $\mathbb{N}$ possiamo trovare una rappresentazione standard (o canonica) dei cicli, senza parentesi.

Consideriamo i cicli (3 1 2) 5 (7 8) (4 6)

\begin{enumerate}
    \item descrivo i cicli partendo dall'elemento maggiore:
    (3 1 2) 5 (8 7) (6 4)
    \item ordino in maniera crescente in base al primo elemento
    (3 1 2) 5 (6 4) (8 7)
    \item ora posso togliere le parentesi, perch\'e so che i cicli finiscono al primo elemento non decrescente
    3 1 2 5 6 4 8 7
\end{enumerate}

Rappresentazione canonica o standard di una permutazione di $[n]$.

Trasposizioni = permutazioni che scambiano due elementi. Quindi hanno un solo ciclo di lunghezza 2 e tutti gli altri di lunghezza 1.

\begin{theorem}
Ogni permutazione si pu\`o scrivere come un prodotto di trasposizioni. Il numero di trasposizioni varia.
\end{theorem}

Se $\sigma$ si esprime come un prodotto di un numero pari di trasposizioni, allora ogni altro prodotto di trasposizioni che mi d\`a $\sigma$ ha un numero pari di trasposizioni. Vale anche con i dispari.

\begin{defn}
Una permutazione \`e pari se si esprime come prodotto di un numero pari di trasposizioni, dispari se si esprime come prodotto di un numero dispari di trasposizioni.
\end{defn}

$A_n =$ insieme delle permutazioni pari $\subseteq S_n$

\`E un sottogruppo di $S_n$?

$1 \in A_n$ (contiene l'unit\`a)

Deve essere un gruppo rispetto alle permutazioni. $\sigma, \mu \in A_n \Rightarrow (\sigma \cdot \mu) \in A_n$.

Allo stesso modo $\sigma^{-1} \in A_n$. Come si ottiene $\sigma^{-1}$?

$\sigma = \tau_{1} \dots \tau_{n}$ con $n = 2t$ e $\tau_i$ una trasposizione. L'inverso di una trasposizione \`e se stessa, ossia $\tau^{-1} = \tau$, quindi:
\[
\sigma^{-1} = \tau_n \dots \tau_1
\]
$A_n$ \`e quindi un sottogruppo di $S_n$ e si chiama gruppo alterno di ordine \label{gruppo_alterno} $n$.

L'insieme delle permutazioni dispari non \`e un sottogruppo di $S_n$ perch\'e il prodotto di due permutazioni dispari \`e una permutazione pari.

Il numero delle permutazioni dispari \`e uguale al numero delle permutazioni pari.
\[
|A_n| = \frac{n!}{2}
\]
\textbf{Esercizio:} trovare la corrispondenza biunivoca fra le permutazioni pari e le permutazioni dispari.

Trovare $F : A_n \to P_n$ e $F^{-1} : P_n \to A_n$, con $P_n$ ad indicare l'insieme delle permutazioni dispari.

Come conseguenza abbiamo quanto detto sopra, ossia siccome la cardinalit\`a dell'insieme delle permutazioni ha cardinalit\`a $n!$ la cardinalit\`a di $A_n$ \`e $\frac{n!}{2}$

L'indice di $A_n$ \`e:
\[
\frac{|S_n|}{A_n} = 2
\]
$A_n$ \`e un sottogruppo normale.

$A_n$ \`e il nucleo del morfismo $f : S_n \to \mathbb{Z}_2$:
\[
f (\sigma) = 
\begin{cases}
0 \text{ se } \sigma \text{ \`e pari} \\
1 \text{ se } \sigma \text{ \`e dispari} 
\end{cases}
\]
\`E un morfismo perch\'e 1 per 1 va in 0 e 0 per 0 va in 0, ossia una permutazione pari per una pari va in una pari, e una permutazione dispari per una dispari va in una pari.

% SONO ARRIVATO QUI A SISTEMARE

A_n = gruppo alterno su n elementi. \`E un sottogruppo di S_n.

|A_n| = \frac{n!}{2}

Esercizio: dimostrarlo

|S_n| = n!

A_n \`e il gruppo di permutazioni pari, ossia l'insieme di permutazioni esprimibili come il prodotto di un numero pari di trasposizioni.

Una trasposizione ha \textit{un} ciclo di lunghezza 2 e tutti gli altri di lunghezza 1.

D_n = insieme delle permutazioni dispari.

C'\`e una biezione F : A_n \to D_n

\forall \sigma \in A_n F(\sigma), come la definisco la biezione?

Prendiamo [n] - \{ 1 \dots n \} sui primi n numeri naturali. \sigma \`e una permutazione pari sui primi n numeri naturali. Per rendere \sigma dispari, la moltiplico per un'altra trasposizione.

F(\sigma) = (1 2) \cdot \sigma \in D_n

F \`e biunivoca. Deve esistere F^{-1} : D_n \to A_n

D_n(\sigma) = (1 2) \cdot \sigma

(F F^{-1}) (\delta) = F((1 2) \delta) = (1 2) \cdot (1 2) \delta = \delta

(F^{-1} F) (\sigma) = F^{-1}((1 2) \sigma) = (1 2) \cdot (1 2) \cdot \sigma = \sigma

Esercizio 2: ogni permutazione si esprime come prodotto di trasposizioni.

Prendiamo una permutazione \sigma. Abbiamo dimostrato che le permutazioni si possono esprimere come prodotto di cicli.

\sigma = \mu_1 \cdot \mu_2 \dots \mu_t

\sigma \`e il prodotto di t permutazioni. Ciascuna \mu_i \`e una permutazione k_i-ciclica, ossia \mu_i ha un solo ciclo di lunghezza i, tutti gli altri cicli sono di lunghezza 1.

1 2 3 4 5 6 7
2 4 1 3 6 5 7

(1 2 4 3) (5 6) = \mu_1 \cdot \mu_2

\mu_1 = (1 2 4 3)

\mu_2 = (5 6)

Quindi basta dimostrare che ogni permutazione k ciclica si pu\`o esprimere come un prodotto di trasposizioni.

Per scrivere una permutazione k ciclica come prodotto di trasposizioni, accoppio gli elementi a due a due.

(1 2 4 3) = (1 2) (2 4) (4 3)

1 2 3 4 5 6 7
1 2 4 3 5 6 7
1 4 2 3 5 6 7
2 4 1 3 5 6 7

L'ordine \`e l'ordine di composizione delle funzioni.

Quindi, dato un ciclo \mu_i = a_1 \dots a_t, lo scrivo come prodotto di trasposizioni:

\mu_i = (a_1 a_2) (a_2 a_3) \dots (a_{t-1} a_t)

Quindi ogni permutazione k_i-ciclica ha parit\`a uguale alla parit\`a di k_i - 1.

Quindi la parit\`a di \sigma = \mu_1 \cdot \mu_2 \dots \mu_t \`e:

\sum_{i = 1}^{t} (k_i - 1) = t + \sum_{i = 1}^{t} k_i

L'ordine di una permutazione \sigma \`e la cardinalit\`a del sottogruppo generato da \sigma, < \sigma >, ossia tutte le potenze di \sigma. Se prendo una permutazione \mu k-ciclica, l'ordine di \mu \`e k.

Una trasposizione ha ordine 2. < (5 6) > = \{ 1, (5 6) \}

Il sottogruppo generato da < (1 2 4 3) > = \{ 1, (1 2 4 3), (1 4) (2 3), (1 3 4 2) \}

(1 2 4 3)^2
1 2 3 4
2 4 1 3
4 3 2 1

Quindi (1 2 4 3)^2 = (1 4) (2 3)

(1 2 4 3)^2
1 2 3 4
2 4 1 3
4 3 2 1
3 1 4 2

Quindi (1 2 4 3)^3 = (1 3 4 2)

k \`e quindi \inf (t : \sigma^t = 1 con t \neq 0), \`e il pi\`u piccolo intero t per cui \sigma^t \`e l'identit\`a.

Se in generale \sigma \`e il prodotto di \mu_1 \dots \mu_t, l'ordine di \sigma \`e il mcm delle lunghezze dell'ordine dei suoi cicli.

\sigma^j = \mu_1^j \dots \mu_t^j = 1

< \sigma > = \{ \sigma^0, \sigma^1 \dots \sigma^j \}

j deve essere il mcm delle lunghezze di ciascun \mu_i

Esercizio:

H = \{ \sigma \in S_4 t.c. \sigma = 1 oppure \sigma \`e il prodotto di trasposizioni disgiunte \}. H \`e un sottogruppo di S_4?

H = \{ 1, (1 2) (3 4), (1 3) (2 4), (1 4) (2 3) \}

H contiene l'unit\`a, quindi verifica una delle propriet\`a. Verifichiamo se, data una permutazione \sigma, H deve contenere \sigma^{-1}. \`E verificato perch\'e l'inverso di un elemento \`e l'elemento stesso.

In generale l'inverso di un prodotto in un gruppo non commutativo \`e il prodotto al contrario degli inversi. Ma in questo caso sono commutativi i singoli elementi, quindi:

\sigma^{-1} = ((1 2) (3 4) )^{-1} = (3 4)^{-1} (1 2)^{-1} = (4 3) (2 1) = (1 2) (3 4)

Si chiama idempotenza.

\sigma_1 \sigma_2 = \sigma_3 = \sigma_2 \sigma_1

\sigma_1 \sigma_3 = \sigma_1 = \sigma_3 \sigma_1

\sigma_2 \sigma_3 = \sigma_2 = \sigma_3 \sigma_2

Proviamolo con \sigma_1 \sigma_2:

\sigma_1 \sigma_2
1 2 3 4
3 4 1 2
4 3 2 1

(1 4) (2 3)

H \`e un sottogruppo commutativo di un gruppo non commutativo.

L'ordine di H \`e 4. L'ordine di S_4 \`e 4! S_4 pu\`o avere sottogruppi di ordine che divide l'ordine di S_4.

I sottogruppi di H (diversi da H e dall'unit\`a) devono avere ordine 2. I sottogruppi di ordine 2 sono 3: < \sigma_1 > = \{ 1, \sigma_1\}, < \sigma_2 > = \{1, \sigma_2\}, < \sigma_3 > = \{1, \sigma_3 \}.

Se esprimo una \sigma come prodotto di cicli, il prodotto dei cicli \`e commutativo perch\'e i cicli sono disgiunti. Le permutazioni cicliche con elementi in comune non sono commutative.

Esercizio:
Determinare un elemento x di S_8 tale che a x a = a c b a b

Dove a = (1 2 3) (2 3 4) (4 5 6)
non \`e un prodotto di cicli.

b = 
1 2 3 4 5 6 7 8
2 1 5 4 3 7 6 8
dal punto di vista dell'occupazione

c = (2 8)
\`e una trasposizione.

Determinare l'ordine di x, la sua parit\`a e una decomposizione in cicli disgiunti.

Per determinare x dobbiamo fare:

a^{-1} a x a a^{-1} = a^{-1} a c b a b a^{-1}
x = c b a b a^{-1}

% SONO ARRIVATO QUI A SCRIVERE

\section{Strutture algebriche con due operazioni}

\begin{description}
    \item[Anelli] un anello \`e una struttura algebrica $(A, +, \cdot)$ t.c. 
    \begin{enumerate}
        \item La prima operazione $\left( A, + \right )$ \`e un gruppo abeliano.
        \item La seconda operazione considerata sull'insieme escluso l'elemento neutro, $(A \setminus \left \{ 0 \right \}, \cdot )$ \`e un semigruppo.
        \item $ \forall a, b, c \in A , \ a \cdot (b + c) = a \cdot b + a \cdot c $
        \item $ \forall a, b, c \in A , \ (a + b) \cdot c = a \cdot c + b \cdot c $
    \end{enumerate}
    \item[Campi] \`e un anello in cui $( A \setminus \left \{ 0 \right \}, \cdot )$ \`e un gruppo abeliano.
\end{description}

Gli interi sono un anello: $\left ( \mathbb{Z}, +, \cdot \right )$.

I razionali sono un campo: $\left ( \mathbb{Q}, +, \cdot \right )$. Anche $\mathbb{R}$ e $\mathbb{C}$ sono un campo.

% SONO ARRIVATO QUI A SISTEMARE

\subsection{Anelli}

un anello \`e una struttura algebrica con 2 operazioni (A, +, \cdot) tale che:

\begin{description}
    \item[1A] (A, +) \`e un gruppo abeliano (ossia un gruppo commutativo)
    \item[2A] (A, \cdot) \`e un semigruppo, ossia una struttura algebrica associativa
    \item[3A] valgono le propriet\`a distributive (devo scriverle entrambe perch\'e non \`e detto che le operazioni siano associative)

    \forall a b c \in A a (b + c) = ab + ac, (b + c) a = b a + c a
\end{description}

Se l'anello \`e un monoide rispetto al prodotto (ossia ha l'unit\`a), si chiama anello unitario.

Anello unitario \Rightarrow (A, \cdot) \`e un monoide.

Anello commutativo \Rightarrow (A, \cdot) \`e una struttura commutativa.

Anello privo di divisori dello 0, con 0 a indicare l'unit\`a di (A, +). Se a \cdot b = 0 allora a = 0 oppure b = 0 (oppure non esclusivo).

(\mathbb{Z}, +, \cdot) \`e un anello commutativo unitario privo di divisori dello 0.

Prendiamo tutte le funzioni R^R rispetto a + e a \cdot (R^R, +, \cdot)

(f + g) : R \to R \`e definita come (f + g) x = f(x) + g(x) 

(f \cdot g) : R \to R \`e definita in modo naturale anche questa come (f \cdot g) (x) = f(x) \cdot g(x)

Questo \`e un anello. Lo 0 di questo anello rispetto a (R^R, +) \`e \underline{0} : R \to R con \underline{0} + f = f = f + \underline{0}.

\underline{0} (x) = 0

1 : R \to R

1(x) = 1

Ha divisori dello 0. Ne facciamo un esempio, trovarne altri.

f : R \to R
f(x) =
\begin{cases}
x se x = 2n \\
0 altrimenti
\end{cases}

f \neq \underline{0}

Prendiamo una g : R \to R definita cos\`i:

g(x) =
\begin{cases}
x se x = 2n + 1 \\
0 altrimenti
\end{cases}
Anche g \neq \underline{0}

ma (f \cdot g) (x) = 0 \forall x \in X

Quindi f \cdot g = \underline{0}. Abbiamo trovato due elementi del gruppo (R^R, \cdot) il cui prodotto da \underline{0} anche se sono entrambi diversi da \underline{0}.

f e g sono due divisori dello 0.

Teorema di divisione

Dato a \in Z e n > 0 con n \in N, allora esistono due numero q, r \in Z tali che
\begin{description}
    \item a = n \cdot q + r
    \item 0 \le r < n
\end{description}
La coppia q, r \`e unica. Ossia se (q', r') soddisfa 1 e 2, allora q = q' e r = r'.

Facciamo questa dimostrazione usando il principio del buon ordinamento di N. Tutti i sottoinsiemi di N diversi dal vuoto hanno un primo elemento.

n \ge 2

Indichiamo con M = \{ m \in N tali che m = a - n \cdot q con q \in Z \}

Il resto r \`e uno degli elementi di M, in particolare il pi\`u piccolo. M \neq \emptyset, perch\'e se a > 0 \Rightarrow a \in M. Se invece a \le 0 \Rightarrow possiamo fare un piccolo trucchetto. a - n \cdot q = - a \cdot (-1 + n \cdot q) e pongo q = a,

a - n \cdot a = - a \cdot (-1 + n \cdot a)

a - n \cdot a \`e positivo, quindi - a \cdot (-1 + n a) \in M

Siccome M \`e diverso dal vuoto e \`e sottoinsieme di N
M \subseteq N
segue per il principio del buon ordinamento che M ha un primo elemento r (il pi\`u piccolo).

Quindi r = a - n q \Rightarrow a = n q + r.

Dobbiamo dimostrare la seconda propriet\`a, ossia che 0 \le r < n

Supponiamo per assurdo che n \le r \Rightarrow r = n + x con x \le r. Siccome a = n q + r, a = n q + (n + x) = n (q + 1) + x \Rightarrow x \in M e minore di r, quindi ho l'assurdo: r \`e il pi\`u piccolo elemento di M.

Se voglio fare il teorema di divisione con un numero qualunque?

Possiamo esprimere il teorema generale di divisione:

a, b \in Z con b \neq 0 allora \exist q, r \in Z tali che a = q b + r e 0 \le r < |b|. La coppia q, r \`e unica anche in questo caso. Segue come conseguenza dal teorema precedente. 

mcm(a,b) = m 
\Leftrightarrow
\begin{cases}
m \ge 0 
m = k a
m = h b
\`e il sup di a e b nel reticolo della divisibilit\`a: z = k' a e z = h' b \Rightarrow m \le z
\end{cases}

MCD(a,b) = d
\Leftrightarrow
\begin{cases}
d \ge 0
d | a (divide)
d | b
d' | a e d' | b \Rightarrow d' \le d
Quindi d \`e l'inf di a e b nel reticolo di cui sopra.
\end{cases}

Esistenza del minimo comune multiplo. Esiste per il principio del buon ordinamento.

M = \{ t \in N tali che t = k a e t = h b \}, M \`e non vuoto e ha un primo elementi, e quindi esiste il mcm.

Perch\'e \`e non vuoto? Perch\'e a \cdot b \in M. mcm(a,b) \`e il pi\`u piccolo elemento di M.

La dimostrazione per il MCD \`e pi\`u lunga.

Dati a, b \in \mathbb{Z} con b \neq 0, MCD(a, b) = MCD(|a|, |b|)

Esistenza del MCD(a, b). Supponiamo a > 0, b > 0.

Consideriamo l'insieme S_{a, b} = \{ n \in \mathbb{N}^{\ast} : n = a \cdot x + b \cdot y con x, y \in \mathbb{Z}\}, ossia n \`e combinazione lineare di a e b. x ed y sono i coefficienti della combinazione. \mathbb{N}^{\ast} = \mathbb{N} \setminus \{ 0 \}.

S_{a, b} \neq \emptyset, infatti a+b, a, b sono tutti \in S_{a,b}.

S_{a,b} ha un minimo d = MCD(a,b). Bisogna dimostrare che:
\begin{itemize}
    \item d | a e d | b
    \item z | a e z | b \Rightarrow z | d, ossia d \`e il \sup(a,b) nel reticolo (\mathbb{N}, |)
\end{itemize}
\begin{proof}
Tesi: d | a, con d = s \cdot a + t \cdot b come ipotesi, inoltre d \`e il minimo di S_{a,b}. Dobbiamo dimostrare che a = q \cdot d + r con r = 0 \Rightarrow r = a - q \cdot d \Rightarrow r = a - q \cdot (s \cdot a + t \cdot b) = a - q \cdot s \cdot a - q \cdot t \cdot b = a \cdot (1 - q \cdot s) + b \cdot (- q \cdot t)

Sappiamo che per il teorema di divisione su \mathbb{Z}, 0 \le r < d. r \`e una combinazione lineare di a, b, quindi r \in S_{a,b} se r \neq 0. Ma r deve essere pi\`u piccolo di d, quindi non pu\`o appartenere a S_{a,b} essendo d il minimo di S_{a,b}. Quindi r \`e necessariamente 0.
\end{proof}
\begin{proof}
Ipotesi: z | a e z | b. Tesi z < d quindi z | d.

a = z \cdot k e b = z \cdot h. Sapendo che d = s \cdot a + b \cdot t, posso sostituire e ottenere d = s \cdot z \cdot k + t \cdot z \cdot h, quindi d = z \cdot ( s \cdot k + t \cdot h), quindi d \`e un multiplo di z.
\end{proof}

Il MCD si calcola con l'algoritmo di Euclide delle divisioni successive.

\begin{lem}
Siano a > b > 0, allora posto d = MCD(a,b) e d' = MCD(b, r) dove a = q \cdot b + r, allora d = d'.
\end{lem}
\begin{proof}
Tesi: d | d' e d' | d, e quindi sono uguali.

d | a, d | b \Rightarrow d | ( b \cdot q + r ), ma siccome d | b , se divide la somma divide gli addendi, quindi d | r \Rightarrow d | r, d | b \Rightarrow d | d' ossia d < d'. \`E pi\`u piccolo del MCD(b, r), poich\'e divide entrambi.

d' | b, d' | r \Rightarrow d' | a = q \cdot b + r \Rightarrow d' | a, d' | b \Rightarrow d' | d
\end{proof}
Su questo lemma si basa l'algoritmo di Euclide.

Algoritmo:

a > b > 0

1 a = b \cdot q_1 + r_1  con 0 \le r_1 < b, se r_1 = 0 ho trovato il MCD(a,b) = b, altrimenti continuo
2 b = r_1 \cdot q_2 + r_2  con 0 \le r_2 < r_1, se r_2 = 0 ho trovato il MCD(a,b) = r_1, ossia l'ultimo resto non nullo, altrimenti continuo
3 r_1 = r_2 \cdot q_3 + r_3 ... il MCD(a,b) \`e l'ultimo resto non 0

Poich\'e 0 \le r_{(i+1)} < r_i, \exists n > 0 t.c. r_{n} \neq 0 e r{(n+1)} = 0. Quindi al passo n-esimo avr\`o:

n r_{(n-2)} = r{(n-1)} \cdot q_n + r_n
(n+1) r_{(n+1)} = q_{(n+1)} \cdot r_n + 0

Per il lemma sopra segue che MCD(a,b) = MCD(r_{(n-1)}, r_{n}) = r_{n}.

Possiamo ricavere l'identit\`a di B\'ezouf

d = MCD(a,b), quindi posso scriverlo come d = a \cdot s + b \cdot t con s, t \in \mathbb{Z}. L'identit\`a di B\'ezouf sono infinite.

Possiamo scrivere d = a \cdot s + b \cdot t prendendo (s + k \cdot b, t - k \cdot a) e fare la combinazione lineare con questi coefficienti:

a (s + k \cdot b) + b \cdot (t - k \cdot a) = a \cdot s + k \cdot b \cdot a + b \cdot t - k \cdot a \cdot b = d

Dimostrarlo per esercizio.

Scriviamo ogni volta il resto in funzione di a e b.

MCD(159, 42)

159 > 42
1. 159 = 43 (3) + 33     33 = 159 - 42 (3)
2. 42 = 33 (1) + 9       9 = 42 - 33
3. 33 = 9 (3) + 6        6 = 33 - 9 (3)
4. 9 = 6 + 3             3 = 9 - 6
5. 6 = 3 (2)

MCD(159, 42) = 3

Vogliamo ora trovare l'identit\`a di B\'ezouf per 3.

Possiamo scrivere 3 = 9 - 6

Sostituiamo via via i resti.

3 = 9 - 6 = 9 - 33 + 9 (3) = 9 (4) - 33 = 42 (4) - 33 (4) - 33 = 42 (4) - 33 (5) = 42 (4) - 159 (5) + 42 (15) = 42 (19) - 159 (5)

Dati a, b \in \mathbb{Z}, a e b si dicono coprimi tra loro se MCD(a,b) = 1

p \in \mathbb{N} si dice primo se p \neq 1 e \`e divisibile solo per 1 e per p.

(\mathbb{Z}, +, \cdot) \`e un anello commutativo unitario privo di divisori dello zero.

Per costruire \mathbb{Z} abbiamo preso \mathbb{N}, costruito \mathbb{N} \times \mathbb{N}, abbiamo considerato il monoide (\mathbb{N}, +) e creato il monoide (\mathbb{N} \times \mathbb{N}, +). Se prendo una struttura algebrica e faccio il prodotto ottengo una struttura algebrica che mantiene le stesse operazioni e le stesse propriet\`a.

Definisco adesso + : (\mathbb{N} \times \mathbb{N}) \times (\mathbb{N} \times \mathbb{N}) \to \mathbb{N} \times \mathbb{N} come (a,b), (c,d) \mapsto (a,b) + (c,d) = (a + c, b + d)

Per ottenere \mathbb{Z} si fa una relazione di equivalenza su \mathbb{N} \times \mathbb{N}, \rho, che \`e una congruenza rispetto a +.

(\mathbb{N} \times \mathbb{N} / \rho, +) \`e un monoide perch\'e \rho \`e una congruenza. 

(a,b) \ \rho \ (c,d) \Leftrightarrow a + d = b + c, ossia a - b = c - d.

Esercizio: dimostrare che \rho \`e una congruenza.


\mathbb{N} \times \mathbb{N} / \simequiv \mathbb{R}, ossia \`e in corrispondenza biunivoca.

[(m,n)] = 
\begin{cases}
[(m,0)] = +m
[(0,0)] = \underline{0}
[(0,m)] = -m
\end{cases}

Non \`e solo un monoide: \`e un gruppo, perch\'e ha l'inverso. [(m, 0)] + [(0, m)] = [(m, m)] = [(0,0)]

\mathbb{Z} ha un'altra operazione, \cdot, che potrei pensare dipendere da (\mathbb{N}, \cdot), ma \rho non \`e una congruenza rispetto a (\mathbb{N} \times \mathbb{N}, \cdot).

\cdot : (\mathbb{N} \times \mathbb{N}) \times (\mathbb{N} \times \mathbb{N}) \to \mathbb{N} \times \mathbb{N} tale che (a, b) \cdot (c, d) = (a \cdot c, b \cdot d) \textit{non \`e} una congruenza. Scelgo quindi un'altra operazione \cdot su \mathbb{N} \times \mathbb{N}.

Vediamo intanto che \rho non \`e una congruenza rispetto a \cdot.

Basta un controesempio: (3, 0) \cdot (2, 0) = (6, 0). Prendiamo ora un elemento nella classe [(3,0)], ad esempio (6,3), ed un elemento nella classe [(2,0)], ad esempio (4,2). Il loro prodotto (6,3) \cdot (4,2) = (24,6) \in [(18,0)] \neq [(6,0)].

Per ottenere quindi una congruenza in \mathbb{Z} devo prendere un'altra moltiplicazione in \mathbb{N} \times \mathbb{N}.

(\mathbb{N} \times \mathbb{N}, +, \cdot). + lo prendo direttamente da (\mathbb{N}, +), mentre \cdot lo prendo come

Se definisco la congruenza come (m, n) \ \rho \ (p, q) \Leftrightarrow m \cdot q = n \cdot p.

Quindi (m - n) (p - q) = m p - n p + n q - m q = m p + n q - ( n p + m q)

Se voglio esprimere questo intero m p + n q - ( n p + m q) in \mathbb{N} \times \mathbb{N}, come devo esprimerlo? Che coppia deve darmi?

(m, n) \cdot (p, q) \in [((m p + n q), (n p + m q))]

Fissato un interno n \ge 2, abbiamo definito la congruenza \equiv_n (congruenza modulo n) come

a, b \in \mathbb{Z}, a \equiv_n b \Leftrightarrow n | (a - b)

\`E una congruenza rispetto a entrambe le operazioni.

(\mathbb{Z} / \equiv_n, +, \cdot) \`e un anello commutativo e unitario. Non \`e privo di divisori dello zero.

\mathbb{Z} / \equiv_2 = \{ [0], [1] \}

Ogni classe [a] \in \mathbb{Z} / \equiv_n pu\`o essere rappresentato con il suo resto, ossia [a] = [r] dove r \`e il resto nella divisione di a per n.

Possiamo considerare \mathbb{Z}_n come l'insieme dei resti delle possibili divisioni. \mathbb{Z}_n = \{ 0, 1 \dots (n-1)\}

\varphi_n : Z / \equiv_n \to Z_n che associa ad ogni classe il resto della divisione, ossia \varphi_n [a] = r

(\mathbb{Z}_n, +, \cdot ) \`e un anello.

\mathbb{Z}_3 = \{ [0], [1], [2] \} \leftrightarrow \mathbb{Z}_3 = \{ 0, 1, 2 \}

Se il gruppo \`e finito possiamo scrivere le tavole di composizione rispetto alle operazioni.

\begin{tabular}{c|ccc}
+ & 0 & 1 & 2 \\
\hline
0 & 0 & 1 & 2 \\
1 & 1 & 2 & 0 \\
2 & 2 & 0 & 1
\end{tabular}

\begin{tabular}{c|ccc}
\cdot & 0 & 1 & 2 \\
\hline
0 & 0 & 0 & 0 \\
1 & 0 & 1 & 2 \\
2 & 0 & 2 & 1
\end{tabular}

Quando l'operazione \`e commutativa, la tavola \`e uguale rispetto alla diagonale (\`e simmetrica).

(\mathbb{Z}_3, +, \cdot) \`e sempre un anello commutativo unitario. Ha l'unit\`a rispetto al \cdot, ossia 1.

\begin{tabular}{c|cccc}
+ & 0 & 1 & 2 & 3 \\
\hline
0 & 0 & 1 & 2 & 3 \\
1 & 1 & 2 & 3 & 0 \\
2 & 2 & 3 & 0 & 1 \\
3 & 3 & 0 & 1 & 2
\end{tabular}

\begin{tabular}{c|cccc}
\cdot & 0 & 1 & 2 & 3 \\
\hline
0 & 0 & 0 & 0 & 0 \\
1 & 0 & 1 & 2 & 3 \\
2 & 0 & 2 & 0 & 2 \\
3 & 0 & 3 & 2 & 1
\end{tabular}

Sia (A, +, \cdot) un anello unitario (monoide associativo rispetto a \cdot), U(A) \`e il gruppo degli elementi invertibili di A rispetto a \cdot.

U(\mathbb{Z}) = \{ -1, 1 \}

U(\mathbb{Z}_2) = \{ 1 \}

U(\mathbb{Z}_3) = \{ 1,  2 \}

Vogliamo capire come \`e fatto in generale il gruppo degli elementi invertibili in \mathbb{Z}_n.

\begin{prop}
[a] \in U(\mathbb{Z} / \equiv_n) \Leftrightarrow a, n sono coprimi, ossia MCD(a, n) = 1
\end{prop}

\begin{cor}
U(\mathbb{Z}_n) = \{ x \in \mathbb{N} : 0 < x < n e MCD(n, x) = 1\}
\end{cor}

Z_p con p primo U(Z_p) = \{ 1 \dots (p-1)\}, ossia sono tutti quanti.

U(Z_4) = \{1, 3\}

Ipotesi: classe [a] invertibile in Z / \equiv_n \Rightarrow \exists [b] t.c. [a] \cdot [b] = [1]. 

Quindi per definizione [a \cdot b] = [a] \cdot [b] = [1]. Quindi a e b nella divisione per n hanno lo stesso resto, ossia 1, quindi a \cdot b = n \cdot q + 1, e per il teorema di B\'ezouf a \cdot b - n \cdot q = 1, ossia 1 \in S_{a, n}, ossia 1 \`e combinazione lineare di a e n. Quindi il MCD(a, n) = 1, quindi a e n sono coprimi.

Dobbiamo dimostrare il viceversa, ossia MCD(a, n) = 1 \Rightarrow [a] \`e invertibile in \mathbb{Z} / \equiv_n.

Quindi 1 \`e combinazione lineare di a e n, ossia 1 = a \cdot s + n \cdot t. Quindi passando alle classi, [1] = [a \cdot s + n \cdot t] = [a \cdot s] + [n \cdot t] = [a] \cdot [s] + [0] = [a] \cdot [s], quindi la classe rappresentata da s \`e l'inverso della classe rappresentata da a, perch\'e [a] \cdot [s] = [1].

Grazie all'identit\`a di B\'ezouf possiamo trovare l'inverso di una classe.

[s] = [a]^{-1}

Se prendiamo Z_p con p primo, U(Z_p) = Z_p - \{ 0 \}, ed \`e un gruppo, quindi (Z_p, +, \cdot) \`e un campo, ossia \`e una struttura algebrica con due operazioni.

Un campo \`e una struttura algebrica (K, +, \cdot) tale che:
\begin{itemize}
    \item (K, +, \cdot) \`e un anello commutativo unitario
    \item (K - \{0\}, \cdot) \`e un gruppo commutativo
\end{itemize}
Inoltre valgono le leggi distributive.

I campi non hanno divisori dello zero, ma gli anelli (Z_n, +, \cdot) hanno divisori dello zero. Infatti posso scrivere n = a \cdot b, siccome n non \`e primo, quindi la classe [0] = [a \cdot b] = [a] \cdot [b] entrambi diversi da 0. Due classi moltiplicate fanno la classe [0], quindi sono due divisori dello [0].

Il campo (Z_p, +, \cdot) non ha divisori dello 0. Sia a \cdot  b = 0 con a \neq 0 e b \neq 0, avendo l'inverso per ogni elemento diverso da 0 avrei che a^{-1} \cdot a \cdot b = a^{-1} \cdot 0 = 0 \Rightarrow b = 0 ma sarebbe un assurdo.

\begin{thm}[Teorema (fondamentale)]
[a] \in Z / \equiv_n, questa classe \`e invertibile \Leftrightarrow MCD(a, n) = 1, ossia se sono coprimi.
\end{thm}

Da questo teorema possiamo dedurre due corollari.

\begin{cor}\label{corollario_interi_primo}
Z_p = \{ 0, 1 \dots (p-1) \}, ossia l'anello dei resti modulo p, \`e un campo \Leftrightarrow p \`e un numero primo, ossia p \`e maggiore di 1 ed \`e divisibile solamente per 1 e per p.

Essere un campo significa che (Z_p \setminus \{ 0 \}, \cdot) con la moltiplicazione \`e un gruppo abeliano.
\end{cor}

\begin{cor}\label{corollario_interi_secondo}
p primo, p | a \cdot b \Rightarrow p | a oppure p | b
\end{cor}
\begin{proof}
Consideriamo il campo (Z_p, +, \cdot). Se p | a \cdot b, significa che [p] = [0] = [a \cdot b] = [a] \cdot [b]. Siamo in un campo, che non ha divisori dello zero. Quindi [a] = [0] oppure [b] = [0].
\end{proof}

\begin{prop}
Ogni numero naturale n \in N maggiore di 1, o \`e primo o \`e prodotto di primi.
\end{prop}
\begin{proof}
Si dimostra per induzione su n. Per n = 2 \`e vero.

Ipotesi di induzione: P(m) \`e vera \forall m \ge 2 con m < n. Dobbiamo dimostrare che P(n) \`e vera. O n \`e primo, e ho verificato P(n), oppure n non \`e primo, ossia n = a \cdot b con a < n e b < n. Per ipotesi di induzione P(a) e P(b) sono vere, quindi a \`e prodotto di primi o \`e primo, b \`e prodotto di primi o \`e primo, quindi n \`e prodotto di primi.
\end{proof}

\begin{thm}[Teorema fondamentale dell'aritmetica]
Ogni numero naturale n \ge 2 si esprime in un unico modo come prodotto di potenze di numeri primi, ossia n ha una sola fattorizzazione.

Ossia, n = p_{1}^{h_1} \dots p_{k}^{h^k} = q_{1}^{t_1} \dots q_{s}^{t_s} \Rightarrow k = s e \forall i \in [1, k] \exists j t.c. p_i^{h_i} = q_j^{t_j}.
\end{thm}
\begin{proof}
Si dimostra per induzione. Per n = 2 \`e vero.

Supponiamo come ipotesi induttiva che P(m) sia vero per ogni 2 \le m < n. Per induzione dimostriamo che P(n) \`e vera.

n = p_{1}^{h_1} \dots p_{k}^{h^k} = q_{1}^{t_1} \dots q_{s}^{t_s}

p_1 divide n, quindi divide q_{1}^{t_1} \dots q_{s}^{t_s}

p_1 | n = q_{1}^{t_1} \dots q_{s}^{t_s}

Per il corollario \ref{corollario_interi_secondo} \exists j = 1 \dots s t.c. p_1 | q_j. Ma posso ripetere lo stesso discorso per q_j.

Per il corollario \ref{corollario_interi_primo} q_j divide n, quindi q_j | n = p_{1}^{h_1} \dots p_{k}^{h^k}. Deve esistere un indice i_j tale che q_j | p_{i_j}.

p_1 | q_j | p_{i_j} \Rightarrow p_1 = q_j = p_{i_j} essendo tutti primi.

Essendo p_1 e q_j uguali, se divido n per p_1 ottengo due scomposizioni:

\frac{n}{p_1} = p_{1}^{h_1 - 1} \dots p_{k}^{h^k} = q_{1}^{t_1} \dots q_{j}^{t_j - 1}\dots q_{s}^{t_s}

Sia m = \frac{n}{p_1} < n \Rightarrow P(\frac{n}{p_1}) \`e vera per ipotesi \Rightarrow P(n) \`e vera.

Infatti per P( \frac{n}{p_1}) k = s e \forall i = 1 \dots k \exists r tale che p_i^{h_i} = q_r^{t_r}
\end{proof}

Da questo segue:
\begin{thm}
I numeri primi sono infiniti.
\end{thm}
\begin{proof}
Supponiamo per assurdo che i numeri primi siano finiti. Abbiamo la lista di numeri primi p_1 \dots p_N. Consideriamo n = (p_1 \dots p_N) + 1. Non pu\`o essere un numero primo, perch\'e non \`e nella lista. Per il teorema fondamentale deve essere il prodotto di numeri primi.

p_i | n = (p_1 \dots p_i \dots p_N) + 1 \Rightarrow p_i | (p_1 \dots p_i \dots p_N) e p_i | 1, che \`e l'assurdo.
\end{proof}

(Z_n, +, \cdot) \`e un anello. Se n \`e primo, \`e un campo, quindi non ha divisori dello zero.

Se n = a \cdot b con a < n e b < n, allora (Z_n, +, \cdot) \`e un anello con divisori dello zero.

Sia U(A) = gruppo degli elementi invertibili dell'anello A. Considerando il campo (Z_p, +, \cdot) il gruppo degli invertibili U(Z_p) = \{1, \dots,  p-1\}, quindi la cardinalit\`a |U(Z_p)| = p-1.

Vogliamo conoscere la cardinalit\`a di U(Z_n) nel caso generale.

Consideriamo la funzione \Phi : N^+ \to N^+ definita come \Phi (n) = numero degli interi minori di n e primi con n. |U(Z_n)| = \Phi(n). Infatti avevamo definito U(Z_n) come:
\[
U(Z_n) = \{ m \in Z_n : MCD(m, n) = 1 \}
\]
Calcoliamo questa funzione (detta funzione di Eulero). Il suo valore si trova con il principio di inclusione ed esclusione.

Sia n = p_{1}^{h_1} \dots p_{k}^{h_k}, se voglio conoscere \Phi(n), so che sono n meno tutti i numeri che dividono n. Chiamo D l'insieme dei numeri m < n che dividono n.
\[
\Phi(n) = n - D
\]
D si calcola con il principio di inclusione ed esclusione. 

Indichiamo con A_{p_i} = \{ m \in [n] : p_i | m \}, D = \bigcup_{i = 1}^{k} A_{p_i}. Quindi:
\[
|\bigcup_{i = 1}^{k} A_{p_i}| = \sum_{i = 1}^{k} |A_{p_i}| - \sum |A_{p_i} \cap A_{p_j}| + \sum |A_{p_i} \cap A_{p_j} \cap A_{p_k}| \dots (-1)^{k-1} |A_{p_i} \cap \dots A_{p_k} | 
\]
Sappiamo che |A_{p_i}| = \frac{n}{p_i}. Infatti m = k \cdot p_i \Rightarrow k = \frac{m}{p_i} \`e la cardinalit\`a dell'insieme dei numeri che dividono n.

Se prendiamo l'intersezione di due insiemi? |A_{p_i} \cap A_{p_j}| con i \neq j \`e:
\[
|A_{p_i} \cap A_{p_j}| = \frac{n}{p_i \cdot p_j}
\]
L'ultima intersezione ha cardinalit\`a 1. Quindi mettendo in evidenza:
\[
\Phi (n) = n - |D| = n \cdot (1 - \frac{1}{p_1}) \dots (1 - \frac{1}{p_k})
\]
\begin{thm}[Teorema di Fermat]
Se prendo a \in Z e n \ge 2 tali che MCD(a, n) = 1, allora a^{\Phi(n)} \equiv_n 1. 
\end{thm}
\begin{proof}
L'insieme degli elementi invertibili di Z_n \`e l'insieme di tutti gli h \in Z_n tali che MCD(h, n) = 1.

U(Z_n) = \{ h \in Z_n : MCD(h, n) = 1 \}

|U(Z_n)| = \Phi(n). 

Il prodotto \prod_{h \in U(Z_n)} [h \cdot a]  per definizione del prodotto fra classi \`e = \prod_{h \in U(Z_n)} [h] \cdot [a] e mettendo in evidenza a = [a]^{\Phi(n)} \cdot \prod_{h \in U(Z_n)} [h].

Siccome il MCD(h, n) = 1 e il MCD(a, n) = 1, allora il MCD(a \cdot h, n) = 1.

Quindi \prod_{h \in U(Z_n)} [h \cdot a] = \prod_{h \in U(Z_n)} [h]

Segue che [a]^{\Phi(n)} = 1 \Rightarrow a^{\Phi(n)} \equiv_n 1.
\end{proof}

\begin{cor}[Piccolo teorema di Fermat]
Dato p primo segue che a^{p-1} \equiv_p 1.
\end{cor}

\textbf{Esercizio:} calcolare le ultime 2 cifre di 81^{82}. Le ultime due cifre sono il resto modulo 100, quindi passando alle classi dobbiamo calcolare r < 100 tale che 81^{82} \equiv_{100} r, ossia il rappresentante della classe.

Usiamo il teorema di Fermat. Calcoliamo 81^{\Phi (100)} \equiv_{100} 1. \Phi(100) = 40, possiamo quindi dire che 81^{40} \equiv_{100} 1.

Quindi: 81^{82} = 81^{80 + 2} = 81^{80} \cdot 81^{2} \equiv_{100} 1 \cdot 81^{2} = 6561 \equiv_{100} 61

\subsection{Equazioni di primo grado in Z_n}

Come \`e fatta un'equazione di primo grado in Z_n?

\begin{enumerate}
    \item se la vedo in Z_n, ho: a \cdot x = b con a, b, x \in Z_n
    \item se la vedo come quoziente, ho: [a] \cdot [x] = [b] in Z / \equiv_n
    \item se la vedo in Z, ho: a \cdot x \equiv_n b con a, b, x \in Z
\end{enumerate}

\begin{oss}
Le soluzioni di 3, se esistono, sono infinite, perch\'e se s \in Z \`e una soluzione, a \cdot s \equiv_n b. Quindi ogni altro s' \in [s] \`e tale che a \cdot s' \equiv_n b.

Passando alle classi si legge come [a \cdot s] = [b] \Rightarrow [a] c\dot [s] = [b], e prendendo [s'] = [s] anche [a] \cdot [s'] = [b]
\end{oss}

In generale si distinguono due casi:
\begin{enumerate}
    \item MCD(a, n) = 1 \Rightarrow [a] \`e invertibilie in Z / \equiv_n, ossia se a < n \Rightarrow a \`e invertibile in Z_n. Come si trova la soluzione?
    \begin{itemize}
        \item Nel caso (1) a \cdot x = b in Z_n, x = a^{-1} \cdot b. La soluzione \`e unica.
        \item Nel caso (2) [a] \cdot [x] = [b], allora la soluzione [x] = [a]^{-1} \cdot [b]. La soluzione \`e unica.
        \item Nel caso (3) le soluzioni sono infinite e sono tutte congruenti a x modulo n.
    \end{itemize}
    \item MCD(a, n) = d > 1
\end{enumerate}

\begin{exmp}
3 x \equiv 2 \pmod{7}. Scrivo l'identit\`a di B\'ezouf:

1 = 3 \cdot s + 7 \cdot t = 3 \cdot (-2) + t \cdot (1)

Quindi:

[3] [x] = [2] \Rightarrow [x] = [3]^{-1} [2]

[1] = [3] \cdot [-2] + [7] = [3] \cdot [-2] + [0] = [3] \cdot [-2] = [3] \cdot [5]

[-2] = -[2], qual \`e un rappresentante della classe -[2]? -[2] \`e la classe che sommata a [2] mi d\`a [7] = [0]. [5] + [2] = [7]

Quindi [x] = [5] \cdot [2] = [10] = [3]. La soluzione \`e [3]

Un altro modo, sempre partendo dall'identit\`a di B\'ezouf, \`e:

1 = 3 \cdot s + 7 \cdot t \Rightarrow 2 = 3 \cdot 2s + 14 \cdot t. Passo alle classi e ho:

[x] = [3] [2s]
\end{exmp}

a, b \in Z e a,b > 0

S_{a,b} =  \{m \in N^+ : ax + by = m con x,y \in Z \}

Questo insieme ha un minimo d = inf S_{a,b} = MCD(a,b)

Anche d si pu\`o scrivere come combinazione lineare di a e b. d \in S, d = a s + b t con s, t \in Z. Si chiama identit\`a di B\'ezouf. Le coppie s,t sono infinite.

D_{a,b} = \{ (s, t) \in Z \times Z : d = a s + b t \}

Data una coppia (s,t), come posso trovarne un'altra? (s + k b, t - k a) \in D_{a,b} al variare di k \in Z. Andando a sostituire ho che:

a (s + kb) + b(t - ka) = as + kab + bt -kba = as + bt

Per calcolare una identit\`a di B\'ezouf si usa l'algoritmo di Euclide delle divisioni successive.

\begin{prop}
S_{a,b} \`e l'insieme di tutti i multipli di d = MCD(a,b).

S_{a,b} = \{ k d : k \in N^+ con d = MCD(a,b) \}
\end{prop}
\begin{proof}
Tesi: se m = kd \Rightarrow m \in S_{a,b}.

m = k d = k (as + bt) = a (ks) + b (kt) \Rightarrow m \in S_{a,b} perch\'e possiamo prendere x = ks e y = kt.

Viceversa se m \in S_{a,b} \Rightarrow m = kd.

m = ax + by. Essendo d = MCD(a,b), ho che a = hd e b = h' d. Quindi m = hdx + h'dy = (hx + h'y) d \Rightarrow m \`e multiplo di d.
\end{proof}
Quindi l'insieme S_{a,b} \`e l'insieme dei multipli del MCD.

Teorema di Fermat:

Se MCD(a, n) = 1 \Rightarrow a^{\Phi(n)} \equiv 1 \pmod n

Il teorema di Fermat \`e un corollario del teorema di Lagrange.

Il teorema di Lagrange dice che dato un gruppo finito, l'ordine di ogni suo sottogruppo divide l'ordine del gruppo.

Dato un gruppo G finito e S sottogruppo di G, |S| | |G| (ossia la cardinalit\`a di S divide la cardinalit\`a di G).

Perch\'e \`e una conseguenza? 

\Phi(n) = | U(Z_n) | ossia la cardinalit\`a del gruppo degli elementi invertibili rispetto a \cdot dell'anello (Z_n, +, \cdot).

G = U(Z_n)

S = < a > \`e il sottogruppo generato da a, ossia tutte le potenze di a.

< a > = \{ a^0, \dots a^t \}. Ha t + 1 elementi. Quindi a^{t+1} = 1.

Esempio:
(Z_12, +, \cdot) con Z_6 = \{0,1,2,3,4,5,6,7,8,9,10,11\}

U(Z_12) = \{ a : MCD(a, 12) = 1\} = \{ 1, 5, 7, 11 \}

L'inverso di 5 \`e 5, perch\'e 5 \cdot 5 = 25 \equiv_{12} 1. L'inverso di 7 \`e 7, l'inverso di 11 \`e 11. Abbiamo tre sottogruppi non banali di ordine 2.

Prendiamo il sottogruppo S = \{ 1, 5 \}. Dire che MCD(a, n) = 1 significa che a \in U(Z_n).

L'ordine \`e il pi\`u piccolo intero positivo che mi d\`a 1. Quindi 5^2 = 1, essendo l'ordine di S = \{ 1, 5 \} pari a 2.

Siccome l'ordine |< a >| = o | \Phi(n), \Phi(n) = k \cdot o. Quindi a^{\Phi(n)} = a^{o \cdot k} = 1^k = 1.

Importante: a \in U(Z_n) \Leftrightarrow MCD(a, n) = 1

Equazioni in Z_n di primo grado

ax = b, a, b \in Z_n, x \in Z_n

\begin{enumerate}
    \item MCD(a, n) = 1. a x = b ha una sola soluzione in Z_n x = a^{-1} \cdot b. Come si trova a^{-1}? Con l'identit\`a di B\'ezouf. Siccome MCD(a,n) = 1 \Rightarrow posso scrivere l'identit\`a di B\'ezouf per a, n. 1 = a \cdot s + n \cdot t. Passo alle classi modulo n.

    [1] = [as] + [nt] = [as] + [0] = [a] [s] \Rightarrow [a^{-1}] = [s], ma poich\'e s = n q + r ho che r = a^{-1}

    1 = as + bt
    ax = b
    b = asb + ntb
    [b] = [asb] = [a] [sb] = [a] [x]

    x = r, con sb = nq + r

    \item MCD(a, n) = d > 1

    \begin{prop}
    L'equazione a x = b ha soluzione in Z_n \Leftrightarrow d | b, altrimenti \`e incompatibile (non ha soluzioni).
    \end{prop}
    \begin{proof}
    Dimostriamo che \`e condizione necessaria.

    Se a x = b \`e compatibile (ammette soluzioni) in Z_n, ossia \exists \ s \in Z_n t.c. a \cdot s = b, allora a s - b = q n, ossia \`e un multiplo di n. Quindi b = a \cdot s - q \cdot n, ossia b \in S_{a, n}. S_{a,n} sono tutti i multipli del MCD(a,n) = d \Rightarrow b = k \cdot d.

    Dimostrare che la condizione \`e sufficiente segue il percorso inverso.
    \end{proof}
    Vediamo come si calcolano le soluzioni di ax = b \in Z_n con MCD(a,n) = d > 1 e d | b.
    \begin{oss}
    MCD(a,n) = d \Leftrightarrow MCD(\frac{a}{d}, \frac{n}{d}) = 1
    \end{oss}
    Quindi se prendo:

    \frac{a}{d} x = \frac{b}{d} in Z_{\frac{n}{d}} 

    questa equazione ricade nel caso 1 e quindi ha una sola soluzione s.

    s \`e soluzione di \frac{a}{d} x = \frac{b}{d} in Z_{\frac{n}{d}} \Leftrightarrow s \`e soluzione di a \cdot x = b in Z_n.

    Non \`e molto chiaro in questo modo. Scriviamolo come congruenza.

    \frac{a}{d} x \equiv \frac{b}{d} \pmod \frac{n}{d} \Leftrightarrow s \`e soluzione a \cdot x \equiv b \pmod n

    Chiamiamo (1) la prima parte e (2) la seconda
    \begin{proof}
    Da (1) segue che:

    \frac{a}{d} s - \frac{b}{d} = q \frac{n}{d} \Rightarrow a s - b = q n \Rightarrow vale la (2).

    Viceversa basta seguire l'ordine inverso.
    \end{proof}
    [s]_{\frac{n}{d}} \`e l'unica soluzione di [\frac{a}{d}]_{\frac{n}{d}} [x]_{\frac{n}{d}} = [\frac{b}{d}]_{\frac{n}{d}}, perch\'e il MCD(\frac{a}{d}, \frac{b}{d}) = 1

    Le soluzioni di ax = b \pmod n si ripartiscono in classi di equivalenza modulo n e precisamente si ha [s]_{\frac{n}{d}} = [s]_{n} \cup [s + \frac{n}{d}]_{n} \cup \dots \cup [s + \frac{n}{d}(d-1)]_{n}

    Ci\`o significa che le soluzioni di [a] \cdot [x] = [b] in Z / \equiv_n sono $d$, e sono le d-classi di equivalenza in cui si ripartisce l'unica soluzione [s]_{\frac{n}{d}} di [\frac{a}{d}] x = [\frac{b}{d}] in Z / \equiv_{\frac{n}{d}}

    \begin{proof}
    Dimostriamo che: [s]_{\frac{n}{d}} = [s]_{n} \cup [s + \frac{n}{d}]_{n} \cup \dots \cup [s + \frac{n}{d}(d-1)]_{n}

    t \in [s]_{\frac{n}{d}} \Leftrightarrow t - s = k \frac{n}{d}

    Quindi : t = s + k \frac{n}{d}

    Bisogna solo dimostrare che k < d

    Dividiamo per d:

    k = n \cdot d + r

    Quindi t = s + (n d + r) \frac{n}{d}, e quindi
    t = s + r \frac{n}{d}, perch\'e n \cdot d \equiv 0 \pmod n

    Quindi t \in s + r \frac{n}{d} con r < d.

    Viceversa si segue l'ordine inverso.

    Dobbiamo mostrare che le classi sono disgiunte. t \in [s + r \frac{n}{d}]_{n}, dove r < d \`e il resto della divisione di k per d. Non possono esserci due resti distinti, quindi [s + r_1 \frac{n}{d}] \cap [s + r_2 \frac{n}{d}] = \emptyset, perch\'e r_1 \neq r_2 ed entrambi r_1, r_2 < d.
    \end{proof}
\end{enumerate}

\begin{exmp}[Esempio del primo caso]
3x = 11 \pmod 25

MCD(3, 25) = 1 \Rightarrow ha una sola soluzione

Scrivo l'identit\`a di B\'ezouf
1 = 3 (-8) + 25 (1) 

Moltiplico entrambi i lati per 11:
11 = 3 (-8) (11) + 25 (11)

Passo alle classi:
[11] = [3] [-8 \cdot 11] = [3] [17] [11] \Rightarrow [x] = [17] [11] = [11 \cdot 17] = [12]

In Z ha infinite soluzioni, tutti gli interi nella classe [12].

L'unica soluzione in Z_{25} \`e x = 12.

Vedendola come classi, [a] [x] = [b] in Z / \equiv_{25} l'unica soluzione \`e la classe [12] = [x].
\end{exmp}

\begin{exmp}[Esempio del secondo caso]
200 x \equiv 62 \pmod 22

Dobbiamo anzitutto verificare la compatibilit\`a dell'equazione.

MCD(200, 22) = 2

Ha soluzioni, poich\'e 2 | 62.

Consideriamo l'equazione ottenuta dividendo tutto per 2.

\frac{200}{2} x \equiv \frac{62}{2} \pmod 22 \Rightarrow 100 x \equiv 31 \pmod 11

100 x \equiv 31 \pmod 11 ha un'unica soluzione, essendo che MCD(100, 11) = 1. Troviamo l'identit\`a di B\'ezouf 1 = 100 \cdot s + 11 \cdot t.

1 = 100 (s) + 11 (t) \Rightarrow 1 = 100 (1) + 11 (-9)

La soluzione \`e la classe [1]_{11}. Va ripartita in 2 classi modulo 22.

[1]_{22} e [1 + \frac{n}{d}]_{22} = [1 + \frac{22}{2}]_{22} = [1 + 11]_{22} = [12]_{22} 
\end{exmp}























\part{Algebra lineare}

\chapter{Spazi vettoriali}

\input{spazi_vettoriali.tex}

\chapter{Risoluzione di sistemi lineari}

\input{risoluzione.tex}

\chapter{Applicazioni lineari}

\begin{center}
\indent
\textit{Rappresentazione con matrici, diagonalizzazione.}
\end{center}

\section{Applicazione lineari: rappresentazione e diagonalizzazione}

Un'applicazione lineare \`e un'applicazione fra due spazi vettoriali $L : V \to W$ ($V$ e $W$ spazi vettoriali in $\mathbb{R}$) che conserva le operazioni:

\begin{enumerate}
    \item conserva la somma: $L (v + v') = L(v) + L(v')$. L'immagine della somma di due vettori \`e uguale alla somma delle immagini dei vettori.
    \item conserva i prodotti scalari: $L(r \cdot v) = r \cdot L(v)$. L'immagine del prodotto di un vettore per uno scalare \`e uguale al prodotto delo scalare per l'immagine del vettore.
\end{enumerate}

\textbf{Esercizio:} Trovare un esempio di applicazione lineare in uno spazio vettoriale visto.

\end{document}